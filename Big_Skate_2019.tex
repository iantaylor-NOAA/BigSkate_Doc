\documentclass[12pt,]{article}
%\usepackage{lmodern}  Melissa removed to deal with font rendering issue
\usepackage{amssymb,amsmath}
\usepackage{ifxetex,ifluatex}
\usepackage{fixltx2e} % provides \textsubscript

%Melissa removed the following section to deal with font rendering issue
%\ifnum 0\ifxetex 1\fi\ifluatex 1\fi=0 % if pdftex
%  \usepackage[T1]{fontenc}
%  \usepackage[utf8]{inputenc}
%%\else % if luatex or xelatex
%  \ifxetex
%    \usepackage{mathspec}
%  \else
%    \usepackage{fontspec}
%  \fi
%  \defaultfontfeatures{Ligatures=TeX,Scale=MatchLowercase}
%  \newcommand{\euro}{€}
%%%%%%\fi

% use upquote if available, for straight quotes in verbatim environments
\IfFileExists{upquote.sty}{\usepackage{upquote}}{}
% use microtype if available
\IfFileExists{microtype.sty}{%
\usepackage{microtype}
\UseMicrotypeSet[protrusion]{basicmath} % disable protrusion for tt fonts
}{}
\usepackage[margin=1in]{geometry}
\usepackage{hyperref}
\PassOptionsToPackage{usenames,dvipsnames}{color} % color is loaded by hyperref
\hypersetup{unicode=true,
            pdftitle={Status of Big Skate (Beringraja binoculata) Off the U.S. Pacific Coast in 2019},
            pdfborder={0 0 0},
            breaklinks=true}
\urlstyle{same}  % don't use monospace font for urls
\usepackage{graphicx,grffile}
\makeatletter
\def\maxwidth{\ifdim\Gin@nat@width>\linewidth\linewidth\else\Gin@nat@width\fi}
\def\maxheight{\ifdim\Gin@nat@height>\textheight\textheight\else\Gin@nat@height\fi}
\makeatother
% Scale images if necessary, so that they will not overflow the page
% margins by default, and it is still possible to overwrite the defaults
% using explicit options in \includegraphics[width, height, ...]{}
\setkeys{Gin}{width=\maxwidth,height=\maxheight,keepaspectratio}
\setlength{\parindent}{0pt}
\setlength{\parskip}{6pt plus 2pt minus 1pt}
\setlength{\emergencystretch}{3em}  % prevent overfull lines
\providecommand{\tightlist}{%
  \setlength{\itemsep}{0pt}\setlength{\parskip}{0pt}}
\setcounter{secnumdepth}{5}

%%% Use protect on footnotes to avoid problems with footnotes in titles
\let\rmarkdownfootnote\footnote%
\def\footnote{\protect\rmarkdownfootnote}

%%% Change title format to be more compact
\usepackage{titling}

% Create subtitle command for use in maketitle
\newcommand{\subtitle}[1]{
  \posttitle{
    \begin{center}\large#1\end{center}
    }
}

\setlength{\droptitle}{-2em}
  \title{Status of Big Skate (\emph{Beringraja binoculata}) Off the U.S. Pacific
Coast in 2019}
  \pretitle{\vspace{\droptitle}\centering\huge}
  \posttitle{\par}
  \author{}
  \preauthor{}\postauthor{}
  \date{}
  \predate{}\postdate{}


% This file contains all of the LaTeX packages you may need to compile the document
% Documentation for each package can be found onlines
\usepackage{tabularx}                                             % table environment providing flexibility
\usepackage{caption}                                              % for creating captions  
\usepackage{longtable}                                            % allows tables to span multiple pages
\usepackage{rotating}                                             % allows for sideways tables
\usepackage{float}                                                % floating environments; may not need in rmarkdown
\usepackage{placeins}                                             % keeps floats from moving
\usepackage{indentfirst}                                          % indents first paragraph of a section
\usepackage{mdwtab}                                               % continued float multi-page figure
\usepackage{enumerate}                                            % create lists
\usepackage{hyperref}                                             % highlight cross references
\hypersetup{colorlinks=true, urlcolor=blue, linktoc=page, linkcolor=blue, citecolor=blue} %define referencing colors
%\usepackage{makebox}                                             % make boxes around text
\usepackage[usenames,dvipsnames]{xcolor}                          % color name options
%\usepackage[space]{grffile}                                      % spaces in file name path
\usepackage{soul}                                                 % highlight text
\usepackage{enumitem}                                             % numbered lists
\usepackage{lineno}                                               % Line numbers; comment out for final
\usepackage{upquote}                                              % produce grave accent in latex
\usepackage{verbatim}                                             % produces verbatim results
\usepackage{fancyvrb}                                             % verbatim in a box
%\usepackage{draftwatermark}                                      % places Draft watermark in background; comment out for final
\usepackage{textcomp}                                             % fixes error with packages interfering
\usepackage{pdflscape}                                               % rotate pages - to allow for landscape longtables
%\pdfinterwordspaceon                                             % fix loss of inter word spacing
\usepackage{cmap}                                                 % fix mapping characters to unicode
\RequirePackage[linewidth = 1]{pdfcomment}                        % pdf comments
\RequirePackage[l2tabu, orthodox]{nag}                            % checks packages related to the accessibility?
%\usepackage[inline]{showlabels}                                   % show table and figure labels; comment out for final
%\RequirePackage[tagged]{accessibilityMeta}
\usepackage{booktabs}                                             % For multi-header tables
\usepackage{geometry}                                             % For landscape display

\linenumbers                                                      % specify use of line numbers


\definecolor{light-gray}{gray}{.85}                               % define light-gray as a color
%\usepackage[tagged]{accessibility-meta}

 
%\showlabels[\color{mred}]{label}

% Redefines (sub)paragraphs to behave more like sections
\ifx\paragraph\undefined\else
\let\oldparagraph\paragraph
\renewcommand{\paragraph}[1]{\oldparagraph{#1}\mbox{}}
\fi
\ifx\subparagraph\undefined\else
\let\oldsubparagraph\subparagraph
\renewcommand{\subparagraph}[1]{\oldsubparagraph{#1}\mbox{}}
\fi

\begin{document}
\maketitle


\begin{center}
\thispagestyle{empty}

\vspace{.7cm}

% \includegraphics{cover_photo}~\\[1cm]
\pdftooltip{\includegraphics{cover_photo}}{This is a fish.}

\vspace{.5cm}

Ian G. Taylor\textsuperscript{1}\\
Vladlena Gertseva\textsuperscript{1}\\
Joseph Bizzarro\textsuperscript{2}\\
Andi Stephens\textsuperscript{3}\\

\vspace{.7cm}

\small

\textsuperscript{1}Northwest Fisheries Science Center, U.S. Department of Commerce, National Oceanic and Atmospheric Administration, National Marine Fisheries Service, 2725 Montlake Boulevard East, Seattle, Washington 98112\\

\vspace{.3cm}

\textsuperscript{2}Southwest Fisheries Science Center, U.S. Department of Commerce, National Oceanic and Atmospheric Administration, National Marine Fisheries Service, 110 Shaffer Road, Santa Cruz, California 95060\\

\vspace{.3cm}

\textsuperscript{3}Northwest Fisheries Science Center, U.S. Department of Commerce, National Oceanic and Atmospheric Administration, National Marine Fisheries Service, 2032 S.E. OSU Drive Newport, Oregon 97365


\vspace{.5cm}

\vfill
DRAFT SAFE\\
Disclaimer: This information is distributed solely for the purpose of pre-dissemination
peer review under applicable information quality guidelines. It has not been formally
disseminated by NOAA Fisheries. It does not represent and should not be construed to
represent any agency determination or policy. 

\vspace{.3cm}
%Bottom of the page
%{\large \today}


\newpage{\thispagestyle{empty}}

\vspace*{\fill}
\begin{flushleft}
This report may be cited as:

Taylor, I.G., Gertseva, V., Bizzarro, J., and Stephens, A. Status of Big Skate (\emph{Beringraja binoculata}) Off the U.S. West Coast, 2019. Pacific Fishery Management Council, Portland, OR. Available from http://www.pcouncil.org/groundfish/stock-assessments/
\end{flushleft}

\maketitle

\pagenumbering{roman}
\setcounter{page}{1}
\end{center}

{
\setcounter{tocdepth}{4}
\tableofcontents
}
\setlength{\parskip}{5mm plus1mm minus1mm}
\pagebreak

\pagenumbering{arabic}

\renewcommand{\thefigure}{\alph{figure}}
\renewcommand{\thetable}{\alph{table}}

\hypertarget{executive-summary}{%
\section*{Executive Summary}\label{executive-summary}}
\addcontentsline{toc}{section}{Executive Summary}

\hypertarget{stock}{%
\subsection*{Stock}\label{stock}}
\addcontentsline{toc}{subsection}{Stock}

This assessment reports the status of the Big Skate
(\emph{Beringraja binoculata}) resource in U.S. waters off the West
Coast using data through 2018.

\hypertarget{catches}{%
\subsection*{Catches}\label{catches}}
\addcontentsline{toc}{subsection}{Catches}

Landings and estimated discards of Big Skate were reconstructed for this
assessment from historical records of other species and from species
composition data collected in the recent fishery. These reflect the
fishery from 1916-1994. The current fishery started in 1995. For records
from 1995-2017, Big Skate landings were estimated from
species-composition samples and the landings of ``Unspecified Skates''.
Beginning in 2017, Big Skate have been recorded in species-specific
landings.

In the current fishery (since 1995), annual total landings of Big Skate
have ranged between 135-528 mt, with landings in 2018 totaling 173 mt.

\FloatBarrier

\FloatBarrier

\begin{figure}
\centering
\includegraphics{r4ss/plots_mod1/catch2 landings stacked.png}
\caption{Catch history of Big Skate in the model.
\label{fig:r4ss_catches}}
\end{figure}

\begin{table}[ht]
\centering
\caption{Recent Big Skate landings (mt)} 
\label{tab:Exec_catch}
\begin{tabular}{l>{\centering}p{.6in}}
  \hline
Year & Landings \\ 
  \hline
2008 & 366.00 \\ 
  2009 & 205.70 \\ 
  2010 & 196.20 \\ 
  2011 & 268.40 \\ 
  2012 & 269.60 \\ 
  2013 & 135.00 \\ 
  2014 & 372.40 \\ 
  2015 & 331.50 \\ 
  2016 & 411.50 \\ 
  2017 & 277.60 \\ 
  2018 & 172.60 \\ 
   \hline
\end{tabular}
\end{table}

\FloatBarrier

\newpage

\hypertarget{data-and-assessment}{%
\subsection*{Data and Assessment}\label{data-and-assessment}}
\addcontentsline{toc}{subsection}{Data and Assessment}

This the first full assessment for Big Skate. It is currently managed
using an OFL which was based on a proxy for \(F_{MSY}\) and a 3-year
recent average of survey biomass. This assessment uses the newest
version of Stock Synthesis (3.30.13). The model begins in 1916, and
assumes the stock was at an unfished equilibrium that year.

\hypertarget{stock-biomass}{%
\subsection*{Stock Biomass}\label{stock-biomass}}
\addcontentsline{toc}{subsection}{Stock Biomass}

The 2018 estimated spawning biomass relative to unfished equilibrium
spawning biomass is above the target of 40\% of unfished spawning
biomass at 72.5\% (95\% asymptotic interval: \(\pm\) 55.2\%-89.7\%)
(Figure \ref{fig:RelDeplete_all}). Approximate confidence intervals
based on the asymptotic variance estimates show that the uncertainty in
the estimated spawning biomass is high.

\FloatBarrier

\begin{table}[ht]
\centering
\caption{Recent trend in beginning of the 
                                      year spawning output and depletion for
                                      the model for Big Skate.} 
\label{tab:SpawningDeplete_mod1}
\begin{tabular}{l>{\centering}p{1.3in}>{\centering}p{1.2in}>{\centering}p{1in}>{\centering}p{1.2in}}
  \hline
Year & Spawning Output (million eggs) & \~{} 95\% confidence interval & Estimated depletion & \~{} 95\% confidence interval \\ 
  \hline
2010 & 1059.250 & (425.78-1692.72) & 0.694 & (0.552-0.837) \\ 
  2011 & 1068.670 & (434.08-1703.26) & 0.700 & (0.56-0.841) \\ 
  2012 & 1073.990 & (438.95-1709.03) & 0.704 & (0.564-0.843) \\ 
  2013 & 1079.980 & (444.55-1715.41) & 0.708 & (0.57-0.846) \\ 
  2014 & 1094.970 & (458.25-1731.69) & 0.718 & (0.583-0.852) \\ 
  2015 & 1095.100 & (458.91-1731.29) & 0.718 & (0.583-0.852) \\ 
  2016 & 1097.700 & (461.69-1733.71) & 0.719 & (0.586-0.853) \\ 
  2017 & 1093.720 & (458.52-1728.92) & 0.717 & (0.583-0.851) \\ 
  2018 & 1097.080 & (461.78-1732.38) & 0.719 & (0.586-0.852) \\ 
  2019 & 1106.070 & (504.33-1707.81) & 0.725 & (0.552-0.897) \\ 
   \hline
\end{tabular}
\end{table}

\FloatBarrier

\begin{figure}
\centering
\includegraphics{r4ss/plots_mod1/ts7_Spawning_output_with_95_asymptotic_intervals_intervals.png}
\caption{Time series of spawning biomass trajectory (circles and line:
median; light broken lines: 95\% credibility intervals) for the base
case assessment model. \label{fig:Spawnbio_all}}
\end{figure}

\begin{figure}
\centering
\includegraphics{r4ss/plots_mod1/ts9_Spawning_depletion_with_95_asymptotic_intervals_intervals.png}
\caption{Estimated relative depletion with approximate 95\% asymptotic
confidence intervals (dashed lines) for the base case assessment model.
\label{fig:RelDeplete_all}}
\end{figure}

\FloatBarrier

\hypertarget{recruitment}{%
\subsection*{Recruitment}\label{recruitment}}
\addcontentsline{toc}{subsection}{Recruitment}

Recruitment was assumed to follow the Beverton-Holt stock recruit curve,
so uncertainty in estimated recruitment is due to uncertainty in
spawning biomass and the unfished equilibrium recruitment \(R_0\)
(Figure \ref{fig:Recruits_all} and Table \ref{tab:Recruit_mod1}).

\begin{table}[ht]
\centering
\caption{Recent recruitment for the model.} 
\label{tab:Recruit_mod1}
\begin{tabular}{>{\centering}p{.8in}>{\centering}p{1.6in}>{\centering}p{1.3in}}
  \hline
Year & Estimated Recruitment (1,000s) & \~{} 95\% confidence interval \\ 
  \hline
2010 & 3435.91 & (2128.69 - 5545.9) \\ 
  2011 & 3450.01 & (2142.11 - 5556.47) \\ 
  2012 & 3457.92 & (2149.79 - 5562.03) \\ 
  2013 & 3466.77 & (2158.45 - 5568.12) \\ 
  2014 & 3488.68 & (2179.48 - 5584.31) \\ 
  2015 & 3488.86 & (2180.18 - 5583.09) \\ 
  2016 & 3492.63 & (2184.26 - 5584.72) \\ 
  2017 & 3486.86 & (2179.33 - 5578.88) \\ 
  2018 & 3491.73 & (2184.37 - 5581.57) \\ 
  2019 & 3504.69 & (2186.12 - 5618.57) \\ 
   \hline
\end{tabular}
\end{table}

\FloatBarrier

\begin{figure}
\centering
\includegraphics{r4ss/plots_mod1/ts11_Age-0_recruits_(1000s)_with_95_asymptotic_intervals.png}
\caption{Time series of estimated Big Skate recruitments for the
base-case model with 95\% confidence or credibility intervals.
\label{fig:Recruits_all}}
\end{figure}

\FloatBarrier

\hypertarget{exploitation-status}{%
\subsection*{Exploitation status}\label{exploitation-status}}
\addcontentsline{toc}{subsection}{Exploitation status}

Harvest rates estimated by the base model indicate catch levels have
been below the limits that would be associated with the SPR = 50\%
target (Table \ref{tab:SPR_Exploit_mod1} and Figure \ref{fig:SPR_all}).

\FloatBarrier

\begin{table}[ht]
\centering
\caption{Recent trend in spawning potential 
                                        ratio and exploitation for Big Skate in the model.  Relative fishing intensity is (1-SPR) 
                                        divided by 50\% (the SPR target) and exploitation 
                                        is catch divided by age 2+ biomass.} 
\label{tab:SPR_Exploit_mod1}
\begin{tabular}{l>{\centering}p{1in}>{\centering}p{1.2in}>{\centering}p{1in}>{\centering}p{1.2in}}
  \hline
Year & Relative fishing intensity & \~{} 95\% confidence interval & Exploitation rate & \~{} 95\% confidence interval \\ 
  \hline
2009 & 0.23 & (0.12-0.34) & 0.01 & (0.01-0.02) \\ 
  2010 & 0.22 & (0.11-0.32) & 0.01 & (0.01-0.02) \\ 
  2011 & 0.29 & (0.15-0.42) & 0.02 & (0.01-0.02) \\ 
  2012 & 0.29 & (0.15-0.42) & 0.02 & (0.01-0.02) \\ 
  2013 & 0.15 & (0.08-0.22) & 0.01 & (0-0.01) \\ 
  2014 & 0.39 & (0.22-0.56) & 0.02 & (0.01-0.03) \\ 
  2015 & 0.35 & (0.19-0.5) & 0.02 & (0.01-0.03) \\ 
  2016 & 0.43 & (0.24-0.61) & 0.02 & (0.01-0.04) \\ 
  2017 & 0.30 & (0.16-0.44) & 0.02 & (0.01-0.02) \\ 
  2018 & 0.19 & (0.1-0.28) & 0.01 & (0.01-0.01) \\ 
   \hline
\end{tabular}
\end{table}

\FloatBarrier

\begin{figure}
\centering
\includegraphics{r4ss/plots_mod1/SPR2_minusSPRseries.png}
\caption{Estimated spawning potential ratio (SPR) for the base-case
model. One minus SPR is plotted so that higher exploitation rates occur
on the upper portion of the y-axis. The management target is plotted as
a red horizontal line and values above this reflect harvests in excess
of the overfishing proxy based on the SPR\textsubscript{50\%} harvest
rate. The last year in the time series is 2018. \label{fig:SPR_all}}
\end{figure}

\FloatBarrier

\hypertarget{ecosystem-considerations}{%
\subsection*{Ecosystem Considerations}\label{ecosystem-considerations}}
\addcontentsline{toc}{subsection}{Ecosystem Considerations}

In this assessment, ecosystem considerations were not explicitly
included in the analysis.\\
This is primarily due to a lack of relevant data and results of analyses
(conducted elsewhere) that could contribute ecosystem-related
quantitative information for the assessment.

\hypertarget{reference-points}\)), and well above the minimum stock size
threshold (\(SB_{25\%}\)). The estimated relative depletion level for
the base model in 2019 is 72.5\% (95\% asymptotic interval: \(\pm\)
55.2\%-89.7\%, corresponding to an unfished spawning biomass of 1106.07
million eggs (95\% asymptotic interval: 504.33-1707.81 million eggs) of
spawning biomass in the base model (Table \ref{tab:Ref_pts_mod1}).
Unfished age 1+ biomass was estimated to be 2,426 mt in the base case
model. The target spawning biomass (\(SB_{40\%}\)) is 610 million eggs,
which corresponds with an equilibrium yield of 558 mt. Equilibrium yield
at the proxy \(F_{MSY}\) harvest rate corresponding to \(SPR_{50\%}\) is
466 mt (Figure \ref{fig:Yield_all}).

\FloatBarrier

\begin{table}[ht]
\centering
\caption{Summary of reference 
                                      points and management quantities for the 
                                      base case model.} 
\label{tab:Ref_pts_mod1}
\begin{tabular}{>{\raggedright}p{4.1in}>{\raggedleft}p{.62in}>{\raggedleft}p{.62in}>{\raggedleft}p{.62in}}
  \hline
\textbf{Quantity} & \textbf{Estimate} & \textbf{Low 2.5\%  limit} & \textbf{High 2.5\%  limit} \\ 
  \hline
Unfished spawning output (million eggs) & 1,526 & 891 & 2,161 \\ 
  Unfished age 1+ biomass (mt) & 2,426 & 1,583 & 3,269 \\ 
  Unfished recruitment ($R_{0}$) & 4,004 & 2,395 & 5,612 \\ 
  Spawning output(2018 million eggs) & 1,097 & 462 & 1,732 \\ 
  Depletion (2018) & 0.719 & 0.586 & 0.852 \\ 
  \textbf{$\text{Reference points based on } \mathbf{SB_{40\%}}$} &  &  &  \\ 
  Proxy spawning output ($B_{40\%}$) & 610 & 373 & 848 \\ 
  SPR resulting in $B_{40\%}$ ($SPR_{B40\%}$) & 0.625 & 0.625 & 0.625 \\ 
  Exploitation rate resulting in $B_{40\%}$ & 0.047 & 0.043 & 0.051 \\ 
  Yield with $SPR_{B40\%}$ at $B_{40\%}$ (mt) & 558 & 362 & 754 \\ 
  \textbf{\textit{Reference points based on SPR proxy for MSY}} &  &  &  \\ 
  Spawning output & 305 & 187 & 424 \\ 
  $SPR_{proxy}$ & 0.5 &  &  \\ 
  Exploitation rate corresponding to $SPR_{proxy}$ & 0.069 & 0.063 & 0.075 \\ 
  Yield with $SPR_{proxy}$ at $SB_{SPR}$ (mt) & 466 & 303 & 629 \\ 
  \textbf{\textit{Reference points based on estimated MSY values}} &  &  &  \\ 
  Spawning output at $MSY$ ($SB_{MSY}$) & 578 & 352 & 804 \\ 
  $SPR_{MSY}$ & 0.612 & 0.608 & 0.615 \\ 
  Exploitation rate at $MSY$ & 0.049 & 0.045 & 0.053 \\ 
  Dead Catch $MSY$ (mt) & 559 & 363 & 755 \\ 
  Retained Catch $MSY$ (mt) & 517 & 337 & 698 \\ 
   \hline
\end{tabular}
\end{table}

\FloatBarrier

\hypertarget{management-performance}{%
\subsection*{Management Performance}\label{management-performance}}
\addcontentsline{toc}{subsection}{Management Performance}

\begin{table}[ht]
\centering
\caption{Recent trend in total catch (mt) relative to the 
                              management guidelines. Big skate was
                              managed in the Other Species complex in 2013 and 2014,
                              designated an Ecosystem Component species in 2015 and
                              2016, and managed with stock-specific harvest
                              specifications since 2017.} 
\label{tab:mnmgt_perform}
\scalebox{0.9}{
\begin{tabular}{>{\raggedleft}p{1in}>{\centering}p{1in}>{\centering}p{1in}>{\centering}p{1in}>{\centering}p{1in}}
  \hline
Year & OFL (mt; ABC prior to 2011) & ABC (mt) & ACL (mt; OY prior to 2011) & Estimated total catch (mt) \\ 
  \hline
\textbf{2009} &  &  &  & 205.70 \\ 
  \textbf{2010} &  &  &  & 196.20 \\ 
  \textbf{2011} &  &  &  & 268.40 \\ 
  \textbf{2012} &  &  &  & 269.60 \\ 
  \textbf{2013} & 458.00 & 317.90 & 317.90 & 135.00 \\ 
  \textbf{2014} & 458.00 & 317.90 & 317.90 & 372.40 \\ 
  \textbf{2015} &  &  &  & 331.50 \\ 
  \textbf{2016} &  &  &  & 411.50 \\ 
  \textbf{2017} & 541.00 & 494.00 & 494.00 & 277.60 \\ 
  \textbf{2018} & 541.00 & 494.00 & 494.00 & 172.60 \\ 
  \textbf{2019} & 541.00 & 494.00 & 494.00 &  \\ 
  \textbf{2020} & 541.00 & 494.00 & 494.00 &  \\ 
   \hline
\end{tabular}
}
\end{table}

\FloatBarrier

\hypertarget{unresolved-problems-and-major-uncertainties}{%
\subsection*{Unresolved Problems and Major
Uncertainties}\label{unresolved-problems-and-major-uncertainties}}
\addcontentsline{toc}{subsection}{Unresolved Problems and Major
Uncertainties}

\FloatBarrier

\hypertarget{decision-table}{%
\subsection*{Decision Table}\label{decision-table}}
\addcontentsline{toc}{subsection}{Decision Table}

\begin{table}[ht]
\centering
\caption{Projections of landings, total mortality, OFL, and ACL values
              based on an SPR target of 50%, a P* of 0.45, and a time-varying
              Category 2 Sigma which creates the buffer shown in the right-hand
              column.} 
\label{tab:OFL_projection}
\begin{tabular}{rrrrrr}
  \hline
Year & Landings (mt) & Estimated total mortality (mt) & OFL (mt) & ACL (mt) & Buffer \\ 
  \hline
2019 & 313.16 & 336.35 & 541.00 & 494.00 & 1.00 \\ 
  2020 & 313.16 & 336.32 & 541.00 & 494.00 & 1.00 \\ 
  2021 & 1042.23 & 1119.74 & 1275.51 & 1119.75 & 0.87 \\ 
  2022 & 987.51 & 1062.58 & 1222.62 & 1062.58 & 0.86 \\ 
  2023 & 942.80 & 1015.91 & 1179.51 & 1015.91 & 0.86 \\ 
  2024 & 906.41 & 977.59 & 1145.41 & 977.59 & 0.85 \\ 
  2025 & 876.49 & 945.64 & 1118.21 & 945.64 & 0.84 \\ 
  2026 & 850.59 & 917.76 & 1095.36 & 917.76 & 0.83 \\ 
  2027 & 828.05 & 893.39 & 1075.04 & 893.39 & 0.83 \\ 
  2028 & 805.87 & 869.37 & 1056.06 & 869.37 & 0.82 \\ 
  2029 & 784.60 & 846.33 & 1037.94 & 846.33 & 0.81 \\ 
  2030 & 764.95 & 825.07 & 1020.44 & 825.07 & 0.80 \\ 
   \hline
\end{tabular}
\end{table}
\begin{table}[ht]
\centering
\caption{Summary of 10-year 
                                             projections beginning in 2020 
                                             for alternate states of nature based on 
                                             an axis of uncertainty for the model.  Columns range over low, mid, and high
                                             states of nature, and rows range over different 
                                             assumptions of catch levels. An entry of "--" 
                                             indicates that the stock is driven to very low 
                                             abundance under the particular scenario.} 
\label{tab:Decision_table_mod1}
\scalebox{0.85}{
\begin{tabular}{l|cc|>{\centering}p{.7in}c|>{\centering}p{.7in}c|>{\centering}p{.7in}c}
   \multicolumn{3}{c}{}  &  \multicolumn{2}{c}{} 
                               & \multicolumn{2}{c}{\textbf{States of nature}} 
                               & \multicolumn{2}{c}{} \\
  \multicolumn{3}{c}{}  &  \multicolumn{2}{c}{Low M 0.05} 
                               & \multicolumn{2}{c}{Base M 0.07} 
                               &  \multicolumn{2}{c}{High M 0.09} \\
 \hline
 & Year & Catch & Spawning Output & Depletion & Spawning Output & Depletion & Spawning Output & Depletion \\ 
  \hline
 & 2019 & - & - & - & - & - & - & - \\ 
   & 2020 & - & - & - & - & - & - & - \\ 
   & 2021 & - & - & - & - & - & - & - \\ 
  40-10 Rule,  & 2022 & - & - & - & - & - & - & - \\ 
  Low M & 2023 & - & - & - & - & - & - & - \\ 
   & 2024 & - & - & - & - & - & - & - \\ 
   & 2025 & - & - & - & - & - & - & - \\ 
   & 2026 & - & - & - & - & - & - & - \\ 
   & 2027 & - & - & - & - & - & - & - \\ 
   & 2028 & - & - & - & - & - & - & - \\ 
   \hline
 & 2019 & - & - & - & - & - & - & - \\ 
   & 2020 & - & - & - & - & - & - & - \\ 
   & 2021 & - & - & - & - & - & - & - \\ 
  40-10 Rule & 2022 & - & - & - & - & - & - & - \\ 
   & 2023 & - & - & - & - & - & - & - \\ 
   & 2024 & - & - & - & - & - & - & - \\ 
   & 2025 & - & - & - & - & - & - & - \\ 
   & 2026 & - & - & - & - & - & - & - \\ 
   & 2027 & - & - & - & - & - & - & - \\ 
   & 2028 & - & - & - & - & - & - & - \\ 
   \hline
 & 2019 & - & - & - & - & - & - & - \\ 
   & 2020 & - & - & - & - & - & - & - \\ 
   & 2021 & - & - & - & - & - & - & - \\ 
  40-10 Rule, & 2022 & - & - & - & - & - & - & - \\ 
  High M & 2023 & - & - & - & - & - & - & - \\ 
   & 2024 & - & - & - & - & - & - & - \\ 
   & 2025 & - & - & - & - & - & - & - \\ 
   & 2026 & - & - & - & - & - & - & - \\ 
   & 2027 & - & - & - & - & - & - & - \\ 
   & 2028 & - & - & - & - & - & - & - \\ 
   \hline
 & 2019 & - & - & - & - & - & - & - \\ 
   & 2020 & - & - & - & - & - & - & - \\ 
   & 2021 & - & - & - & - & - & - & - \\ 
  Average & 2022 & - & - & - & - & - & - & - \\ 
  Catch & 2023 & - & - & - & - & - & - & - \\ 
   & 2024 & - & - & - & - & - & - & - \\ 
   & 2025 & - & - & - & - & - & - & - \\ 
   & 2026 & - & - & - & - & - & - & - \\ 
   & 2027 & - & - & - & - & - & - & - \\ 
   & 2028 & - & - & - & - & - & - & - \\ 
   \hline
\end{tabular}
}
\end{table}

\newgeometry{hmargin=1in,vmargin=1in}

\begin{landscape}


\begin{table}[ht]
\centering
\caption{Base case results summary.} 
\label{tab:base_summary}
\scalebox{0.6}{
\begin{tabular}{r>{\centering}p{1.1in}>{\centering}p{1.1in}>{\centering}p{1.1in}>{\centering}p{1.1in}>{\centering}p{1.1in}>{\centering}p{1.1in}>{\centering}p{1.1in}>{\centering}p{1.1in}>{\centering}p{1.1in}>{\centering}p{1.1in}}
  \hline
Quantity & 2010 & 2011 & 2012 & 2013 & 2014 & 2015 & 2016 & 2017 & 2018 & 2019 \\ 
  \hline
Landings (mt) &  313.160 &  313.160 & 1042.228 &  987.509 &  942.796 &  906.409 &  876.485 &  850.594 &  828.055 &  805.865 \\ 
  Total Est. Catch (mt) &  336.345 &  336.325 & 1119.745 & 1062.581 & 1015.907 &  977.592 &  945.644 &  917.761 &  893.393 &  869.365 \\ 
  OFL (mt) &  541.00 &  541.00 & 1275.51 & 1222.62 & 1179.51 & 1145.41 & 1118.21 & 1095.36 & 1075.04 & 1056.06 \\ 
  ACL (mt) &  494.000 &  494.000 & 1119.750 & 1062.580 & 1015.910 &  977.592 &  945.643 &  917.762 &  893.393 &  869.365 \\ 
   \hline
(1-$SPR$)(1-$SPR_{50\%}$) & 0.22 & 0.29 & 0.29 & 0.15 & 0.39 & 0.35 & 0.43 & 0.30 & 0.19 &  \\ 
   \hline
Exploitation rate & 0.01 & 0.02 & 0.02 & 0.01 & 0.02 & 0.02 & 0.02 & 0.02 & 0.01 &  \\ 
  Age 1+ biomass (mt) & 17752.6 & 17914.2 & 18070.4 & 18140.7 & 18203.3 & 18389.4 & 18320.0 & 18306.6 & 18214.4 & 18273.9 \\ 
   \hline
Spawning Output & 1059.2 & 1068.7 & 1074.0 & 1080.0 & 1095.0 & 1095.1 & 1097.7 & 1093.7 & 1097.1 & 1106.1 \\ 
  ~95\% CI & (425.78-1692.72) & (434.08-1703.26) & (438.95-1709.03) & (444.55-1715.41) & (458.25-1731.69) & (458.91-1731.29) & (461.69-1733.71) & (458.52-1728.92) & (461.78-1732.38) & (504.33-1707.81) \\ 
   \hline
Depletion & 0.7 & 0.7 & 0.7 & 0.7 & 0.7 & 0.7 & 0.7 & 0.7 & 0.7 & 0.7 \\ 
  ~95\% CI & (0.552-0.837) & (0.56-0.841) & (0.564-0.843) & (0.57-0.846) & (0.583-0.852) & (0.583-0.852) & (0.586-0.853) & (0.583-0.851) & (0.586-0.852) & (0.552-0.897) \\ 
   \hline
Recruits & 3435.91 & 3450.01 & 3457.92 & 3466.77 & 3488.68 & 3488.86 & 3492.63 & 3486.86 & 3491.73 & 3504.69 \\ 
  ~95\% CI & (2128.69 - 5545.9) & (2142.11 - 5556.47) & (2149.79 - 5562.03) & (2158.45 - 5568.12) & (2179.48 - 5584.31) & (2180.18 - 5583.09) & (2184.26 - 5584.72) & (2179.33 - 5578.88) & (2184.37 - 5581.57) & (2186.12 - 5618.57) \\ 
   \hline
\end{tabular}
}
\end{table}

\end{landscape}
\restoregeometry

\begin{figure}
\centering
\includegraphics{r4ss/plots_mod1/yield1_yield_curve.png}
\caption{Equilibrium yield curve for the base case model. Values are
based on the 2018 fishery selectivity and with steepness fixed at 0.718.
\label{fig:Yield_all}}
\end{figure}

\FloatBarrier

\newpage

\hypertarget{research-and-data-needs}{%
\subsection*{Research and Data Needs}\label{research-and-data-needs}}
\addcontentsline{toc}{subsection}{Research and Data Needs}

We recommend the following research be conducted before the next
assessment:

\begin{enumerate}

\item \textbf{Data!}: 

\item \textbf{xxxx}:

\item \textbf{xxxx}:

\item \textbf{xxxx}:

\item \textbf{xxxx}:

\end{enumerate}

\FloatBarrier

\newpage
\renewcommand{\thefigure}{\arabic{figure}}
\renewcommand{\thetable}{\arabic{table}}
\setcounter{figure}{0}
\setcounter{table}{0}

\newpage
\renewcommand{\thefigure}{\arabic{figure}}
\renewcommand{\thetable}{\arabic{table}}
\setcounter{figure}{0}
\setcounter{table}{0}

\newpage

\hypertarget{fishery-data}{%
\section{Fishery Data}\label{fishery-data}}

\hypertarget{data}{%
\subsection{Data}\label{data}}

Data used in the Big Skate assessment are summarized in Figure
\ref{fig:data_plot}. Descriptions of the data sources are in the
following sections.

\hypertarget{fishery-landings-and-discards}{%
\subsection{Fishery Landings and
discards}\label{fishery-landings-and-discards}}

Catch information for Big Skate is very limited, in part because the
requirement to sort landings of Big Skate in the shore-based Individual
Fishing Quota fishery from landings in the ``Unidentified Skate''
category was not implemented until June 2015. The historical catch of
Big Skate therefore relies on the historical reconstruction of the
landings of all skates as well as an analysis of discards of Longnose
Skate. The estimated landings for each state and the tribal fishery are
provided in Table \ref{tab:Reconstructed_Landings_byState} and shown in
Figure \ref{fig:catch_by_state}.

\hypertarget{washington-commercial-skate-landings-reconstruction}{%
\subsubsection{Washington Commercial Skate Landings
Reconstruction}\label{washington-commercial-skate-landings-reconstruction}}

Estimates of landings of Big Skate in Washington state were estimated as
a fraction of total skate landings as described in (Gertseva, V.
\protect\hyperlink{ref-Gertseva2019}{2019}\protect\hyperlink{ref-Gertseva2019}{a}).
The approached relied on trawl survey estimates of depth distributions
for each species, combined with logbook estimates of fishing depths in
each year.

The WCGBT Survey data was used to estimate proportions of longnose and
big skates by depth (aggregated into 100m bins) and year for the period
of the survey (between 2003 and 2018). Big Skate were primarily found in
the 0--100m and 100--200m. Trawl logbook data include information on the
amount of retained catch of skate (all species combined) within each
haul as well depth of catch. The proportion of Big Skate for each depth
bin was assigned to the skate catch for each haul within those depth
bins and summed to get a total for each year. When survey skate
information was available (2003-2018), survey skate proportions were
applied by depth and year to account for inter-annual variability in
those proportions. Prior to 2003, average proportions from 2003-2007
within each depth bin were applied.

These estimated annual proportion of Big Skate relative to all skates
from the logbook analysis was then applied to total Washington skate
landings by year (provided by WDFW) to account for landings that weren't
included in the available logbook data. Prior to 1987 (when no logbook
data were available), the average proportion Big Skate within the
combined skate category, calculated from 1987-1992 logbook data, was
applied to total skate landings in Washington. Estimated Big Skate
landings provided by WDFW were used for the period from 2004 forward.

\hypertarget{oregon-commercial-skate-landings-reconstruction}{%
\subsubsection{Oregon Commercial Skate Landings
Reconstruction}\label{oregon-commercial-skate-landings-reconstruction}}

Oregon Department of Fish and Wildlife (ODFW) provided newly
reconstructed commercial landings for all observed skate species for the
2019 assessment cycle (1978 -- 2018). In addition, the methods were
reviewed at a pre-assessment workshop. Historically, skates were landed
as a single skate complex in Oregon. In 2009, longnose skates were
separated into their own single-species landing category, and in 2014,
big skates were also separated. The reconstruction methodology differed
by these three time blocks in which species composition collections
diverged (1978 -- 2008; 2009 -- 2014; 2015 -- 2018).

Species compositions of skate complexes from commercial port sampling
are available throughout this time period but are generally limited,
which precluded the use of all strata for reconstructing landings.
Quarter and port were excluded, retaining gear type, PMFC area, and
market category for stratifying reconstructed landings within the three
time blocks. Bottom trawl gear types include multiple bottom trawl
gears, and account for greater than 98\% of skate landings . Minor gear
types include primarily bottom longline gear, but also include mid-water
trawl, hook and line, shrimp trawl, pot gear and scallop dredge.

For bottom trawl gears, trawl logbook areas and adjusted skate catches
were matched with strata-specific species compositions. In Time Block 1
(1978 -- 2008), all bottom trawl gear types were aggregated due to a
lack of specificity in the gear recorded on the fish tickets. However,
in Time Blocks 2 and 3, individual bottom trawl gear types were
retained. Some borrowing of species compositions was required (31\% of
strata) and when necessary, borrowed from the closest area or from the
most similar gear type . Longline gear landings were reconstructed in a
similar fashion as to bottom trawl and required some borrowing among
strata as well (25\%).

Due to insufficient species compositions, mid-water trawl landings were
reconstructed using a novel depth-based approach. Available compositions
indicate that the proportion by weight of big skates within a
composition drops to zero at approximately 100 fathoms, and an inverse
relationship is observed for longnose skate, where the proportion by
weight is consistently one beyond 100 -- 150 fathoms . Complex-level
landings were assigned a depth from logbook entries and these species
specific depth associations were used to parse out landings by species.
The approach differed somewhat by time block . Landings from shrimp
trawls were handled using a similar methodology. Finally, very minor
landings from hook and line, pot gear and scallop dredges were assigned
a single aggregated species composition, as they lack any gear-specific
composition samples. Landings from within a time block were apportioned
by year using the proportion of the annual ticket landings.

Results indicate that the species-specific landings from this
reconstruction are very similar to those from Oregon's commercial catch
reconstruction (Karnowski et al.~2014) during the overlapping years but
cover a greater time period with methodology more applicable to skates
in particular. ODFW intends to incorporate reconstructed skate landings
into PacFIN in the future (A. Whitman, ODFW; pers. comm.).

\hypertarget{california-catch-reconstruction}{%
\subsubsection{California Catch
Reconstruction}\label{california-catch-reconstruction}}

A reconstruction of historical skate landings from California waters was
developed for the 1916--2017 time period using a combination of
commercial catch data (spatially explicit block summary catches and port
sample data from 2009-2017) and fishery-independent survey data
(Bizzarro, J. \protect\hyperlink{ref-Bizzarro2019}{2019}). Virtually all
landings in California were of ``unspecified skate'' until
species-composition sampling of skate market categories began in 2009.

From 2009 through 2017, catch estimates were based on these market
category species-composition samples, and the average of those
species-compositions was hindcast to 2002, based on the assumption that
those data were representative of the era of large area closures in the
post-2000 period.

For the period from 1936-1980, spatially explicit landings data (the
California Department of Fisheries and Wildlife (CDFW) block summary
data) were merged with survey data to provide species-specific
estimates.

For years 1981-2001, a ``blended'' product of these two approaches was
taken, in which a linear weighting scheme blended the two sets of catch
estimates through that period. Landings estimates were also scaled
upwards by an expansion factor for skates landed as ``dressed'' based on
fish ticket data. Prior to 1981 these data had not been reported and
skate landings were scaled by the ``average'' percentage landed as
dressed in the 1981-1985 time period, but by the late 1980s nearly all
skates were landed round.

As no spatial information on catch is available from 1916-1930, and the
block summary data were very sparse in the first few years of the CDFW
fish ticket program (1931--1934), spatial information from the late
1930's was used to hindcast to the 1916--1935 time period. However,
since Washington and Oregon did not have catch estimates for this year
period, the California estimates of catch prior to 1938 were not used as
they were subsumed into an estimated of the total catch across all
states increasing linearly from 1916 to 1950.

\hypertarget{tribal-catch-in-washington}{%
\subsubsection{Tribal Catch in
Washington}\label{tribal-catch-in-washington}}

Tribal catch of Big Skate was provided by WDFW as all landings took
place in Washington State. The landings were estimated from limited
state sampling of species compositions in combined skate category.
Anecdotal evidence suggest that most of the catch in tribal fishery is
retained, and discard is minimal.

\hypertarget{fishery-discards}{%
\subsubsection{Fishery Discards}\label{fishery-discards}}

Fishery discards of Big Skate are highly uncertain. The method used to
estimate discards for Longnose Skate was based on a strong correlation
(R\textsuperscript{2} = 95.7\%) between total mortality of that species,
and total mortality of Dover Sole for the years 2009--2017 during which
Longnose were landed separately from other skates. In contrast, the
sorting requirement for Big Skate occurred too recently to provide an
adequate range of years for this type of correlation. Furthermore, there
is greater uncertainty in the total mortality for the shallow-water
species with which Big Skate most often co-occurs, such as Sand Sole and
Starry Flounder, than there is for Dover Sole, which has been the
subject of recurring stock assessments.

Both what discard rate information is available and anecdotal
information from those involved in the fishery for both skate species
indicate that discarding for Big Skate and Longnose Skate in the years
prior to 1995 were driven by the same market forced and the discard
rates were similar. Therefore, the discard rate for Longnose Skate was
used as a proxy for the discards of Big Skate in order to estimate Big
Skate discards.

The reconstructed landings of Big Skate for the period 1950--1995 had a
mean of 63.1 t with no significant trend (a linear model fit to the data
increased from 62.8 t in 1950 to 63.5 in 1995. The estimated tribal
catch prior to 1995 averaged less than 1 t and was not included in this
analysis of Big Skate discards for the years prior to 1995.

The mean discard rate for LN was 92.46\%, also with no significant
linear trend (the linear fit decreased from 92.8\% in 1950 to 92.1\% in
1995). An estimate of the mean annual discard amount can therefore be
calculated as from the mean discard rate and the mean landings as
\(\bar{L} / (1 - \bar{d})\) where \(\bar{L}\) is the mean landings
across that time period and \(\bar{d}\) is the mean discards (Figure
\ref{fig:Discard_calculations.png}).

Two alternative methods were used to estimate the mean annual discard
amount: applying the annual LN discard rates to the annual BS catch, and
applying 3-year moving averages of these two quantities. The use of the
annual values resulted in an implausibly high degree of annual
variability among the estimates, with the most extreme being a spike of
2146.4 in 1979 compared to 1032.7 t the year before and 654.0 the year
after. The use of the 3-year moving average dampened this variability
and these estimates were retained for a sensitivity analysis (Figure
\ref{fig:Discard_calculations}).

A discard mortality rate of 50 percent was assumed for all discards,
following the assumption used for the Longnose Skate assessment
conducted for the U.S. West Coast in 2007 ({\textbf{???}}) The same rate
has been used for skates in the trawl fishery in British Columbia, based
on an approximate average of these reported rates. In 2015, PFMC's
Groundfish Management Team (GMT) conducted a comprehensive literature
review of skate discard mortality, and concluded that the current
assumption regarding Big Skate discard mortality is consistent with
existing reported rates for other similar species.

Estimation of discard rates (discards amount relative to total catch)
during the period of the West Coast Groundfish Observer Program (WCGOP),
which began in 2002, was hindered by the landings of Big Skate primarily
occurring in the ``unspecified skate'' category prior to 2015.
Therefore, a discard rate was computed using the combination of Big
Skate and unspecified skate under the assumption that the vast majority
of the unspecified skates were Big Skate. A coefficient of variation was
calculated for the by bootstrapping vessels within ports because the
observer program randomly chooses vessels within ports to be observed.
For the years after the catch share program was implemented in 2011, the
trawl fishery was subject to 100\% observer coverage and discarding is
assumed to be known with minimal error (CV = 0.01).

The mean body weight of discarded Big Skates, calculated from the weight
and count of baskets of discarded Big Skate, was available for the years
2002--2017.

\hypertarget{fishery-independent-data-sources}{%
\section{Fishery-Independent Data
Sources}\label{fishery-independent-data-sources}}

\hypertarget{indices-of-abundance}{%
\subsection{Indices of abundance}\label{indices-of-abundance}}

\hypertarget{alaska-fisheries-science-center-afsc-triennial-shelf-survey}{%
\subsubsection{Alaska Fisheries Science Center (AFSC) Triennial Shelf
Survey}\label{alaska-fisheries-science-center-afsc-triennial-shelf-survey}}

Research surveys have been used since the 1970s to provide
fishery-independent information about the abundance, distribution, and
biological characteristics of Big Skate. A coast-wide survey was
conducted in 1977 (Gunderson, Donald Raymond and Sample, Terrance M.
\protect\hyperlink{ref-Gunderson1980}{1980}) by the Alaska Fisheries
Science Center, and repeated every three years through 2001. The final
year of this survey, 2004, was conducted by the NWFSC according to the
AFSC protocol. We refer to this as the \textbf{Triennial Survey}.

The survey design used equally-spaced transects from which searches for
tows in a specific depth range were initiated. The depth range and
latitudinal range was not consistent across years, but all years in the
period 1980-2004 included the area from 40\(^\circ\) 10'N north to the
Canadian border and a depth range that included 55-366 meters, which
spans the range where the vast majority of Big Skate encountered in all
trawl surveys. Therefore the index was based on this depth range. The
survey as conducted in 1977 had incomplete coverage and is not believe
to be comparable to the later years, and is not used in the index.

\hypertarget{northwest-fisheries-science-center-west-coast-groundfish-bottom-trawl-survey}{%
\subsubsection{Northwest Fisheries Science Center West Coast Groundfish
Bottom Trawl
Survey}\label{northwest-fisheries-science-center-west-coast-groundfish-bottom-trawl-survey}}

In 2003, the NWFSC took over an ongoing slope survey the AFSC had been
conducting, and expanded it spatially to include the continental shelf.
This survey, referred to in this document as the ``WCGBT Survey'' or
``WCGBTS'', is conducted annually. It uses a random-grid design covering
the coastal waters from a depth of 55 m to 1,280 m from late-May to
early-October (Bradburn, M.J. and Keller, A.A and Horness, B.H.
\protect\hyperlink{ref-Bradburn2011}{2011} , Keller, A.A. and Wallace,
J.R. and Methot, R.D. \protect\hyperlink{ref-Keller2017}{2017}). Four
chartered industry vessels are used each year (with the exception of
2013 when the U.S. federal-government shutdown curtailed the survey).

\hypertarget{index-standardization}{%
\subsubsection{Index Standardization}\label{index-standardization}}

The index standardization methods for the two bottom trawl surveys
matched that used for Longnose Skate and additional detail is provided
in (Gertseva, V.
\protect\hyperlink{ref-Gertseva2019}{2019}\protect\hyperlink{ref-Gertseva2019}{a}).
The data from both surveys was analyzed using a spatio-temporal
delta-model (Thorson, J. T. and Shelton, A. O. and Ward, E. J. and
Skaug, H. J. \protect\hyperlink{ref-Thorson2015}{2015}), implemented as
an R package VAST (Thorson, James T. and Barnett, Lewis A. K.
\protect\hyperlink{ref-Thorson2017a}{2017}) and publicly available
online (\url{https://github.com/James-Thorson/VAST}). Spatial and
spatio-temporal variation is specifically included in both encounter
probability and positive catch rates, a logit-link for encounter
probability, and a log-link for positive catch rates. Vessel-year
effects were included for each unique combination of vessel and year in
the database for the WCGBT Survey but not the Triennial survey. Further
details regarding model structure are available in the user manual
(\url{https://github.com/James-Thorson/VAST/blob/master/examples/VAST_user_manual.pdf}).

Spatial patterns in the survey estimates show Big Skate widely
distributed along the coast, with higher densities in the central and
more northern areas and closer to shore
\ref{fig:VAST_Yearly_Dens_Triennial}.

\hypertarget{internation-pacific-halibut-commission-longline-survey}{%
\subsubsection{Internation Pacific Halibut Commission Longline
Survey}\label{internation-pacific-halibut-commission-longline-survey}}

The IPHC has conducted an annual longline survey for Pacific Halibut off
the coast of Oregon and Washington since 1997 (no surveys were performed
in 1998 or 2000). Beginning in 1999, this has been a fixed station
design, with 84 locations in this area (station locations differed in
1997, and are therefore not comparable with subsequent surveys). 400 to
800 hooks have been deployed at each station in 100-hook groups
(typically called ``skates'' although that term will be avoided here to
avoid confusion). The gear used to conduct the survey was designed to
efficiently sample Pacific Halibut and used 16/0 (\#3) circle hooks
baited with Chum Salmon.

In some years from 2011 onward, additional stations were added to the
survey to sample Yelloweye Rockfish. These stations were excluded from
the analysis, as were additional stations added in 2013, 2014, and 2017,
off the coast of California (south of 42 degrees latitude). Some
variability in exact sampling location is practically unavoidable, and
leeway is given in the IPHC methods to center the set on the target
coordinates while allowing wind and currents to dictate the actual
direction in which the gear is deployed. This can result in different
habitats being accessed at each fixed deployment location across years.
One station that was very close to the U.S. Canada border had the
mid-point of the set in Canada in 2 out of the 19 years of the survey.
For consistency among years, all samples from this station were included
in the analysis, including those in Canada.

In most years, bycatch of non-halibut species has been recorded during
this survey on the first 20 hooks of each 100-hook group, although in
2003 only 10\% of the hooks were observed for bycatch, and starting in
2012, some stations had 100\% of the hooks observed for bycatch.
Combining these observation pattern with the number of hooks deployed
each year, resulted in most stations having 80, 100, 120, 140, or 160
hooks observed, with a mean of 144 hooks and a maximum of 800 hooks
observed. The depth range of the 84 stations considered was 42---530 m,
thus extending beyond the range of Big Skate, but 74\% of the stations
were shallower than 200 m. Big Skate have been observed at 51 of the 84
the standard stations that were retained for this analysis, but no
station had Big Skates observed in more than 12 out of the 19 years of
survey data, and only 10\% of the station/year combinations had at least
one observed Big Skate (Figure X). Of those station/year combinations
with at least one Big Skate observed, the Big Skates were observed on an
average of 1.3\% of the hooks observed. The highest proportion was 10
Big Skates out of 81 hooks observed at one station.

The IPHC longline survey catch data were standardized using a
Generalized Linear Model (GLM) with binomial error structure.
Catch-per-hook was modeled, rather than catch per station due to the
variability in the number of hooks deployed and observed each year. The
binomial error structure was considered logical, given the binary nature
of capturing (or not) a Longnose Skate on each longline hook. The
modeling approach is identical to that which has been applied in the
past for Yelloweye Rockfish ({\textbf{???}}), and Spiny Dogfish
(Gertseva, V.
\protect\hyperlink{ref-Gertseva2011}{2019}\protect\hyperlink{ref-Gertseva2011}{b}).
MCMC sampling of the GLM parameters was used to estimate the variability
around each index estimate. The median index estimates themselves were
approximately equal to the observed mean catch rate in each year (Figure
Y). In recent years, the IPHC standardization of the index of halibut
abundance has included an adjustment to account for missing baits on
hooks returned empty in an effort to account for reduced catchability of
the gear that may result from the lost bait. This adjustment was not
included in the analysis for Big Skate although it could be considered
in future years. \newpage

\hypertarget{biological-parameters-and-data}{%
\subsection{Biological Parameters and
Data}\label{biological-parameters-and-data}}

\hypertarget{measurement-details-and-conversion-factors}{%
\subsubsection{Measurement Details and Conversion
Factors}\label{measurement-details-and-conversion-factors}}

Some size measurements were taken as either disc width or inter-spiracle
width rather than total length. A conversion from disc width to total
length was estimated as \(L = 1.3399 * W\) based on from 95 samples from
WCGBT Survey where both measurements collected (R-squared = 0.9983).
Little sex difference observed, so using single relationship for both
sexes. This estimate is similar to the conversion estimated by Ebert
(\protect\hyperlink{ref-Ebert2008}{2008}) for Big Skate in Alaska. The
inter-spiracle width to total length was converted based on estimates
from Downs \& Cheng ({\textbf{???}}):

\begin{centering}

$L = 12.111 + 9.761*ISW$ (females),

$L = 3.824 + 10.927*ISW$ (males).

\end{centering}

\hypertarget{fishery-dependent-length-and-age-composition-data}{%
\subsubsection{Fishery dependent length and age composition
data}\label{fishery-dependent-length-and-age-composition-data}}

Fishery length composition data was available from PacFIN were available
for the years 1995--2018 (with the exception of 2000) as shown in Table
\ref{tab:PacFIN_Samples}. Ages were available from only 2004, 2008-2012,
and 2018. These were all represented as conditioned on length in order
to provide more detailed information about the relationship between age
and length, to reduce any influence of size-based selectivity on the age
composition, and to ensure independence from the length samples.
Furthermore, the samples from Washington in 2009 were sampled using a
length-stratified system, so should only be treated as conditioned on
length.

Length compositions of Big Skate discarded in commercial fisheries
measured by the West Coast Groundfish Observer program were available
for the years 2010--2017.

The input sample sizes for the length compositions were calculated via
the Stewart Method (Ian Stewart, personal communication, IPHC):

\begin{centering}

Input N = $N_{\text{hauls}} + 0.138 * N_{\text{fish}}$ if $N_{\text{fish}}/N_{\text{hauls}}$ is $<$ 44,

Input N = $7.06 * N_{\text{hauls}}$ if $N_{\text{fish}}/N_{\text{hauls}}$ is $\geq$ 44.

\end{centering}

However, no haul had greater than 44 Big Skate sampled, so only the
first formula was used.

\hypertarget{survey-length-and-age-composition-data}{%
\subsubsection{Survey length and age composition
data}\label{survey-length-and-age-composition-data}}

Lengths of Big Skate were only collected form the Triennial survey in
1998, 2001, and 2004, but 1998 had only 3 samples and were excluded from
this analysis. Length compositions were available for all years of the
WCGBT Survey. Sample sizes for both surveys are provided in Table
\ref{tab:Survey_Samples}. The WCGBT Survey used disc width for the years
2006 and 2007 and total length in all other years. Those samples where
only disc width was measured were converted to total length using the
formula above.

The length compositions from the fishery and each of the two surveys
aggregated across all years is shown in Figure
\ref{fig:comp_lendat_aggregated_across_time}.

Ages were available from the WCGBT Survey in the years 2009, 2010, 2016,
2017, and 2018. No ages were available from the Triennial Survey.

\vspace{.5cm}

\textbf{Ageing Precision and Bias}

Ages of Big Skate were all estimated based on growth band counts of
sectioned vertebrae. Ageing precision and bias were estimated using
double-reads of 518 Big Skate vertebrae using the approach of Punt et
al.~({\textbf{???}}). The results showed strong agreement among readers
(Figure \ref{fig:ageing_comparison}), with a standard deviation of the
ageing error increasing from about 0.4 at age 0 to 1.6 years at age 15
(Figure \ref{fig:ageing_imprecision}).

\vspace{.5cm}

\textbf{Weight-Length}

The mean weight as a function of length was estimated from 1159 samples
from the WCGBT Survey using a linear regression on a log-log scale. Sex
was not found to be a significant predictor, so a single relationship
was estimated: \(Weight = 0.0000074924 * Length ^ 2.9925\) (Figure
\ref{fig:weight-length}).

\vspace{.5cm}

\textbf{Sex Ratio, Maturity, and Fecundity}

The female maturity relationship was based on visual maturity estimates
from port samplers (n = 278, of which 241 were from Oregon and 37 from
Washington, with 24 mature specimens) as well as 55 samples from the
WCGBT Survey (of which 4 were mature). The resulting relationship was
\(L_{50\%} = 148.2453\) with a slope parameter of \(Beta = -0.13155\) in
the relationship \(M = (1 + Beta(L - L_{50\%}))^{-1}\) (Figure
\ref{fig:maturity}).

\vspace{.5cm}

\hypertarget{environmental-or-ecosystem-data-included-in-the-assessment}{%
\subsubsection{Environmental or Ecosystem Data Included in the
Assessment}\label{environmental-or-ecosystem-data-included-in-the-assessment}}

In this assessment, neither environmental nor ecosystem considerations
were explicitly included in the analysis. This is primarily due to a
lack of relevant data or results of analyses that could contribute
ecosystem-related quantitative information for the assessment.

\newpage

\hypertarget{assessment}{%
\section{Assessment}\label{assessment}}

\hypertarget{previous-assessments}{%
\subsection{Previous Assessments}\label{previous-assessments}}

\hypertarget{history-of-modeling-approaches-used-for-this-stock}{%
\subsubsection{History of Modeling Approaches Used for this
Stock}\label{history-of-modeling-approaches-used-for-this-stock}}

No previous stock assessment has been conducted for Big Skate. The
current management is based on an OFL estimate calculated from a proxy
for \(F_{MSY}\) and average survey biomass from the WCGBT Survey during
the years 2010--2012 ({\textbf{???}}). The \(F_{MSY}\) estimate was
based on the product of an assumed \(F_{MSY}/M\) ratio and an \(M\)
estimate of 0.162 based on the maximum age of 26 reported by McFarlane
and King (McFarlane GA and King JR
\protect\hyperlink{ref-McFandKing2006}{2006}). Values were sampled from
an assumed distribution around all these quantities to develop a measure
of uncertainty around the OFL estimate.

\hypertarget{model-description}{%
\subsection{Model Description}\label{model-description}}

\hypertarget{modeling-software}{%
\subsubsection{Modeling Software}\label{modeling-software}}

The STAT team used Stock Synthesis version 3.30.13 (Methot, Richard D.
and Wetzel, Chantell R. (\protect\hyperlink{ref-Methot2013}{2013}),
({\textbf{???}})). The r4ss package version 1.35.1 (Taylor et al.
\protect\hyperlink{ref-Taylor2019}{2019}) was used to post-process the
output data from Stock Synthesis.

\hypertarget{summary-of-data-for-fleets-and-areas}{%
\subsubsection{Summary of Data for Fleets and
Areas}\label{summary-of-data-for-fleets-and-areas}}

Catch is divided among 4 fleets in the base model: \emph{Fishery
(current)} combines all non-tribal sources of catch for the years 1995
onward, \emph{Discard (historical)} includes the estimated discard
amount calculated from the estimated Longnose Skate discard rate as
described above. The input catch for this fleet was 50\% of the total
estimate to account for the assumed 50\% discard mortality rate. This
data covers the period 1916--1994. \emph{Fishery (historical)} includes
the reconstructed landings estimates from each of the three states for
1916--1994. \emph{Tribal} includes the estimates of catch of Big Skate
by treaty tribes.

\hypertarget{other-specifications}{%
\subsubsection{Other Specifications}\label{other-specifications}}

This assessment covers the U.S. West Coast stock of Big Skate in off the
coasts of Washington, Oregon and California, the area bounded by the
U.S.-Canada border to the north, and the U.S.-Mexico border to the
south. The population is treated as a single coastwide stock with no net
movement in or out of the area. Females and males are modeled separately
as there is evidence for differences in growth based on both the age and
length data, as well as patterns in the sex ratios associated with the
length composition data. Natural Mortality is estimated within the model
using natural mortality prior developed by Hamel (2015). A Beverton-Holt
stock-recruit function is assumed with no deviations from the
spawner-recruit curve estimated.

The length composition data are stratified into 37 5-cm bins, ranging
between 20 and 200 cm and the age data are stratified into ages 0--15+,
conditioned on the same length bin structure. The population dynamics
are computed over a larger range of lengths age ages, with the 5-cm
length bins extending up to 250 cm and the numbers-at-age computed up to
age 20.

\hypertarget{data-weighting}{%
\subsubsection{Data Weighting}\label{data-weighting}}

The Francis (\protect\hyperlink{ref-Francis2011}{2011}) data weighting
method ``TA1.8'' as implemented in the r4ss package was used for all
length and age composition data.

\hypertarget{priors}{%
\subsubsection{Priors}\label{priors}}

\emph{Natural mortality}

A log-normal prior for natural mortality was based on a meta-analysis
completed by Hamel (\protect\hyperlink{ref-Hamel2015}{2015}). The Hamel
prior for M is lognormal(ln(5.4/max age), .438), which based on the
single 15-year-old fish observed out of 1034 ages from the WCGBT Survey.
This results in lognormal(log(0.36)=-1.021651, 0.438) prior.

\vspace{.5cm}

\emph{Survey catchability}

The lack of contrast in the data resulted in unstable model results
under a variety of configurations. To keep biomass estimates within a
plausibel range, a prior was applied to the catchability parameter
(\(q\)) for the WCGBT Survey. This same prior was developed for the 2007
Longnose Skate assessment ({\textbf{???}}) and is being used for the
current Longnose Skate assessment (Gertseva, V.
\protect\hyperlink{ref-Gertseva2019}{2019}\protect\hyperlink{ref-Gertseva2019}{a}).
The prior for the WCGBT Survey was derived as follows.

The WCGBT Survey covers the full latitudinal range of longnose skate
modeled in the assessment, and thus, the latitudinal availability factor
was assumed to be one (complete latitudinal coverage). The survey
coverage exceeds the maximum depth distribution of longnose skate but
may not fully cover the shallow end of the skate distribution. A range
of 95 to 100 percent was assumed for the depth availability. A range of
75 to 95 percent was assumed for vertical availability on the basis that
longnose skate are known to bury in the mud and therefore some may be
unavailable to the bottom trawl gear. The largest bounds were placed on
the probability of capture, given a fish is in the net path. It is known
that flatfish can be herded by trawl gear, and it is possible that this
could also occur for skates. However, it is also possible that skate
could avoid the trawl nets. For capture probability, a range of 75 to
150 percent was assumed. Best estimates for each factors were set at the
midpoint of the range for individual factors, except for the probability
of capture, which was given a value of one. The overall estimate for the
survey catchability was thus estimated to be 0.83 and the consequent
bounds on catch, and the best assumption are: (0.53, 1.43) and 0.83
respectively. The best estimate was equated to the median of a lognormal
distribution and the bounds to 99 percent of that distribution. This
resulted in a normal prior on \(log(q)\), with a mean of -0.188, and
standard deviation of 0.187.

\hypertarget{estimated-and-fixed-parameters}{%
\subsubsection{Estimated and Fixed
Parameters}\label{estimated-and-fixed-parameters}}

A full list of all estimated and fixed parameters is provided in Tables
\ref{tab:model_params}.

The base model has a total of 44 estimated parameters in the following
categories:

\begin{itemize}
  \item 1 natural mortality parameter,
  \item 6 parameters related to female growth and the variability in length age age,
  \item 2 parameters related to male growth relative to female growth,
  \item 1 stock-recruit parameter ($log(R_0)$ controlling equilibrium recruitment)
  \item 3 catchability parameters (1 for the WCGBT Survey and 1 each for the early and late periods of the Triennial Survey)
  \item 2 extra standard deviation parameters (1 for each survey), and
  \item 29 selectivity parameters, including 16 related to time-varying retention rate
\end{itemize}

The estimated parameters are described in greater detail below and a
full list of all estimated and parameters is provided in Table
\ref{tab:model_params}.

\emph{Growth.}

Examination of patterns of age-at-length and length-at-age indicated
unusual patterns of growth for Big Skate, including almost linear growth
for the early years during which both sexes appeared to have similar
average size, followed by strong differences in size at older ages. This
led to the choice to model growth using the ``growth cessation model''
recently developed by Maunder et
al.~(\protect\hyperlink{ref-maunder2018growth}{2018}). This model
provided two key advantages over the more common von Bertalanffy growth
model in the case of Big Skate: it allowed essentially linear growth for
the early years and it allowed growth for the earlier ages to be similar
between females and males while diverging at older ages. The growth
cessation model also improve the negative log likelihood by 45 units
relative to the von Bertalanffy growth model.

\emph{Natural Mortality.}

Male natural mortality was assumed equal to the value estimated for
females. Sensitivity analyses were used to test the impact of both the
the prior on natural mortality and the assumption of equal natural
mortality for both sexes.

\emph{Selectivity.}

A double-normal selectivity function was used for all fleets to allow
consideration of both asymptotic and dome-shaped patterns. For the
fishery and the Triennial survey, the difference in likelihood between
dome-shaped and asymptotic patterns was very small and in the case of
the Triennial survey, the dome-shape occurred at a length beyond almost
all observations, indicating that this shape was likely driven by fit to
other data sources, such as the index, rather than the length
composition data. The WCGBT Survey was allowed to remain dome-shaped as
this survey had the selectivity peak at a smaller length than the other
fleets and the likelihood was improved by the dome-shape. The WCGBT
Survey also has the shortest hauls, with 15 minutes or less of bottom
contact, so larger skates may be better able to escape the net.

In order to fit a strong skew in the sex ratios toward males for the
length bins in which the majority of the samples were found, it was
necessary to estimate a sex-specific offset of selectivity. Two offset
parameters were estimated for all fleets, one for the difference in
length at peak selectivity and another for the maximum selectivity. The
ascending slope was assumed equal in all cases, as was the descending
slope for the WCGBT Survey.

\emph{Other Estimated Parameters.}

\emph{Other Fixed Parameters.}

Steepness was fixed at 0.4.

\hypertarget{model-selection-and-evaluation}{%
\subsection{Model Selection and
Evaluation}\label{model-selection-and-evaluation}}

\hypertarget{key-assumptions-and-structural-choices}{%
\subsubsection{Key Assumptions and Structural
Choices}\label{key-assumptions-and-structural-choices}}

\textbf{\(\color{red}{\text{To be added}}\)}

\hypertarget{alternate-models-considered}{%
\subsubsection{Alternate Models
Considered}\label{alternate-models-considered}}

\textbf{\(\color{red}{\text{To be added}}\)}

\hypertarget{convergence}{%
\subsubsection{Convergence}\label{convergence}}

\textbf{\(\color{red}{\text{To be added}}\)}

\hypertarget{response-to-the-current-star-panel-requests}{%
\subsection{Response to the Current STAR Panel
Requests}\label{response-to-the-current-star-panel-requests}}

\begin{description}[style=sameline]

\item[Request No. 1: ] \hfill \\
  
\textbf{Rationale:} xxx   
    
\textbf{STAT Response:} xxx


\item[Request No. 2: ] \hfill \\


\textbf{Rationale:} xxx 


\textbf{STAT Response:} xxx
    

\item[Request No. 3: ] \hfill \\

\textbf{Rationale:} x.  
    
  
\textbf{STAT Response:} xxx

\item[Request No. 4: ] \hfill \\

\textbf{Rationale:} xxx 
    
    
\textbf{STAT Response:} xxx


\item[Request No. 5: ] \hfill \\

\textbf{Rationale:} xxx
  
\textbf{STAT Response:} xxx  
    


\end{description}

\hypertarget{base-case-model-results}{%
\subsection{Base Case Model Results}\label{base-case-model-results}}

The following description of the model results reflects a base model
that incorporates all of the changes made during the STAR panel (see
previous section). The base model parameter estimates and their
approximate asymptotic standard errors are shown in Table
\ref{tab:model_params} and the likelihood components are in Table
\ref{tab:like_components}. Estimates of derived reference points and
approximate 95\% asymptotic confidence intervals are shown in Table
\ref{tab:Ref_pts_mod1}. Time-series of estimated stock size over time
are shown in Table \ref{tab:Timeseries_mod1}.

\hypertarget{parameter-estimates}{%
\subsubsection{Parameter Estimates}\label{parameter-estimates}}

The additional survey variability (process error added directly to each
year's input variability) for all surveys was estimated within the
model.

(Figure
\ref{fig:ts11_Age-0_recruits_(1000s)_with_95_asymptotic_intervals} ).

The stock-recruit curve \ldots{} Figure \ref{fig:SR_curve2} with
estimated recruitments also shown.

\hypertarget{fits-to-the-data}{%
\subsubsection{Fits to the Data}\label{fits-to-the-data}}

Model fits to the indices of abundance, fishery length composition,
survey length composition, discard rates, mean body weight, and
conditional age-at-length observations are all discussed below.

The observed indices show much more variability than the model
expectation, with the fit to the WCGBT Survey essentially a flat line
(Figure \ref{fig:index2_cpuefit_WCGBTS}) and the fit to the Triennial
Survey only showing a noticeable change over time due to the separate
catchability parameter estimated for the early and late periods (Figure
\ref{fig:index2_cpuefit_Triennial}).

The fits to the length data are much better thanks to the combination of
the growth cessation model and the sex-specific offsets to selectivity
(Figures
\ref{fig:comp_lenfit_aggregated_across_time}--\ref{fig:comp_lenfit__multi-fleet_comparison}).

The conditional age-at-length data is likewise fit reasonably well, with
some patterns in residuals showing variability among years, but no clear
pattern that is consistent across years (Figures
\ref{fig:age_fit_fishery} and \ref{fig:age_fit_WCGBTS}).

Sex ratio data is not included in the likelihood as such, but a part of
the length composition likelihood in which the proportions of females
and males are included in a single vector comapred to the model
expectations in the multinomial likelihood. The patterns in sex ratio by
length bin show fewer females than males for the middle range of sizes
(70--120 cm), with a shift to almost 100\% females for the largest size
bins (over 130 cm). The use of sex-specific growth curves was adequate
to fit the ratios for the largest bins, but ratio skews toward males at
lengths where the mean ages are similar for females and males. The fit
to this part of the sex ratio pattern required an offset is selectivity.

\hypertarget{uncertainty-and-sensitivity-analyses}{%
\subsubsection{Uncertainty and Sensitivity
Analyses}\label{uncertainty-and-sensitivity-analyses}}

A number of sensitivity analyses were conducted, including:

\begin{enumerate}

  \item Setting all selectivity curves to be asymptotic
  
  \item Setting all selectivity curves to be dome-shaped

  \item Removing the sex-specific offset on the selectivity curves
  
  \item Removing the prior on catchability for the WCGBT Survey
  
  \item Estimating a single catchability for all years in the Triennial Survey
  
  \item Estimating separate natural mortality parameters for males and females
  
  \item Removing the prior on natural mortality
  
  \item Using the von Bertalanffy growth model

  \item Using the Richards growth model

  \item Tuning the sample sizes using the McAllister-Ianelli method
  
  \item Estimating historic discards based on 3yr average of discard rates and landings
  
  \item Changeing discard mortality from 0.5 to 0.4
  
  \item Changeing discard mortality from 0.5 to 0.6
  
\end{enumerate}

Results of these sensitivities are shown in Figures
\ref{fig:Sensitivity_sel_and_Q} to \ref{fig:Sensitivity_bio_and_misc},
and Tables \ref{tab:Sensitivity_sel_and_Q} to
\ref{tab:Sensitivity_bio_and_misc}.

\textbf{\(\color{red}{\text{Additional text to be added}}\)}

\hypertarget{retrospective-analysis}{%
\subsubsection{Retrospective Analysis}\label{retrospective-analysis}}

\textbf{\(\color{red}{\text{To be added}}\)}

\hypertarget{likelihood-profiles}{%
\subsubsection{Likelihood Profiles}\label{likelihood-profiles}}

Likelihood profiles were conducted over \(log(R_0)\), stock-recruit
steepness (\(h\)) and natural mortality (\(M\)).

Results of these profiles are shown in Figures \ref{fig:profile_logR0}
to \ref{fig:profile_M_compare1_spawnbio}.

\textbf{\(\color{red}{\text{Additional text to be added}}\)}

\hypertarget{reference-points-1}\) reference harvest rate and with a 95\%
confidence interval of 303 mt based on estimates of uncertainty. The
spawning biomass equivalent to 40\% of the unfished level
(\(SB_{40\%}\)) was 610 mt.

(Figure
\ref{fig:ts7_Spawning_biomass_(mt)_with_95_asymptotic_intervals_intervals}

The 2018 spawning biomass relative to unfished equilibrium spawning
biomass is above/below the target of 40\% of unfished levels (Figure
\ref{fig:ts9_Spawning_depletion_with_95_asymptotic_intervals_intervals}).
The relative fishing intensity, \((1-SPR)/(1-SPR_{50\%})\), has been xxx
the management target for the entire time series of the model.

Table \ref{tab:Ref_pts_mod1} shows the full suite of estimated reference
points for the base model and Figure \ref{fig:yield1_yield_curve} shows
the equilibrium curve based on a steepness value xxx.

\newpage

\hypertarget{harvest-projections-and-decision-tables}{%
\section{Harvest Projections and Decision
Tables}\label{harvest-projections-and-decision-tables}}

The forecasts of stock abundance and yield were developed using the
final base model, with the forecasted projections of the OFL presented
in Table \ref{tab:OFL_projection}.

The forecasted projections of the OFL for each model are presented in
Table \ref{tab:Decision_table_mod1}.

\newpage

\hypertarget{regional-management-considerations}{%
\section{Regional Management
Considerations}\label{regional-management-considerations}}

\newpage

\hypertarget{research-needs}{%
\section{Research Needs}\label{research-needs}}

There are a number of areas of research that could improve the stock
assessment for Big Skate. Below are issues identified by the STAT team
and the STAR panel:

\begin{enumerate}

\item \textbf{Data!}: 

\item \textbf{xxxx}:

\item \textbf{xxxx}:

\item \textbf{xxxx}:

\item \textbf{xxxx}:

\end{enumerate}

\hypertarget{acknowledgments}{%
\section{Acknowledgments}\label{acknowledgments}}

The authors gratefully acknowledge the time and effort reviewers Stacey
Miller, Jim Hastie and Owen Hamel put into making this a polished
document.

\newpage
\FloatBarrier
\newpage

\hypertarget{tables}{%
\section{Tables}\label{tables}}

\hypertarget{data-tables}{%
\subsection{Data Tables}\label{data-tables}}

\begin{longtable}{rrrrrr}
\caption{Landings by source.  Landings are reconstructed histories 1916-1995.} \\ 
  \hline
Year & CA (mt) & OR (mt) & WA (mt) & Tribal (mt) & Total (mt) \\ 
  \hline 
\endhead 
\hline 
\multicolumn{3}{l}{\footnotesize Continued on next page} 
\endfoot 
\endlastfoot 
 \hline
1916 & 78.30 & 0.00 & 0.00 & 0.00 & 78.30 \\ 
  1917 & 80.10 & 0.00 & 0.00 & 0.00 & 80.10 \\ 
  1918 & 101.20 & 0.00 & 0.00 & 0.00 & 101.20 \\ 
  1919 & 75.20 & 0.00 & 0.00 & 0.00 & 75.20 \\ 
  1920 & 122.00 & 0.00 & 0.00 & 0.00 & 122.00 \\ 
  1921 & 17.80 & 0.00 & 0.00 & 0.00 & 17.80 \\ 
  1922 & 30.80 & 0.00 & 0.00 & 0.00 & 30.80 \\ 
  1923 & 34.20 & 0.00 & 0.00 & 0.00 & 34.20 \\ 
  1924 & 33.40 & 0.00 & 0.00 & 0.00 & 33.40 \\ 
  1925 & 46.70 & 0.00 & 0.00 & 0.00 & 46.70 \\ 
  1926 & 59.30 & 0.00 & 0.00 & 0.00 & 59.30 \\ 
  1927 & 67.10 & 0.00 & 0.00 & 0.00 & 67.10 \\ 
  1928 & 116.70 & 0.00 & 0.00 & 0.00 & 116.70 \\ 
  1929 & 107.50 & 0.00 & 0.00 & 0.00 & 107.50 \\ 
  1930 & 70.80 & 0.00 & 0.00 & 0.00 & 70.80 \\ 
  1931 & 43.60 & 0.00 & 0.00 & 0.00 & 43.60 \\ 
  1932 & 73.30 & 0.00 & 0.00 & 0.00 & 73.30 \\ 
  1933 & 46.50 & 0.00 & 0.00 & 0.00 & 46.50 \\ 
  1934 & 57.40 & 0.00 & 0.00 & 0.00 & 57.40 \\ 
  1935 & 70.60 & 0.00 & 0.00 & 0.00 & 70.60 \\ 
  1936 & 87.70 & 0.00 & 0.00 & 0.00 & 87.70 \\ 
  1937 & 115.40 & 0.00 & 0.00 & 0.00 & 115.40 \\ 
  1938 & 99.40 & 0.00 & 0.00 & 0.00 & 99.40 \\ 
  1939 & 90.90 & 0.00 & 0.00 & 0.00 & 90.90 \\ 
  1940 & 60.30 & 5.30 & 0.00 & 0.00 & 65.70 \\ 
  1941 & 53.10 & 56.40 & 0.00 & 0.00 & 109.40 \\ 
  1942 & 27.00 & 34.40 & 0.00 & 0.00 & 61.40 \\ 
  1943 & 20.40 & 0.90 & 0.00 & 0.00 & 21.30 \\ 
  1944 & 7.80 & 1.60 & 0.00 & 0.00 & 9.50 \\ 
  1945 & 13.30 & 0.30 & 0.00 & 0.00 & 13.50 \\ 
  1946 & 17.10 & 1.80 & 0.00 & 0.00 & 18.90 \\ 
  1947 & 24.10 & 0.00 & 0.00 & 0.00 & 24.10 \\ 
  1948 & 30.70 & 5.70 & 0.00 & 0.00 & 36.30 \\ 
  1949 & 31.90 & 0.00 & 7.20 & 0.00 & 39.10 \\ 
  1950 & 32.20 & 2.10 & 2.10 & 0.00 & 36.40 \\ 
  1951 & 21.70 & 4.70 & 3.90 & 0.00 & 30.30 \\ 
  1952 & 39.10 & 0.10 & 7.80 & 0.00 & 46.90 \\ 
  1953 & 124.90 & 1.20 & 1.60 & 0.00 & 127.60 \\ 
  1954 & 38.80 & 2.30 & 1.20 & 0.00 & 42.40 \\ 
  1955 & 45.70 & 35.60 & 1.60 & 0.00 & 82.90 \\ 
  1956 & 40.40 & 2.60 & 3.10 & 0.00 & 46.10 \\ 
  1957 & 49.50 & 0.00 & 2.50 & 0.00 & 52.00 \\ 
  1958 & 38.80 & 0.00 & 0.20 & 0.00 & 38.90 \\ 
  1959 & 46.50 & 0.00 & 0.80 & 0.00 & 47.30 \\ 
  1960 & 39.20 & 0.00 & 0.70 & 0.00 & 39.80 \\ 
  1961 & 54.40 & 40.90 & 4.60 & 0.00 & 99.80 \\ 
  1962 & 44.40 & 27.90 & 5.20 & 0.00 & 77.60 \\ 
  1963 & 53.20 & 30.40 & 2.10 & 0.00 & 85.70 \\ 
  1964 & 49.90 & 28.30 & 2.70 & 0.00 & 80.90 \\ 
  1965 & 34.30 & 12.80 & 3.50 & 0.00 & 50.60 \\ 
  1966 & 36.40 & 20.10 & 0.60 & 0.00 & 57.00 \\ 
  1967 & 53.30 & 15.60 & 6.60 & 0.00 & 75.50 \\ 
  1968 & 55.30 & 45.40 & 8.80 & 0.00 & 109.50 \\ 
  1969 & 32.50 & 33.80 & 6.60 & 0.00 & 72.90 \\ 
  1970 & 16.30 & 11.90 & 0.10 & 0.00 & 28.20 \\ 
  1971 & 18.50 & 3.10 & 0.00 & 0.00 & 21.60 \\ 
  1972 & 33.50 & 2.00 & 0.10 & 0.00 & 35.60 \\ 
  1973 & 40.70 & 0.90 & 0.00 & 0.00 & 41.70 \\ 
  1974 & 21.90 & 5.90 & 0.10 & 0.00 & 27.80 \\ 
  1975 & 39.80 & 2.00 & 0.00 & 0.00 & 41.80 \\ 
  1976 & 20.70 & 31.30 & 0.20 & 0.00 & 52.20 \\ 
  1977 & 32.80 & 31.50 & 0.60 & 0.00 & 64.90 \\ 
  1978 & 67.70 & 77.30 & 4.00 & 0.00 & 149.10 \\ 
  1979 & 90.50 & 75.50 & 30.40 & 0.00 & 196.40 \\ 
  1980 & 17.60 & 34.10 & 5.20 & 0.00 & 56.90 \\ 
  1981 & 138.00 & 14.80 & 6.50 & 0.00 & 159.30 \\ 
  1982 & 78.30 & 5.20 & 14.60 & 0.00 & 98.10 \\ 
  1983 & 55.30 & 14.20 & 8.90 & 0.00 & 78.40 \\ 
  1984 & 26.20 & 4.90 & 1.60 & 0.00 & 32.70 \\ 
  1985 & 60.30 & 0.40 & 4.90 & 0.00 & 65.60 \\ 
  1986 & 27.20 & 1.60 & 8.90 & 0.00 & 37.80 \\ 
  1987 & 22.60 & 1.90 & 18.40 & 1.00 & 43.90 \\ 
  1988 & 15.30 & 0.30 & 10.90 & 1.20 & 27.60 \\ 
  1989 & 18.90 & 0.20 & 6.20 & 0.00 & 25.30 \\ 
  1990 & 25.10 & 0.00 & 9.60 & 0.10 & 34.90 \\ 
  1991 & 22.80 & 0.20 & 21.50 & 0.10 & 44.60 \\ 
  1992 & 24.60 & 0.30 & 11.20 & 0.00 & 36.10 \\ 
  1993 & 29.00 & 0.20 & 21.00 & 0.60 & 50.70 \\ 
  1994 & 27.70 & 2.50 & 20.50 & 0.10 & 50.70 \\ 
  1995 & 43.00 & 41.20 & 21.80 & 0.10 & 106.00 \\ 
  1996 & 146.70 & 138.50 & 22.80 & 0.10 & 308.10 \\ 
  1997 & 228.40 & 215.40 & 84.00 & 0.20 & 528.00 \\ 
  1998 & 120.50 & 51.40 & 22.70 & 0.20 & 194.90 \\ 
  1999 & 109.50 & 131.30 & 41.40 & 0.40 & 282.60 \\ 
  2000 & 69.40 & 193.60 & 97.70 & 0.30 & 361.00 \\ 
  2001 & 75.30 & 115.10 & 26.70 & 0.40 & 217.50 \\ 
  2002 & 34.70 & 102.80 & 70.80 & 4.80 & 213.10 \\ 
  2003 & 48.80 & 223.00 & 65.70 & 5.40 & 342.80 \\ 
  2004 & 45.20 & 105.90 & 98.00 & 4.60 & 253.80 \\ 
  2005 & 33.40 & 151.30 & 113.10 & 15.70 & 313.40 \\ 
  2006 & 102.40 & 206.60 & 66.20 & 24.90 & 400.00 \\ 
  2007 & 35.50 & 190.40 & 29.10 & 19.90 & 274.90 \\ 
  2008 & 46.00 & 280.10 & 36.80 & 3.20 & 366.00 \\ 
  2009 & 9.60 & 162.00 & 16.50 & 17.50 & 205.70 \\ 
  2010 & 1.20 & 157.50 & 25.00 & 12.50 & 196.20 \\ 
  2011 & 0.50 & 231.50 & 10.00 & 26.40 & 268.40 \\ 
  2012 & 6.80 & 216.30 & 5.00 & 41.60 & 269.60 \\ 
  2013 & 20.90 & 92.30 & 13.00 & 8.80 & 135.00 \\ 
  2014 & 41.00 & 286.00 & 16.80 & 28.60 & 372.40 \\ 
  2015 & 35.20 & 218.80 & 1.00 & 76.60 & 331.50 \\ 
  2016 & 15.00 & 317.50 & 1.20 & 77.80 & 411.50 \\ 
  2017 & 28.00 & 188.00 & 1.40 & 60.20 & 277.60 \\ 
  2018 & 23.80 & 115.80 & 2.40 & 30.60 & 172.60 \\ 
   \hline
\hline
\label{tab:Reconstructed_Landings_byState}
\end{longtable}

\begin{center}\rule{0.5\linewidth}{\linethickness}\end{center}

\FloatBarrier
\newpage

\begin{table}[ht]
\centering
\caption{Index inputs.} 
\label{tab:index_inputs}
\begin{tabular}{l>{\centering}p{0.6in}>{\centering}p{0.6in}>{\centering}p{0.6in}>{\centering}p{0.6in}>{\centering}p{0.6in}>{\centering}p{0.6in}}
  \hline
   \multicolumn{1}{c}{} & \multicolumn{2}{c}{WCGBTS} & \multicolumn{2}{c}{Triennial} & \multicolumn{2}{c}{IPHC} \\  \cmidrule(lr){2-3} \cmidrule(lr){4-5} \cmidrule(lr){6-7}
  Year & Obs & se\_log & Obs & se\_log & Obs & se\_log \\ 
  \hline
1980 &  &  & 467.83 & 0.53 &  &  \\ 
  1983 &  &  & 911.85 & 0.30 &  &  \\ 
  1986 &  &  & 996.75 & 0.29 &  &  \\ 
  1989 &  &  & 1431.65 & 0.22 &  &  \\ 
  1992 &  &  & 2426.18 & 0.20 &  &  \\ 
  1995 &  &  & 497.24 & 0.26 &  &  \\ 
  1998 &  &  & 2437.75 & 0.20 &  &  \\ 
  1999 &  &  &  &  & 0.00 & 0.17 \\ 
  2001 &  &  & 1669.73 & 0.23 & 0.00 & 0.29 \\ 
  2002 &  &  &  &  & 0.00 & 0.53 \\ 
  2003 & 8170.51 & 0.20 &  &  & 0.00 & 0.43 \\ 
  2004 & 14349.00 & 0.18 & 3674.14 & 0.19 & 0.00 & 0.20 \\ 
  2005 & 12122.52 & 0.16 &  &  & 0.00 & 0.18 \\ 
  2006 & 9273.79 & 0.18 &  &  & 0.00 & 0.64 \\ 
  2007 & 8137.47 & 0.18 &  &  & 0.00 & 0.34 \\ 
  2008 & 5494.76 & 0.21 &  &  & 0.00 & 0.81 \\ 
  2009 & 10721.30 & 0.17 &  &  & 0.00 & 0.48 \\ 
  2010 & 11475.29 & 0.14 &  &  & 0.00 & 0.24 \\ 
  2011 & 8029.69 & 0.16 &  &  & 0.00 & 0.20 \\ 
  2012 & 11593.79 & 0.16 &  &  & 0.00 & 0.61 \\ 
  2013 & 11521.85 & 0.17 &  &  & 0.00 & 0.20 \\ 
  2014 & 19855.79 & 0.13 &  &  & 0.00 & 0.19 \\ 
  2015 & 19251.41 & 0.13 &  &  & 0.00 & 0.16 \\ 
  2016 & 17141.95 & 0.15 &  &  & 0.00 & 0.17 \\ 
  2017 & 13237.37 & 0.14 &  &  & 0.00 & 0.18 \\ 
  2018 & 14568.79 & 0.14 &  &  & 0.00 & 0.26 \\ 
   \hline
  \end{tabular}
\end{table}

\FloatBarrier
\newpage

\begin{table}[ht]
\centering
\caption{PacFIN Samples.} 
\label{tab:PacFIN_Samples}
\begin{tabular}{lrrrrrrrrrr}
  \hline
   \multicolumn{1}{c}{} & \multicolumn{2}{c}{CA} & \multicolumn{2}{c}{OR} & \multicolumn{2}{c}{WA} & \multicolumn{2}{c}{All Landings} & \multicolumn{2}{c}{Discards} \\  \cmidrule(lr){2-3} \cmidrule(lr){4-5} \cmidrule(lr){6-7} \cmidrule(lr){8-9} \cmidrule(lr){10-11}
  Year & Ntows & Nfish & Ntows & Nfish & Ntows & Nfish & Ntows & Nfish & Ntows & Nfish \\ 
  \hline
Lengths &  &  &  &  &  &  &  &  &  &  \\ 
  1995 &  &  &   6 &  55 &  &  &   6 &  55 &  &  \\ 
  1996 &  &  &   3 &   8 &  &  &   3 &   8 &  &  \\ 
  1997 &  &  &   1 &  14 &  &  &   1 &  14 &  &  \\ 
  1998 &  &  &   1 &   2 &  &  &   1 &   2 &  &  \\ 
  1999 &  &  &   1 &   8 &  &  &   1 &   8 &  &  \\ 
  2000 &  &  &  &  &  &  &  &  &  &  \\ 
  2001 &  &  &   3 &  43 &  &  &   3 &  43 &  &  \\ 
  2002 &  &  &   6 & 199 &  &  &   6 & 199 &  &  \\ 
  2003 &  &  &   9 & 202 &  &  &   9 & 202 &  &  \\ 
  2004 &  &  &   2 &  27 &   2 &  12 &   4 &  39 &  &  \\ 
  2005 &  &  &   7 & 123 &   6 &  87 &  13 & 210 &  &  \\ 
  2006 &  &  &  13 & 310 &  15 & 191 &  28 & 501 &  &  \\ 
  2007 &   1 &   1 &  10 & 128 &   9 & 172 &  20 & 301 &  &  \\ 
  2008 &  &  &  10 &  94 &   8 &  94 &  18 & 188 &  &  \\ 
  2009 &   8 &  32 &  17 & 234 &   1 &  18 &  26 & 284 &  &  \\ 
  2010 &   2 &   8 &  15 & 186 &  &  &  17 & 194 & 149 & 349 \\ 
  2011 &   2 &   2 &  29 & 418 &   4 &   9 &  35 & 429 & 554 & 1518 \\ 
  2012 &   3 &  43 &  24 & 477 &   3 &  38 &  30 & 558 & 544 & 1405 \\ 
  2013 &  11 & 201 &  11 & 252 &   8 & 168 &  30 & 621 & 443 & 987 \\ 
  2014 &  15 & 217 &  11 & 237 &   5 & 249 &  31 & 703 & 676 & 1625 \\ 
  2015 &  25 & 237 &  21 & 411 &   2 &   5 &  48 & 653 & 688 & 1557 \\ 
  2016 &  14 & 181 &  34 & 444 &   7 &  98 &  55 & 723 & 652 & 1456 \\ 
  2017 &  14 & 239 &  50 & 668 &  12 &  47 &  76 & 954 & 508 & 1248 \\ 
  2018 &  15 & 133 &  46 & 552 &  14 &  98 &  75 & 783 &  &  \\ 
  Ages &  &  &  &  &  &  &  &  &  &  \\ 
  2004 &  &  &  &  &   2 &  11 &   2 &  11 &  &  \\ 
  2008 &  &  &   8 &  80 &  &  &   8 &  80 &  &  \\ 
  2009 &  &  &  10 &  87 &   8 &  65 &  18 & 152 &  &  \\ 
  2010 &  &  &  10 & 102 &  &  &  10 & 102 &  &  \\ 
  2011 &  &  &  21 & 202 &  &  &  21 & 202 &  &  \\ 
  2012 &  &  &  12 & 120 &  &  &  12 & 120 &  &  \\ 
  2018 &  &  &   6 &  39 &  13 &  93 &  19 & 132 &  &  \\ 
   \hline
  \end{tabular}
\end{table}

\FloatBarrier
\newpage

\begin{table}[ht]
\centering
\caption{Samples from the surveys.} 
\label{tab:Survey_Samples}
\begin{tabular}{lllllll}
  \hline
   \multicolumn{1}{c}{} & \multicolumn{2}{c}{Triennial} & \multicolumn{2}{c}{WCGBTS} & \multicolumn{2}{c}{IPHC} \\  \cmidrule(lr){2-3} \cmidrule(lr){4-5} \cmidrule(lr){6-7}
  NA. & Triennial & NA..1 & WCGBTS & NA..2 & IPHC & NA..3 \\ 
  \hline
Year & Ntows & Nfish & Ntows & Nfish & Nsets & Nfish \\ 
  Lengths &  &  &  &  &  &  \\ 
  2001 & 41 & 81 &  &  &  &  \\ 
  2003 &  &  & 60 & 197 &  &  \\ 
  2004 & 39 & 100 & 81 & 262 &  &  \\ 
  2005 &  &  & 99 & 328 &  &  \\ 
  2006 &  &  & 67 & 154 &  &  \\ 
  2007 &  &  & 76 & 192 &  &  \\ 
  2008 &  &  & 53 & 159 &  &  \\ 
  2009 &  &  & 82 & 305 &  &  \\ 
  2010 &  &  & 130 & 466 &  &  \\ 
  2011 &  &  & 99 & 360 &  &  \\ 
  2012 &  &  & 104 & 395 &  &  \\ 
  2013 &  &  & 84 & 316 &  &  \\ 
  2014 &  &  & 149 & 552 & 14 & 54 \\ 
  2015 &  &  & 134 & 546 &  &  \\ 
  2016 &  &  & 105 & 422 &  &  \\ 
  2017 &  &  & 125 & 496 &  &  \\ 
  2018 &  &  & 123 & 331 &  &  \\ 
   &  &  &  &  &  &  \\ 
  Ages &  &  &  &  &  &  \\ 
  2009 &  &  & 77 & 230 &  &  \\ 
  2010 &  &  & 124 & 333 &  &  \\ 
  2016 &  &  & 100 & 138 &  &  \\ 
  2017 &  &  & 110 & 164 &  &  \\ 
  2018 &  &  & 118 & 169 &  &  \\ 
   \hline
  \end{tabular}
\end{table}

\FloatBarrier
\newpage

\newpage
\FloatBarrier

\FloatBarrier

\FloatBarrier

\FloatBarrier

\newpage

\hypertarget{model-results-tables}{%
\subsection{Model Results Tables}\label{model-results-tables}}

\FloatBarrier

\begin{table}[ht]
\centering
\caption{Results from 100 jitters from the base 
                                      case model.} 
\label{tab:jitter}
\begin{tabular}{llll}
  \hline
Description & Value & NA & NA \\ 
  \hline
Returned to base case & - & - & - \\ 
  Found local minimum & - & - & - \\ 
  Found better solution & - & - & - \\ 
  Error in likelihood & - & - & - \\ 
  Total & 100 & 100 & 100 \\ 
   \hline
\end{tabular}
\end{table}

\begin{landscape}
\begin{longtable}{lp{2.5in}lrcccl}
\caption{List of parameters used in
                                              the base model, including estimated 
                                              values and standard deviations (SD), 
                                              bounds (minimum and maximum), 
                                              estimation phase (negative values indicate
                                              not estimated), status (indicates if 
                                              parameters are near bounds, and prior type
                                              information (mean, SD).} \\ 
  \hline
No. & Parameter & Value & Phase & Bounds & Status & SD & Prior (Exp.Val, SD)  \\ 
  \hline 
\endhead 
\hline 
\multicolumn{3}{l}{\footnotesize Continued on next page} 
\endfoot 
\endlastfoot 
 \hline
1 & NatM\_p\_1\_Fem\_GP\_1 & 0.384 & 1 & (0.01, 0.8) & OK & 0.014 & Log\_Norm (-1.02165, 0.0438) \\ 
  2 & L\_at\_Amin\_Fem\_GP\_1 & 20.393 & 2 & (10, 40) & OK & 1.020 & None \\ 
  3 & Linf\_Fem\_GP\_1 & 176.000 & 2 & (100, 300) & OK & 3.927 & None \\ 
  4 & VonBert\_K\_Fem\_GP\_1 & 11.994 & 1 & (0.005, 30) & OK & 0.312 & None \\ 
  5 & Cessation\_Fem\_GP\_1 & 3.877 & 3 & (0.1, 5) & OK & 6.181 & None \\ 
  6 & SD\_young\_Fem\_GP\_1 & 5.683 & 5 & (1, 20) & OK & 0.916 & None \\ 
  7 & SD\_old\_Fem\_GP\_1 & 7.378 & 5 & (1, 20) & OK & 0.886 & None \\ 
  8 & Wtlen\_1\_Fem\_GP\_1 & 0.000 & -3 & (0, 3) &  &  & None \\ 
  9 & Wtlen\_2\_Fem\_GP\_1 & 2.993 & -3 & (2, 4) &  &  & None \\ 
  10 & Mat50\%\_Fem\_GP\_1 & 148.245 & -3 & (10, 140) &  &  & None \\ 
  11 & Mat\_slope\_Fem\_GP\_1 & -0.132 & -3 & (-0.09, -0.05) &  &  & None \\ 
  12 & Eggs/kg\_inter\_Fem\_GP\_1 & 0.500 & -3 & (-3, 3) &  &  & None \\ 
  13 & Eggs/kg\_slope\_wt\_Fem\_GP\_1 & 0.000 & -3 & (-3, 3) &  &  & None \\ 
  14 & NatM\_p\_1\_Mal\_GP\_1 & 0.000 & -2 & (-3, 3) &  &  & None \\ 
  15 & L\_at\_Amin\_Mal\_GP\_1 & 0.000 & -2 & (-1, 1) &  &  & None \\ 
  16 & Linf\_Mal\_GP\_1 & -0.381 & 2 & (-1, 1) & OK & 0.025 & None \\ 
  17 & VonBert\_K\_Mal\_GP\_1 & 0.109 & 3 & (-10, 20) & OK & 0.032 & None \\ 
  18 & Cessation\_Mal\_GP\_1 & 0.200 & -3 & (-3, 3) &  &  & None \\ 
  19 & SD\_young\_Mal\_GP\_1 & 0.000 & -5 & (-1, 1) &  &  & None \\ 
  20 & SD\_old\_Mal\_GP\_1 & 0.000 & -5 & (-1, 1) &  &  & None \\ 
  21 & Wtlen\_1\_Mal\_GP\_1 & 0.000 & -3 & (0, 3) &  &  & None \\ 
  22 & Wtlen\_2\_Mal\_GP\_1 & 2.993 & -3 & (2, 4) &  &  & None \\ 
  23 & CohortGrowDev & 1.000 & -5 & (0, 2) &  &  & None \\ 
  24 & FracFemale\_GP\_1 & 0.500 & -99 & (0.001, 0.999) &  &  & None \\ 
  25 & SR\_LN(R0) & 8.295 & 1 & (5, 15) & OK & 0.205 & None \\ 
  26 & SR\_BH\_steep & 0.400 & -3 & (0.2, 1) &  &  & None \\ 
  27 & SR\_sigmaR & 0.300 & -2 & (0, 0.4) &  &  & None \\ 
  28 & SR\_regime & 0.000 & -1 & (-2, 2) &  &  & None \\ 
  29 & SR\_autocorr & 0.000 & -99 & (0, 0) &  &  & None \\ 
  78 & LnQ\_base\_WCGBTS(5) & -0.144 & 1 & (-2, 2) & OK & 0.187 & Normal (-0.188, 0.187) \\ 
  79 & Q\_extraSD\_WCGBTS(5) & 0.161 & 5 & (0, 2) & OK & 0.057 & None \\ 
  80 & LnQ\_base\_Triennial(6) & -1.382 & 1 & (-10, 2) & OK & 0.559 & None \\ 
  81 & Q\_extraSD\_Triennial(6) & 0.365 & 5 & (0, 2) & OK & 0.146 & None \\ 
  82 & LnQ\_base\_Triennial(6)\_\_1995 & -1.065 & 1 & (-7, 0) & OK & 0.559 & None \\ 
  83 & Size\_DblN\_peak\_(1) & 86.826 & 4 & (80, 150) & OK & 4.112 & None \\ 
  84 & Size\_DblN\_top\_logit\_(1) & -15.000 & -5 & (-15, 4) &  &  & None \\ 
  85 & Size\_DblN\_ascend\_se\_(1) & 7.064 & 4 & (-1, 9) & OK & 0.126 & None \\ 
  86 & Size\_DblN\_descend\_se\_(1) & 20.000 & -5 & (-1, 20) &  &  & None \\ 
  87 & Size\_DblN\_start\_logit\_(1) & -999.000 & -4 & (-999, 9) &  &  & None \\ 
  88 & Size\_DblN\_end\_logit\_(1) & -999.000 & -5 & (-999, 9) &  &  & None \\ 
  89 & Retain\_L\_infl\_(1) & 66.645 & 2 & (15, 150) & OK & 0.629 & None \\ 
  90 & Retain\_L\_width\_(1) & 4.962 & 2 & (0.1, 10) & OK & 0.350 & None \\ 
  91 & Retain\_L\_asymptote\_logit\_(1) & 2.111 & 3 & (-10, 20) & OK & 0.352 & None \\ 
  92 & Retain\_L\_maleoffset\_(1) & 0.000 & -3 & (0, 0) &  &  & None \\ 
  93 & DiscMort\_L\_infl\_(1) & 5.000 & -4 & (5, 15) &  &  & None \\ 
  94 & DiscMort\_L\_width\_(1) & 0.000 & -4 & (0.001, 10) &  &  & None \\ 
  95 & DiscMort\_L\_level\_old\_(1) & 0.500 & -5 & (0, 1) &  &  & None \\ 
  96 & DiscMort\_L\_male\_offset\_(1) & 0.000 & -5 & (0, 0) &  &  & None \\ 
  97 & SzSel\_Fem\_Peak\_(1) & -4.986 & 4 & (-50, 50) & OK & 2.038 & None \\ 
  98 & SzSel\_Fem\_Ascend\_(1) & 0.000 & -4 & (-5, 5) &  &  & None \\ 
  99 & SzSel\_Fem\_Descend\_(1) & 0.000 & -4 & (-5, 5) &  &  & None \\ 
  100 & SzSel\_Fem\_Final\_(1) & 0.000 & -4 & (-5, 5) &  &  & None \\ 
  101 & SzSel\_Fem\_Scale\_(1) & 0.774 & 4 & (0.5, 1.5) & OK & 0.083 & None \\ 
  102 & Size\_DblN\_peak\_WCGBTS(5) & 72.392 & 4 & (50, 150) & OK & 5.639 & None \\ 
  103 & Size\_DblN\_top\_logit\_WCGBTS(5) & -15.000 & -5 & (-15, 4) &  &  & None \\ 
  104 & Size\_DblN\_ascend\_se\_WCGBTS(5) & 6.440 & 4 & (-1, 9) & OK & 0.371 & None \\ 
  105 & Size\_DblN\_descend\_se\_WCGBTS(5) & 10.061 & 5 & (-1, 20) & OK & 1.621 & None \\ 
  106 & Size\_DblN\_start\_logit\_WCGBTS(5) & -5.000 & -4 & (-999, 9) &  &  & None \\ 
  107 & Size\_DblN\_end\_logit\_WCGBTS(5) & -999.000 & -5 & (-999, 9) &  &  & None \\ 
  108 & SzSel\_Fem\_Peak\_WCGBTS(5) & -7.134 & 4 & (-50, 50) & OK & 3.982 & None \\ 
  109 & SzSel\_Fem\_Ascend\_WCGBTS(5) & 0.000 & -4 & (-5, 5) &  &  & None \\ 
  110 & SzSel\_Fem\_Descend\_WCGBTS(5) & 0.000 & -4 & (-5, 5) &  &  & None \\ 
  111 & SzSel\_Fem\_Final\_WCGBTS(5) & 0.000 & -4 & (-5, 5) &  &  & None \\ 
  112 & SzSel\_Fem\_Scale\_WCGBTS(5) & 0.743 & 4 & (0.5, 1.5) & OK & 0.121 & None \\ 
  113 & Size\_DblN\_peak\_Triennial(6) & 176.755 & 4 & (50, 180) & OK & 26.076 & None \\ 
  114 & Size\_DblN\_top\_logit\_Triennial(6) & -15.000 & -5 & (-15, 4) &  &  & None \\ 
  115 & Size\_DblN\_ascend\_se\_Triennial(6) & 8.481 & 4 & (-1, 9) & OK & 0.381 & None \\ 
  116 & Size\_DblN\_descend\_se\_Triennial(6) & 20.000 & -5 & (-1, 20) &  &  & None \\ 
  117 & Size\_DblN\_start\_logit\_Triennial(6) & -4.025 & 4 & (-15, 9) & OK & 0.527 & None \\ 
  118 & Size\_DblN\_end\_logit\_Triennial(6) & -999.000 & -5 & (-999, 9) &  &  & None \\ 
  119 & SzSel\_Fem\_Peak\_Triennial(6) & 0.000 & -4 & (-50, 50) &  &  & None \\ 
  120 & SzSel\_Fem\_Ascend\_Triennial(6) & 0.000 & -4 & (-5, 5) &  &  & None \\ 
  121 & SzSel\_Fem\_Descend\_Triennial(6) & 0.000 & -4 & (-5, 5) &  &  & None \\ 
  122 & SzSel\_Fem\_Final\_Triennial(6) & 0.000 & -4 & (-5, 5) &  &  & None \\ 
  123 & SzSel\_Fem\_Scale\_Triennial(6) & 0.600 & 4 & (0.5, 1.5) & OK & 0.128 & None \\ 
  124 & Retain\_L\_asymptote\_logit\_\_2005 & 2.325 & 4 & (-10, 20) & OK & 0.562 & None \\ 
  125 & Retain\_L\_asymptote\_logit\_\_2006 & 3.330 & 4 & (-10, 20) & OK & 1.315 & None \\ 
  126 & Retain\_L\_asymptote\_logit\_\_2007 & 4.000 & 4 & (-10, 20) & OK & 2.027 & None \\ 
  127 & Retain\_L\_asymptote\_logit\_\_2008 & 11.158 & 4 & (-10, 20) & OK & 111.095 & None \\ 
  128 & Retain\_L\_asymptote\_logit\_\_2009 & 4.991 & 4 & (-10, 20) & OK & 3.975 & None \\ 
  129 & Retain\_L\_asymptote\_logit\_\_2010 & 13.248 & 4 & (-10, 20) & OK & 88.075 & None \\ 
  130 & Retain\_L\_asymptote\_logit\_\_2011 & 14.665 & 4 & (-10, 20) & OK & 73.786 & None \\ 
  131 & Retain\_L\_asymptote\_logit\_\_2012 & 13.918 & 4 & (-10, 20) & OK & 81.260 & None \\ 
  132 & Retain\_L\_asymptote\_logit\_\_2013 & 3.475 & 4 & (-10, 20) & OK & 0.337 & None \\ 
  133 & Retain\_L\_asymptote\_logit\_\_2014 & 3.653 & 4 & (-10, 20) & OK & 0.279 & None \\ 
  134 & Retain\_L\_asymptote\_logit\_\_2015 & 3.430 & 4 & (-10, 20) & OK & 0.263 & None \\ 
  135 & Retain\_L\_asymptote\_logit\_\_2016 & 2.901 & 4 & (-10, 20) & OK & 0.193 & None \\ 
  136 & Retain\_L\_asymptote\_logit\_\_2017 & 2.822 & 4 & (-10, 20) & OK & 0.192 & None \\ 
   \hline
\hline
\label{tab:model_params}
\end{longtable}
\end{landscape}

\FloatBarrier

\begin{table}[ht]
\centering
\caption{Likelihood components from the base model.} 
\label{tab:like_components}
\begin{tabular}{lr}
  \hline
Likelihood component & Value \\ 
  \hline
TOTAL & 1097.30 \\ 
  Catch & 0.00 \\ 
  Survey & -98.12 \\ 
  Length composition & 763.02 \\ 
  Age composition & 421.52 \\ 
  Recruitment & 10.88 \\ 
  Forecast recruitment & 0.00 \\ 
  Parameter priors & 0.00 \\ 
  Parmeter soft bounds & 0.01 \\ 
   \hline
\end{tabular}
\end{table}

\newpage

\begin{longtable}{c>{\centering}p{.6in}>{\centering}p{.6in}>{\centering}p{.6in}>{\centering}p{.6in}>{\centering}p{.8in}>{\centering}p{.8in}c}
\caption{Time-series of population estimates 
                                        from the base-case model. Relative exploitation 
                                        rate is $(1-SPR)/(1-SPR_{50\%})$.} \\ 
  \hline
Year & Total biomass (mt) & Spawning biomass (mt) & Depletion & Age-0 recruits & Total catch (mt) & Relative exploitation rate & SPR \\ 
  \hline  \endfirsthead \caption[]{Time-series of population estimates 
                                        from the base-case model. Relative exploitation 
                                        rate is $(1-SPR)/(1-SPR_{50\%})$.} \label{tab:Timeseries_mod1} \\ \hline Year & Total biomass (mt) & Spawning biomass (mt) & Depletion & Age-0 recruits & Total catch (mt) & Relative exploitation rate & SPR \\ \hline  \endhead \hline \multicolumn{5}{l}{\textit{Continues next page}} \ 
                                 \endfoot
                                 \endlastfoot \hline
1916 & 24263 & 1526 & 0.000 & 4004 & 0 & 0.00 & 1.00 \\ 
  1917 & 24263 & 1526 & 0.000 & 4004 & 12 & 0.00 & 0.99 \\ 
  1918 & 24251 & 1525 & 0.999 & 4003 & 25 & 0.00 & 0.99 \\ 
  1919 & 24228 & 1524 & 0.998 & 4001 & 37 & 0.00 & 0.98 \\ 
  1920 & 24196 & 1521 & 0.997 & 3999 & 49 & 0.00 & 0.98 \\ 
  1921 & 24156 & 1518 & 0.995 & 3996 & 62 & 0.00 & 0.97 \\ 
  1922 & 24108 & 1514 & 0.992 & 3992 & 74 & 0.00 & 0.97 \\ 
  1923 & 24054 & 1510 & 0.989 & 3987 & 86 & 0.00 & 0.96 \\ 
  1924 & 23994 & 1504 & 0.986 & 3982 & 99 & 0.00 & 0.96 \\ 
  1925 & 23928 & 1498 & 0.982 & 3976 & 111 & 0.00 & 0.95 \\ 
  1926 & 23857 & 1492 & 0.977 & 3969 & 123 & 0.01 & 0.95 \\ 
  1927 & 23780 & 1485 & 0.973 & 3962 & 136 & 0.01 & 0.94 \\ 
  1928 & 23699 & 1477 & 0.968 & 3954 & 148 & 0.01 & 0.94 \\ 
  1929 & 23614 & 1469 & 0.963 & 3946 & 160 & 0.01 & 0.93 \\ 
  1930 & 23524 & 1461 & 0.957 & 3938 & 172 & 0.01 & 0.92 \\ 
  1931 & 23430 & 1453 & 0.952 & 3929 & 185 & 0.01 & 0.92 \\ 
  1932 & 23332 & 1444 & 0.946 & 3920 & 197 & 0.01 & 0.91 \\ 
  1933 & 23231 & 1435 & 0.940 & 3911 & 210 & 0.01 & 0.91 \\ 
  1934 & 23126 & 1426 & 0.934 & 3901 & 222 & 0.01 & 0.90 \\ 
  1935 & 23018 & 1416 & 0.928 & 3890 & 234 & 0.01 & 0.90 \\ 
  1936 & 22907 & 1406 & 0.921 & 3880 & 246 & 0.01 & 0.89 \\ 
  1937 & 22794 & 1396 & 0.915 & 3868 & 259 & 0.01 & 0.89 \\ 
  1938 & 22677 & 1386 & 0.908 & 3857 & 271 & 0.01 & 0.88 \\ 
  1939 & 22558 & 1375 & 0.901 & 3845 & 329 & 0.02 & 0.86 \\ 
  1940 & 22393 & 1361 & 0.892 & 3830 & 329 & 0.02 & 0.86 \\ 
  1941 & 22242 & 1348 & 0.884 & 3815 & 363 & 0.02 & 0.84 \\ 
  1942 & 22069 & 1334 & 0.874 & 3798 & 351 & 0.02 & 0.84 \\ 
  1943 & 21922 & 1320 & 0.865 & 3783 & 343 & 0.02 & 0.85 \\ 
  1944 & 21794 & 1308 & 0.857 & 3769 & 350 & 0.02 & 0.84 \\ 
  1945 & 21669 & 1296 & 0.850 & 3754 & 364 & 0.02 & 0.84 \\ 
  1946 & 21539 & 1284 & 0.842 & 3740 & 379 & 0.02 & 0.83 \\ 
  1947 & 21402 & 1272 & 0.834 & 3725 & 394 & 0.02 & 0.82 \\ 
  1948 & 21258 & 1260 & 0.826 & 3710 & 412 & 0.02 & 0.81 \\ 
  1949 & 21106 & 1248 & 0.818 & 3694 & 426 & 0.02 & 0.81 \\ 
  1950 & 20951 & 1235 & 0.809 & 3679 & 424 & 0.02 & 0.81 \\ 
  1951 & 20808 & 1223 & 0.802 & 3664 & 418 & 0.02 & 0.81 \\ 
  1952 & 20681 & 1212 & 0.794 & 3650 & 434 & 0.02 & 0.80 \\ 
  1953 & 20546 & 1201 & 0.787 & 3634 & 515 & 0.03 & 0.76 \\ 
  1954 & 20341 & 1185 & 0.776 & 3613 & 430 & 0.02 & 0.80 \\ 
  1955 & 20232 & 1174 & 0.770 & 3599 & 470 & 0.02 & 0.78 \\ 
  1956 & 20090 & 1162 & 0.762 & 3583 & 434 & 0.02 & 0.79 \\ 
  1957 & 19992 & 1153 & 0.755 & 3570 & 439 & 0.02 & 0.79 \\ 
  1958 & 19892 & 1144 & 0.750 & 3558 & 426 & 0.02 & 0.80 \\ 
  1959 & 19809 & 1136 & 0.745 & 3547 & 435 & 0.02 & 0.79 \\ 
  1960 & 19720 & 1129 & 0.740 & 3537 & 427 & 0.02 & 0.79 \\ 
  1961 & 19641 & 1122 & 0.735 & 3528 & 487 & 0.03 & 0.77 \\ 
  1962 & 19506 & 1113 & 0.729 & 3515 & 465 & 0.03 & 0.77 \\ 
  1963 & 19401 & 1105 & 0.724 & 3504 & 473 & 0.03 & 0.77 \\ 
  1964 & 19293 & 1097 & 0.719 & 3492 & 468 & 0.03 & 0.77 \\ 
  1965 & 19197 & 1090 & 0.714 & 3481 & 438 & 0.02 & 0.78 \\ 
  1966 & 19136 & 1084 & 0.710 & 3473 & 444 & 0.02 & 0.78 \\ 
  1967 & 19071 & 1078 & 0.706 & 3464 & 463 & 0.03 & 0.77 \\ 
  1968 & 18991 & 1071 & 0.702 & 3453 & 497 & 0.03 & 0.76 \\ 
  1969 & 18881 & 1062 & 0.696 & 3440 & 460 & 0.03 & 0.77 \\ 
  1970 & 18812 & 1056 & 0.692 & 3432 & 416 & 0.02 & 0.79 \\ 
  1971 & 18788 & 1054 & 0.690 & 3427 & 409 & 0.02 & 0.79 \\ 
  1972 & 18770 & 1052 & 0.689 & 3424 & 423 & 0.02 & 0.79 \\ 
  1973 & 18737 & 1049 & 0.687 & 3420 & 429 & 0.02 & 0.78 \\ 
  1974 & 18697 & 1046 & 0.686 & 3416 & 415 & 0.02 & 0.79 \\ 
  1975 & 18671 & 1045 & 0.684 & 3414 & 429 & 0.02 & 0.78 \\ 
  1976 & 18631 & 1042 & 0.683 & 3410 & 440 & 0.02 & 0.78 \\ 
  1977 & 18584 & 1039 & 0.681 & 3406 & 452 & 0.03 & 0.77 \\ 
  1978 & 18527 & 1036 & 0.679 & 3400 & 536 & 0.03 & 0.73 \\ 
  1979 & 18393 & 1027 & 0.673 & 3387 & 584 & 0.03 & 0.71 \\ 
  1980 & 18224 & 1015 & 0.665 & 3368 & 444 & 0.03 & 0.77 \\ 
  1981 & 18202 & 1011 & 0.663 & 3362 & 547 & 0.03 & 0.72 \\ 
  1982 & 18083 & 1001 & 0.656 & 3346 & 486 & 0.03 & 0.75 \\ 
  1983 & 18030 & 995 & 0.652 & 3336 & 466 & 0.03 & 0.76 \\ 
  1984 & 17998 & 991 & 0.649 & 3329 & 420 & 0.02 & 0.78 \\ 
  1985 & 18008 & 989 & 0.648 & 3327 & 453 & 0.03 & 0.76 \\ 
  1986 & 17981 & 987 & 0.647 & 3323 & 425 & 0.03 & 0.78 \\ 
  1987 & 17977 & 987 & 0.647 & 3323 & 431 & 0.03 & 0.77 \\ 
  1988 & 17965 & 987 & 0.647 & 3324 & 415 & 0.02 & 0.78 \\ 
  1989 & 17965 & 989 & 0.648 & 3326 & 413 & 0.02 & 0.78 \\ 
  1990 & 17967 & 991 & 0.649 & 3329 & 422 & 0.02 & 0.78 \\ 
  1991 & 17960 & 991 & 0.650 & 3330 & 432 & 0.03 & 0.77 \\ 
  1992 & 17944 & 991 & 0.649 & 3329 & 424 & 0.02 & 0.78 \\ 
  1993 & 17938 & 990 & 0.649 & 3329 & 438 & 0.03 & 0.77 \\ 
  1994 & 17921 & 989 & 0.648 & 3326 & 438 & 0.03 & 0.77 \\ 
  1995 & 17905 & 987 & 0.647 & 3323 & 119 & 0.01 & 0.93 \\ 
  1996 & 18199 & 1003 & 0.657 & 3348 & 347 & 0.02 & 0.82 \\ 
  1997 & 18257 & 1006 & 0.659 & 3353 & 594 & 0.03 & 0.71 \\ 
  1998 & 18075 & 994 & 0.651 & 3334 & 219 & 0.01 & 0.88 \\ 
  1999 & 18268 & 1004 & 0.658 & 3351 & 318 & 0.02 & 0.83 \\ 
  2000 & 18354 & 1010 & 0.662 & 3360 & 406 & 0.02 & 0.79 \\ 
  2001 & 18349 & 1010 & 0.662 & 3361 & 245 & 0.01 & 0.87 \\ 
  2002 & 18500 & 1021 & 0.669 & 3377 & 239 & 0.01 & 0.87 \\ 
  2003 & 18643 & 1032 & 0.676 & 3394 & 385 & 0.02 & 0.80 \\ 
  2004 & 18635 & 1034 & 0.677 & 3397 & 285 & 0.02 & 0.85 \\ 
  2005 & 18723 & 1042 & 0.683 & 3409 & 347 & 0.02 & 0.82 \\ 
  2006 & 18747 & 1046 & 0.685 & 3416 & 429 & 0.02 & 0.79 \\ 
  2007 & 18697 & 1045 & 0.685 & 3414 & 292 & 0.02 & 0.85 \\ 
  2008 & 18786 & 1051 & 0.689 & 3423 & 387 & 0.02 & 0.81 \\ 
  2009 & 18783 & 1050 & 0.688 & 3422 & 217 & 0.01 & 0.89 \\ 
  2010 & 18946 & 1059 & 0.694 & 3436 & 207 & 0.01 & 0.89 \\ 
  2011 & 19107 & 1069 & 0.700 & 3450 & 282 & 0.02 & 0.86 \\ 
  2012 & 19180 & 1074 & 0.704 & 3458 & 282 & 0.02 & 0.86 \\ 
  2013 & 19245 & 1080 & 0.708 & 3467 & 144 & 0.01 & 0.92 \\ 
  2014 & 19436 & 1095 & 0.718 & 3489 & 397 & 0.02 & 0.81 \\ 
  2015 & 19370 & 1095 & 0.718 & 3489 & 351 & 0.02 & 0.83 \\ 
  2016 & 19357 & 1098 & 0.719 & 3493 & 441 & 0.02 & 0.79 \\ 
  2017 & 19265 & 1094 & 0.717 & 3487 & 297 & 0.02 & 0.85 \\ 
  2018 & 19324 & 1097 & 0.719 & 3492 & 185 & 0.01 & 0.90 \\ 
  2019 & 19491 & 1106 & 0.725 & 3505 &  &  &  \\ 
   \hline
\hline
\end{longtable}

\FloatBarrier

\begin{landscape}

\begin{table}[ht]
\centering
\caption{Sensitivity of the base model 
                                          to assumptions about selectivity and
                                          catchability.} 
\label{tab:Sensitivity_sel_and_Q}
\scalebox{0.9}{
\begin{tabular}{l>{\centering}p{.8in}>{\centering}p{.8in}>{\centering}p{.8in}>{\centering}p{.8in}>{\centering}p{.8in}>{\centering}p{.8in}}
  \hline
Label & Base model & Sel all asymptotic & Sel all domed & Sel no sex offset & Q no prior on WCGBTS & Q no offset on triennial \\ 
  \hline
TOTAL like & 441.63 & 441.41 & 441.63 & 441.63 & 441.13 & 444.19 \\ 
  Survey like & -9.78 & -9.78 & -9.78 & -9.78 & -10.06 & -8.37 \\ 
  Length comp like & 366.25 & 366.14 & 366.25 & 366.25 & 366.93 & 366.81 \\ 
  Age comp like & 110.51 & 110.44 & 110.51 & 110.51 & 110.12 & 110.30 \\ 
  Parm priors like & 1.12 & 1.09 & 1.12 & 1.12 & 0.66 & 1.99 \\ 
  Size at age like & 0.00 & 0.00 & 0.00 & 0.00 & 0.00 & 0.00 \\ 
  Recr Virgin millions & 4.00 & 3.95 & 4.00 & 4.00 & 2.81 & 3.33 \\ 
  log(R0) & 8.29 & 8.28 & 8.29 & 8.29 & 7.94 & 8.11 \\ 
  NatM Female  & 0.38 & 0.38 & 0.38 & 0.38 & 0.38 & 0.39 \\ 
  NatM Male  & 0.38 & 0.38 & 0.38 & 0.38 & 0.38 & 0.39 \\ 
  Linf Female  & 176.00 & 175.90 & 176.00 & 176.00 & 175.97 & 176.05 \\ 
  Linf Male  & 120.24 & 120.20 & 120.24 & 120.24 & 120.38 & 120.21 \\ 
  Q WCGBTS & 0.87 & 0.87 & 0.87 & 0.87 & 1.48 & 1.03 \\ 
  SSB Virgin thousand mt & 1.53 & 1.52 & 1.53 & 1.53 & 1.16 & 1.22 \\ 
  SSB 2019 thousand mt & 1.11 & 1.10 & 1.11 & 1.11 & 0.65 & 0.78 \\ 
  Bratio 2019 & 0.72 & 0.72 & 0.72 & 0.72 & 0.56 & 0.64 \\ 
  SPRratio 2018 & 0.19 & 0.19 & 0.19 & 0.19 & 0.31 & 0.25 \\ 
  Ret Catch MSY & 517.36 & 513.43 & 517.36 & 517.36 & 380.95 & 425.03 \\ 
  Dead Catch MSY & 559.33 & 555.07 & 559.33 & 559.33 & 410.59 & 458.73 \\ 
   \hline
\end{tabular}
}
\end{table}
\end{landscape}
\FloatBarrier

\newpage

\begin{landscape}

\begin{table}[ht]
\centering
\caption{Sensitivity of the base model 
                                          to assumptions about catches.} 
\label{tab:Sensitivity_catch}
\scalebox{0.9}{
\begin{tabular}{l>{\centering}p{.8in}>{\centering}p{.8in}>{\centering}p{.8in}>{\centering}p{.8in}}
  \hline
Label & Base model & Discards based on 3yr averages & Discard mortality   0 4 & Discard mortality   0 6 \\ 
  \hline
TOTAL like & 441.63 & 440.89 & 441.18 & 442.05 \\ 
  Survey like & -9.78 & -10.00 & -10.08 & -9.50 \\ 
  Length comp like & 366.25 & 365.86 & 366.41 & 366.12 \\ 
  Age comp like & 110.51 & 110.53 & 110.46 & 110.54 \\ 
  Parm priors like & 1.12 & 1.13 & 1.06 & 1.17 \\ 
  Size at age like & 0.00 & 0.00 & 0.00 & 0.00 \\ 
  Recr Virgin millions & 4.00 & 3.91 & 4.02 & 4.01 \\ 
  log(R0) & 8.29 & 8.27 & 8.30 & 8.30 \\ 
  NatM Female  & 0.38 & 0.38 & 0.38 & 0.38 \\ 
  NatM Male  & 0.38 & 0.38 & 0.38 & 0.38 \\ 
  Linf Female  & 176.00 & 176.08 & 176.04 & 175.95 \\ 
  Linf Male  & 120.24 & 120.24 & 120.25 & 120.24 \\ 
  Q WCGBTS & 0.87 & 0.90 & 0.88 & 0.85 \\ 
  SSB Virgin thousand mt & 1.53 & 1.49 & 1.55 & 1.51 \\ 
  SSB 2019 thousand mt & 1.11 & 1.03 & 1.09 & 1.13 \\ 
  Bratio 2019 & 0.72 & 0.69 & 0.70 & 0.75 \\ 
  SPRratio 2018 & 0.19 & 0.20 & 0.19 & 0.19 \\ 
  Ret Catch MSY & 517.36 & 505.17 & 532.98 & 506.35 \\ 
  Dead Catch MSY & 559.33 & 545.81 & 567.38 & 555.88 \\ 
   \hline
\end{tabular}
}
\end{table}
\end{landscape}

\FloatBarrier

\newpage

\begin{landscape}

\begin{table}[ht]
\centering
\caption{Sensitivity of the base model 
                                          to assumptions about biology and misc.} 
\label{tab:Sensitivity_bio_and_misc}
\scalebox{0.9}{
\begin{tabular}{l>{\centering}p{.8in}>{\centering}p{.8in}>{\centering}p{.8in}>{\centering}p{.8in}>{\centering}p{.8in}>{\centering}p{.8in}}
  \hline
Label & Base model & Bio separate M by sex & Bio no M prior & Bio von bertalanffy growth & Bio Richards growth & Misc  McAllister Ianelli tuning \\ 
  \hline
TOTAL like & 441.63 & 441.14 & 437.95 & 486.39 & 497.09 & 1132.64 \\ 
  Survey like & -9.78 & -9.86 & -9.71 & -9.80 & -9.73 & -9.76 \\ 
  Length comp like & 366.25 & 364.91 & 365.51 & 404.62 & 387.88 & 572.70 \\ 
  Age comp like & 110.51 & 111.02 & 108.63 & 117.88 & 144.80 & 594.48 \\ 
  Parm priors like & 1.12 & 1.45 & 0.01 & 0.04 & 0.01 & 4.59 \\ 
  Size at age like & 0.00 & 0.00 & 0.00 & 0.00 & 0.00 & 0.00 \\ 
  Recr Virgin millions & 4.00 & 3.47 & 6.29 & 3.26 & 0.00 & 4.82 \\ 
  log(R0) & 8.29 & 8.15 & 8.75 & 8.09 & 8.02 & 8.48 \\ 
  NatM Female  & 0.38 & 0.39 & 0.45 & 0.36 & 0.36 & 0.41 \\ 
  NatM Male  & 0.38 & 0.36 & 0.45 & 0.36 & 0.36 & 0.41 \\ 
  Linf Female  & 176.00 & 176.07 & 175.91 &  & 2666.88 & 177.98 \\ 
  Linf Male  & 120.24 & 119.90 & 120.98 &  & 137.29 & 120.30 \\ 
  Q WCGBTS & 0.87 & 0.88 & 0.81 & 0.83 & 0.85 & 0.87 \\ 
  SSB Virgin thousand mt & 1.53 & 1.28 & 1.12 & 1.37 & 0.00 & 1.42 \\ 
  SSB 2019 thousand mt & 1.11 & 0.86 & 0.84 & 0.90 & 0.00 & 1.04 \\ 
  Bratio 2019 & 0.72 & 0.67 & 0.75 & 0.65 & 0.00 & 0.73 \\ 
  SPRratio 2018 & 0.19 & 0.23 & 0.17 & 0.24 & 0.89 & 0.19 \\ 
  Ret Catch MSY & 517.36 & 456.10 & 564.75 & 446.68 & 0.00 & 530.89 \\ 
  Dead Catch MSY & 559.33 & 492.46 & 610.56 & 482.14 & 0.00 & 573.72 \\ 
   \hline
\end{tabular}
}
\end{table}
\end{landscape}

\FloatBarrier

\newpage

\begin{landscape}
\begin{table}[ht]
\centering
\caption{Summaries of key assessment outputs 
                                              and likelihood values from the retrospective 
                                              analysis.  Note that male 
                                              growth parameters are exponential 
                                              offsets from female parameters, and 
                                              depletion and SPR ratio are for the year of 2017. 
                                              The base model includes all of the data.  Retro1 
                                             removes the last year of data (2016), Retro2 removes the last 
                                             two years of data, Retro3 removes three years and Retro4 
                                             removes four years.} 
\label{tab:retro}
\scalebox{0.9}{
\begin{tabular}{lrrrrr}
  \hline
Label & Base & Retro1 & Retro2 & Retro3 & Retro4 \\ 
  \hline
Female natural mortality & 0.26 & 0.26 & 0.26 & 0.26 & 0.26 \\ 
  Steepness & 0.72 & 0.72 & 0.72 & 0.72 & 0.72 \\ 
  lnR0 & 8.16 & 8.09 & 8.07 & 8.04 & 8.08 \\ 
  Total Biomass (mt) & 2796.86 & 2593.78 & 2568.77 & 2498.07 & 2650.36 \\ 
  Depletion & 57.41 & 53.57 & 50.74 & 50.72 & 54.78 \\ 
  SPR ratio & 0.72 & 0.76 & 0.79 & 0.80 & 0.74 \\ 
  Female Lmin & 12.43 & 12.45 & 12.90 & 12.63 & 13.03 \\ 
  Female Lmax & 33.31 & 33.50 & 33.39 & 33.37 & 33.46 \\ 
  Female K & 0.25 & 0.24 & 0.24 & 0.25 & 0.23 \\ 
  Male Lmin (offset) & 0.00 & 0.00 & 0.00 & 0.00 & 0.00 \\ 
  Male Lmax (offset) & -0.16 & -0.16 & -0.15 & -0.16 & -0.15 \\ 
  Male K (offset) & -0.29 & -0.30 & -0.43 & -0.41 & -0.56 \\ 
  Negative log-likelihood & 1097.30 & 1047.56 & 1009.37 & 961.81 & 897.04 \\ 
  No. parameters & 0.00 & 0.00 & 0.00 & 0.00 & 0.00 \\ 
  TOTAL & 0.00 & 0.00 & 0.00 & 0.00 & 0.00 \\ 
  Equililibrium catch & -98.12 & -92.00 & -89.12 & -81.75 & -80.59 \\ 
  Survey & 763.02 & 739.90 & 720.39 & 700.10 & 670.66 \\ 
  Length composition & 421.52 & 390.56 & 369.97 & 336.26 & 299.84 \\ 
  Age composition & 10.88 & 9.09 & 8.12 & 7.20 & 7.12 \\ 
  Recruitment & 0.00 & 0.00 & 0.00 & 0.00 & 0.00 \\ 
  Forecast Recruitment & 0.00 & 0.00 & 0.00 & 0.00 & 0.00 \\ 
  Parameter priors & 0.01 & 0.01 & 0.01 & 0.01 & 0.01 \\ 
   \hline
\end{tabular}
}
\end{table}
\end{landscape}

\newpage

\begin{landscape}
\begin{table}[ht]
\centering
\caption{Summaries of key assessment outputs 
                                              and likelihood values from selected 
                                              likelihood profile runs on virgin 
                                              recruitment (lnR0) and steepness.  Note that male 
                                              growth parameters are exponential 
                                              offsets from female parameters, and 
                                              depletion and SPR ratio are for the year of 2017.} 
\label{tab:like_profiles}
\begin{tabular}{c|ccccc|ccccc}
  \hline
Label & R07400 & R07800 & R08200 & R08600 & R09000 & h0410 & h0570 & h0710 & h0870 & h0990 \\ 
  \hline
Female M & 0.26 & 0.26 & 0.26 & 0.26 & 0.26 & 0.26 & 0.26 & 0.26 & 0.26 & 0.26 \\ 
  Steepness & 0.72 & 0.72 & 0.72 & 0.72 & 0.72 & 0.41 & 0.57 & 0.71 & 0.87 & 0.99 \\ 
  lnR0 & 7.40 & 7.80 & 8.20 & 8.60 & 9.00 & 8.34 & 8.21 & 8.16 & 8.13 & 8.11 \\ 
  Total biomass (m) & 1623.19 & 2113.03 & 2894.72 & 4173.95 & 6142.97 & 3313.42 & 2943.85 & 2802.69 & 2712.12 & 2667.97 \\ 
  Depletion (\%) & 46.83 & 49.83 & 58.31 & 66.23 & 71.80 & 51.20 & 55.27 & 57.32 & 58.81 & 59.60 \\ 
  SPR ratio & 1.05 & 0.91 & 0.70 & 0.49 & 0.34 & 0.68 & 0.71 & 0.72 & 0.72 & 0.73 \\ 
  Female Lmin & 12.16 & 12.41 & 12.43 & 12.39 & 12.36 & 12.43 & 12.44 & 12.43 & 12.43 & 12.43 \\ 
  Female Lmax & 34.29 & 33.83 & 33.26 & 32.76 & 32.42 & 33.19 & 33.28 & 33.31 & 33.33 & 33.34 \\ 
  Female K & 0.24 & 0.25 & 0.25 & 0.26 & 0.26 & 0.25 & 0.25 & 0.25 & 0.25 & 0.25 \\ 
  Male Lmin (offset) & 0.00 & 0.00 & 0.00 & 0.00 & 0.00 & 0.00 & 0.00 & 0.00 & 0.00 & 0.00 \\ 
  Male Lmax (offset) & -0.18 & -0.17 & -0.16 & -0.15 & -0.15 & -0.16 & -0.16 & -0.16 & -0.16 & -0.16 \\ 
  Male K (offset) & -0.22 & -0.31 & -0.29 & -0.24 & -0.21 & -0.27 & -0.29 & -0.29 & -0.30 & -0.30 \\ 
  Negative log-likelihood &  &  &  &  &  &  &  &  &  &  \\ 
  TOTAL & 1117.15 & 1101.02 & 1097.33 & 1099.69 & 1102.95 & 1101.35 & 1098.58 & 1097.35 & 1096.72 & 1100.21 \\ 
  Catch & 0.00 & 0.00 & 0.00 & 0.00 & 0.00 & 0.00 & 0.00 & 0.00 & 0.00 & 0.00 \\ 
  Equil\_catch & 0.00 & 0.00 & 0.00 & 0.00 & 0.00 & 0.00 & 0.00 & 0.00 & 0.00 & 0.00 \\ 
  Survey & -100.10 & -99.20 & -97.99 & -97.00 & -96.37 & -98.27 & -98.18 & -98.12 & -98.06 & -98.03 \\ 
  Length\_comp & 761.18 & 760.12 & 763.44 & 767.61 & 770.76 & 765.11 & 763.69 & 763.05 & 762.58 & 762.33 \\ 
  Age\_comp & 437.32 & 427.37 & 421.09 & 418.57 & 417.98 & 420.58 & 421.24 & 421.51 & 421.68 & 421.77 \\ 
  Recruitment & 18.74 & 12.72 & 10.80 & 10.50 & 10.58 & 12.55 & 11.40 & 10.90 & 10.56 & 10.38 \\ 
  Forecast\_Recruitment & 0.00 & 0.00 & 0.00 & 0.00 & 0.00 & 0.00 & 0.00 & 0.00 & 0.00 & 0.00 \\ 
  Parm\_priors & 0.00 & 0.00 & 0.00 & 0.00 & 0.00 & 1.38 & 0.42 & 0.01 & -0.04 & 3.76 \\ 
  Parm\_softbounds & 0.01 & 0.01 & 0.01 & 0.01 & 0.01 & 0.01 & 0.01 & 0.01 & 0.01 & 0.01 \\ 
  Parm\_devs & 0.00 & 0.00 & 0.00 & 0.00 & 0.00 & 0.00 & 0.00 & 0.00 & 0.00 & 0.00 \\ 
  Crash\_Pen & 0.00 & 0.00 & 0.00 & 0.00 & 0.00 & 0.00 & 0.00 & 0.00 & 0.00 & 0.00 \\ 
   \hline
\end{tabular}
\end{table}
\begin{table}[ht]
\centering
\caption{Summaries of key assessment outputs 
                                              and likelihood values from selected 
                                              likelihood profile runs on female 
                                              natural mortality.  Note that male 
                                              growth parameters are exponential 
                                              offsets from female parameters, and 
                                              depletion and SPR ratio are for the year of 2017.} 
\label{tab:like_profiles}
\begin{tabular}{c|ccccc}
  \hline
Label & M0220 & M0260 & M0300 & M0350 & M0400 \\ 
  \hline
Female M & 0.22 & 0.26 & 0.30 & 0.35 & 0.40 \\ 
  Steepness & 0.72 & 0.72 & 0.72 & 0.72 & 0.72 \\ 
  lnR0 & 7.67 & 8.20 & 8.95 & 12.21 & 31.00 \\ 
  Total biomass (m) & 2259.39 & 2861.79 & 4632.81 & 89473.50 & 9753570000000.00 \\ 
  Depletion (\%) & 47.72 & 58.15 & 68.08 & 79.27 & 79.74 \\ 
  SPR ratio & 0.97 & 0.70 & 0.41 & 0.02 & 0.00 \\ 
  Female Lmin & 12.39 & 12.44 & 12.43 & 12.39 & 12.24 \\ 
  Female Lmax & 33.23 & 33.31 & 33.31 & 33.25 & 33.73 \\ 
  Female K & 0.25 & 0.25 & 0.25 & 0.25 & 0.24 \\ 
  Male Lmin (offset) & 0.00 & 0.00 & 0.00 & 0.00 & 0.00 \\ 
  Male Lmax (offset) & -0.16 & -0.16 & -0.15 & -0.15 & -0.15 \\ 
  Male K (offset) & -0.27 & -0.30 & -0.31 & -0.32 & -0.36 \\ 
  Negative log-likelihood &  &  &  &  &  \\ 
  TOTAL & 1102.66 & 1096.96 & 1092.96 & 1089.92 & 1091.52 \\ 
  Catch & 0.00 & 0.00 & 0.00 & 0.00 & 0.00 \\ 
  Equil\_catch & 0.00 & 0.00 & 0.00 & 0.00 & 0.00 \\ 
  Survey & -97.79 & -98.14 & -98.33 & -98.33 & -98.95 \\ 
  Length\_comp & 765.50 & 762.85 & 760.88 & 759.19 & 755.26 \\ 
  Age\_comp & 422.97 & 421.41 & 420.05 & 418.75 & 425.16 \\ 
  Recruitment & 11.91 & 10.82 & 10.30 & 10.05 & 9.54 \\ 
  Forecast\_Recruitment & 0.00 & 0.00 & 0.00 & 0.00 & 0.00 \\ 
  Parm\_priors & 0.06 & 0.00 & 0.06 & 0.25 & 0.51 \\ 
  Parm\_softbounds & 0.01 & 0.01 & 0.01 & 0.00 & 0.00 \\ 
  Parm\_devs & 0.00 & 0.00 & 0.00 & 0.00 & 0.00 \\ 
  Crash\_Pen & 0.00 & 0.00 & 0.00 & 0.00 & 0.00 \\ 
   \hline
\end{tabular}
\end{table}
\end{landscape}

\FloatBarrier

\newpage

\newpage
\begin{table}[ht]
\centering
\caption{Projection of potential
                                        OFL, spawning biomass, and depletion for the
                                        base case model.} 
\label{tab:Forecast_mod1}
\begin{tabular}{c>{\centering}p{1in}>{\centering}p{1in}>{\centering}p{1in}>{\centering}p{1in}>{\centering}p{1in}}
  \hline
Yr & OFL contribution (mt) & ACL landings (mt) & Age 5+ biomass (mt) & Spawning Biomass (mt) & Depletion \\ 
  \hline
2019 & 1274.290 & 1185.906 & 18438.000 & 1106.070 & 0.725 \\ 
  2020 & 1211.220 & 1125.230 & 17564.900 & 1048.210 & 0.687 \\ 
  2021 & 1159.120 & 1074.895 & 16847.100 & 993.337 & 0.651 \\ 
  2022 & 1117.470 & 1034.993 & 16248.300 & 941.818 & 0.617 \\ 
  2023 & 1083.860 & 1003.371 & 15744.500 & 893.809 & 0.586 \\ 
  2024 & 1055.150 & 976.699 & 15309.700 & 849.368 & 0.557 \\ 
  2025 & 1029.120 & 952.644 & 14919.700 & 808.738 & 0.530 \\ 
  2026 & 1004.390 & 929.838 & 14555.100 & 772.649 & 0.506 \\ 
  2027 & 980.334 & 907.640 & 14202.800 & 742.174 & 0.486 \\ 
  2028 & 956.747 & 885.819 & 13859.300 & 717.965 & 0.470 \\ 
  2029 & 933.761 & 864.469 & 13527.100 & 699.544 & 0.458 \\ 
  2030 & 911.621 & 843.793 & 13209.600 & 684.888 & 0.449 \\ 
   \hline
\end{tabular}
\end{table}

\FloatBarrier

\newpage

\hypertarget{figures}{%
\section{Figures}\label{figures}}

\begin{figure}
\centering
\includegraphics{Figures/survey_hauls_map.png}
\caption{Map showing the distribution of Big Skate within the area
covered by the West Coast Groundfish Bottom Trawl Survey aggregated over
the years 2003--2018. \label{fig:boundary_map}}
\end{figure}

\begin{figure}
\centering
\includegraphics{r4ss/plots_mod1/data_plot.png}
\caption{Summary of data sources used in the model.
\label{fig:data_plot}}
\end{figure}

\begin{figure}
\centering
\includegraphics{Figures/catch_by_source.png}
\caption{Reconstructed landings by area. Tribal catch was all landed in
Washington. \label{fig:catch_by_state}}
\end{figure}

\begin{figure}
\centering
\includegraphics{Figures/discard_calculations.png}
\caption{Estimated total catch using different assumptions for discards.
\label{fig:discard_calculations}}
\end{figure}

\begin{figure}
\centering
\includegraphics{r4ss/plots_mod1/catch2 landings stacked.png}
\caption{Catch data input to the model under assumed fleet structure.
The historical discards shown in green have been scaled to account for
an assumed 50\% discard mortality. Discards during the period from 1995
onward are not represented here as they are estimated within the model.
\label{fig:catch_input_plot}}
\end{figure}

\FloatBarrier

\begin{figure}
\centering
\includegraphics{Figures/VAST_Yearly_Dens_Triennial.png}
\caption{Map of estimated density by year for Big Skate in the Triennial
survey. \label{fig:VAST_Yearly_Dens_Triennial}}
\end{figure}

\begin{figure}
\centering
\includegraphics{Figures/VAST_Yearly_Dens_WCGBTS.png}
\caption{Map of estimated density by year for Big Skate in the WCGBT
Survey. \label{fig:VAST_Yearly_Dens_Triennial}}
\end{figure}

\begin{figure}
\centering
\includegraphics{Figures/IPHC_BigSkate_map.png}
\caption{Map of catch rates by year for Big Skate in the International
Pacific Halibut Commission longline survey. \label{fig:IPHC_map}}
\end{figure}

\FloatBarrier

\FloatBarrier

\FloatBarrier

\FloatBarrier

\FloatBarrier

\FloatBarrier

\newpage

\begin{figure}
\centering
\includegraphics{r4ss/plots_mod1/comp_lendat__aggregated_across_time.png}
\caption{Length comp data, aggregated across time by fleet.
\label{fig:comp_lendat_aggregated_across_time}}
\end{figure}

\begin{figure}
\centering
\includegraphics{r4ss/plots_mod1/comp_lendat__multi-fleet_comparison.png}
\caption{Length comp data for all years and fleets. Bubble size
indicates the observed proportions, with females in red and males in
blue. \label{fig:comp_lendat__multi-fleet_comparison}}
\end{figure}

\begin{figure}
\centering
\includegraphics{r4ss/plots_mod1/comp_condAALdat_bubflt1mkt2.png}
\caption{Conditional age-at-length data from the fishery.
\label{fig:age_dat_fishery}}
\end{figure}

\begin{figure}
\centering
\includegraphics{r4ss/plots_mod1/comp_condAALdat_bubflt5mkt0.png}
\caption{Conditional age-at-length data from the WCGBT Survey.
\label{fig:age_dat_fishery}}
\end{figure}

\includegraphics{Figures/Reader 1 vs Reader 2.png} Reader 1 vs Reader
2.png

\begin{figure}
\centering
\includegraphics{r4ss/plots_mod1/numbers10_ageerror_matrix_1.png}
\caption{Estimated ageing imprecision.\label{fig:ageing_imprecision}}
\end{figure}

\begin{figure}
\centering
\includegraphics{Figures/Big Skate bio relationships.png}
\caption{Estimated relationship between length and weight (left) and
disc-width and length (right) for Big Skate. Colored points show
observed values and the black line indicates the estimated relationship
\(W = 0.0000074924L^{2.9925}\).\label{fig:weight-length}}
\end{figure}

\begin{figure}
\centering
\includegraphics{Figures/BigSkate_maturity.png}
\caption{Estimated maturity relationship for female Big Skate. Gray
points indicate average observed functional maturity within each length
bin with point size proportional to the number of samples (indicated by
text within each point).\label{fig:maturity}}
\end{figure}

\newpage

\FloatBarrier

\FloatBarrier

\FloatBarrier

\FloatBarrier

\begin{figure}
\centering
\includegraphics{r4ss/plots_mod1/sel01_multiple_fleets_length1.png}
\caption{Selectivity at length for all of the fleets in the base model.
\label{fig:sel01_multiple_fleets_length1}}
\end{figure}

\FloatBarrier

\begin{figure}
\centering
\includegraphics{r4ss/plots_mod1/index2_cpuefit_WCGBT Survey.png}
\caption{Fit to index data for WCGBT Survey. Lines indicate 95\%
uncertainty interval around index values. Thicker lines indicate input
uncertainty before addition of estimated additional uncertainty
parameter. The blue line indicates the model
estimate.\label{fig:index2_cpuefit_WCGBTS}}
\end{figure}

\begin{figure}
\centering
\includegraphics{r4ss/plots_mod1/index2_cpuefit_Triennial Survey.png}
\caption{Fit to index data for Triennial Survey. Lines indicate 95\%
uncertainty interval around index values. Thicker lines indicate input
uncertainty before addition of estimated additional uncertainty
parameter. The blue line indicates the model estimate with a change
between 1992 and 1995 associated with the estimated change in
catchability.\label{fig:index2_cpuefit_Triennial}}
\end{figure}

\begin{figure}
\centering
\includegraphics{r4ss/plots_mod1/comp_lenfit__aggregated_across_time.png}
\caption{Fits to length comp data, aggregated across time by fleet.
\label{fig:comp_lenfit_aggregated_across_time}}
\end{figure}

\begin{figure}
\centering
\includegraphics{r4ss/plots_mod1/comp_lenfit__multi-fleet_comparison.png}
\caption{Pearson residuals for length comp data for all years and
fleets, with females in red and males in blue.
\label{fig:comp_lenfit__multi-fleet_comparison}}
\end{figure}

\begin{figure}
\centering
\includegraphics{r4ss/plots_mod1/comp_condAALfit_residsflt1mkt2.png}
\caption{Pearson residuals for the fit to conditional age-at-length data
from the fishery. \label{fig:age_fit_fishery}}
\end{figure}

\begin{figure}
\centering
\includegraphics{r4ss/plots_mod1/comp_condAALfit_residsflt5mkt0.png}
\caption{Pearson residuals for the fit to conditional age-at-length data
from the WCGBT Survey. \label{fig:age_fit_WCGBTS}}
\end{figure}

\begin{figure}
\centering
\includegraphics{r4ss/plots_mod1/sexratio_len_flt1mkt2.png}
\caption{Observed sex ratios (points) from the fishery length comp data
with 75\% intervals (vertical lines) calculated as a Jeffreys interval
based on the adjusted input sample size. The model expectation is shown
in the blue line.\label{fig:sexratio_len_flt1mkt2}}
\end{figure}

\begin{figure}
\centering
\includegraphics{r4ss/plots_mod1/sexratio_len_flt5mkt0.png}
\caption{Observed sex ratios (points) from the WCGBT Survey length comp
data with 75\% intervals (vertical lines) calculated as a Jeffreys
interval based on the adjusted input sample size. The model expectation
is shown in the blue line.\label{fig:sexratio_len_flt5mkt0}}
\end{figure}

\begin{figure}
\centering
\includegraphics{r4ss/plots_mod1/discard_fitFishery.png}
\caption{Fit to the discard fraction estimates. Points are model
estimates with 95\% uncertainty intervals. The model estimate is shown
in the blue lines.\label{fig:discard_fitFishery}}
\end{figure}

\begin{figure}
\centering
\includegraphics{r4ss/plots_mod1/bodywt_fit_fltFishery.png}
\caption{Fit to the mean weight of the discards. Points are model
estimates with 95\% uncertainty intervals. The model estimate is shown
in the blue lines.\label{fig:bodywt_fit_fltFishery}}
\end{figure}

file:

\FloatBarrier

\begin{figure}
\centering
\includegraphics{r4ss/plots_mod1/ts7_Spawning_output_with_95_asymptotic_intervals_intervals.png}
\caption{Estimated spawning biomass (mt) with approximate 95\%
asymptotic intervals.
\label{fig:ts7_Spawning_biomass_(mt)_with_95_asymptotic_intervals_intervals}}
\end{figure}

\begin{figure}
\centering
\includegraphics{r4ss/plots_mod1/ts9_Spawning_depletion_with_95_asymptotic_intervals_intervals.png}
\caption{Estimated spawning depletion with approximate 95\% asymptotic
intervals.
\label{fig:ts9_Spawning_depletion_with_95_asymptotic_intervals_intervals}}
\end{figure}

\begin{figure}
\centering
\includegraphics{r4ss/plots_mod1/ts11_Age-0_recruits_(1000s)_with_95_asymptotic_intervals.png}
\caption{Estimated time-series of recruitment for Big Skate.
\label{fig:ts11_Age-0_recruits_(1000s)_with_95_asymptotic_intervals}}
\end{figure}

\begin{figure}
\centering
\includegraphics{r4ss/plots_mod1/SR_curve2.png}
\caption{Estimated recruitment and the assumed stock-recruit
relationship. \label{fig:SR_curve2}}
\end{figure}

\begin{figure}
\centering
\includegraphics{r4ss/plots_mod1/catch5 total catch (including discards) stacked.png}
\caption{Estimated total catch including discards estimated within the
model. The historical discards shown in green have been scaled to
account for an assumed 50\% discard mortality but the discards in the
recent period show both live and dead discards.
\label{fig:catch_total_plot}}
\end{figure}

\hypertarget{sensitivity-analyses-for-model}{%
\subsubsection{Sensitivity analyses for
model}\label{sensitivity-analyses-for-model}}

\begin{figure}
\centering
\includegraphics{Figures/sens.sel_and_Q_compare1_spawnbio.png}
\caption{Time series of spawning output estimated in sensitivity
analyses related to selectivity and catchability.
\label{fig:Sensitivity_sel_and_Q}}
\end{figure}

\begin{figure}
\centering
\includegraphics{Figures/sens.catch_compare1_spawnbio.png}
\caption{Time series of spawning output estimated in sensitivity
analyses related to historic catch and discards.
\label{fig:Sensitivity_catch}}
\end{figure}

\begin{figure}
\centering
\includegraphics{Figures/sens.bio_and_misc_compare1_spawnbio.png}
\caption{Time series of spawning output estimated in sensitivity
analyses related to biology and other assumptions.
\label{fig:Sensitivity_bio_and_misc}}
\end{figure}

\begin{figure}
\centering
\includegraphics{Figures/growth_curve_comparison.png}
\caption{Comparison of the estimated growth curves from the
sensitivities analyses.\label{fig:growth_curve_comparison}}
\end{figure}

\FloatBarrier

\begin{figure}
\centering
\includegraphics{Figures/profile_logR0.png}
\caption{Likelihood profile over the log of equilibrium recruitment
(\(R_0\)).\label{fig:profile_logR0}}
\end{figure}

\begin{figure}
\centering
\includegraphics{Figures/profile_R0_compare1_spawnbio.png}
\caption{Time series of spawning output estimated for the models
included in the profile over the log of equilibrium recruitment
(\(R_0\)).\label{fig:profile_R0_compare1_spawnbio}}
\end{figure}

\FloatBarrier

\begin{figure}
\centering
\includegraphics{Figures/profile_h.png}
\caption{Likelihood profile over stock-recruit steepness (\(h\)).
\label{fig:profile_h}}
\end{figure}

\begin{figure}
\centering
\includegraphics{Figures/profile_h_compare1_spawnbio.png}
\caption{Time series of spawning output estimated for the models
included in the profile over stock-recruit steepness (\(h\)).
\label{fig:profile_h_compare1_spawnbio}}
\end{figure}

\FloatBarrier

\begin{figure}
\centering
\includegraphics{Figures/profile_M.png}
\caption{Likelihood profile over natural mortality (\(M\)).
\label{fig:profile_M}}
\end{figure}

\begin{figure}
\centering
\includegraphics{Figures/profile_M_compare1_spawnbio.png}
\caption{Time series of spawning output estimated for the models
included in the profile over natural mortality (\(M\)).
\label{fig:profile_M_compare1_spawnbio}}
\end{figure}

\FloatBarrier

\FloatBarrier

\begin{figure}
\centering
\includegraphics{r4ss/plots_mod1/yield1_yield_curve.png}
\caption{Equilibrium yield curve for the base case model. Values are
based on the 2018 fishery selectivity and with steepness fixed at 0.718.
\label{fig:yield1_yield_curve}}
\end{figure}

\FloatBarrier

\newpage

\FloatBarrier
\newpage

\#Appendix A. Detailed fits to length composition data \{-\}
\renewcommand{\thepage}{A-\arabic{page}}

\renewcommand{\thefigure}{A\arabic{figure}}
\setcounter{page}{1}

\begin{figure}
\centering
\includegraphics{./r4ss/plots_mod1/comp_lenfit_flt1mkt2.png}
\caption{Length comps, retained, Fishery. `N adj.' is the input sample
size after data\_weighting adjustment. N eff. is the calculated
effective sample size used in the McAllister\_Iannelli tuning method.
\label{fig:mod1_1_comp_lenfit_flt1mkt2}}
\end{figure}

\begin{figure}
\centering
\includegraphics{./r4ss/plots_mod1/comp_lenfit_flt1mkt1.png}
\caption{Length comps, discard, Fishery. `N adj.' is the input sample
size after data\_weighting adjustment. N eff. is the calculated
effective sample size used in the McAllister\_Iannelli tuning method.
\label{fig:mod1_2_comp_lenfit_flt1mkt1}}
\end{figure}

\begin{figure}
\centering
\includegraphics{./r4ss/plots_mod1/comp_lenfit_flt5mkt0.png}
\caption{Length comps, whole catch, WCGBT Survey. `N adj.' is the input
sample size after data\_weighting adjustment. N eff. is the calculated
effective sample size used in the McAllister\_Iannelli tuning method.
\label{fig:mod1_3_comp_lenfit_flt5mkt0}}
\end{figure}

\begin{figure}
\centering
\includegraphics{./r4ss/plots_mod1/comp_lenfit_flt6mkt0.png}
\caption{Length comps, whole catch, Triennial Survey. `N adj.' is the
input sample size after data\_weighting adjustment. N eff. is the
calculated effective sample size used in the McAllister\_Iannelli tuning
method. \label{fig:mod1_4_comp_lenfit_flt6mkt0}}
\end{figure}

\newpage

\color{black}

\hypertarget{references}{%
\section*{References}\label{references}}
\addcontentsline{toc}{section}{References}

\renewcommand{\thepage}{}

\hypertarget{refs}{}
\leavevmode\hypertarget{ref-Bizzarro2019}{}%
Bizzarro, J. 2019. Manuscript in preparation.

\leavevmode\hypertarget{ref-Bradburn2011}{}%
Bradburn, M.J. and Keller, A.A and Horness, B.H. 2011. The 2003 to 2008
US West Coast bottom trawl surveys of groundfish resources off
Washington, Oregon, and California: estimates of distribution,
abundance, length, and age composition. NOAA Technical Memorandum NMFS
NOAA-TM-NMFS-NWFSC-114: 323 pp.

\leavevmode\hypertarget{ref-Ebert2008}{}%
Ebert, D.A., Smith, W.D., and Cailliet, G.M. 2008. Reproductive biology
of two commercially exploited skates, raja binoculata and r. Rhina, in
the western gulf of alaska. Fisheries Research \textbf{94}(1): 48--57.
Elsevier.

\leavevmode\hypertarget{ref-Francis2011}{}%
Francis, R.I.C.C. 2011. Data weighting in statistical fisheries stock
assessment models. Canadian Journal of Fisheries and Aquatic Sciencies
\textbf{68}: 1124--1138.

\leavevmode\hypertarget{ref-Gertseva2019}{}%
Gertseva, V. 2019a. Manuscript in preparation.

\leavevmode\hypertarget{ref-Gertseva2011}{}%
Gertseva, V. 2019b. Manuscript in preparation.

\leavevmode\hypertarget{ref-Gunderson1980}{}%
Gunderson, Donald Raymond and Sample, Terrance M. 1980. Distribution and
abundance of rockfish off Washington, Oregon and California during 1977.
Northwest and Alaska Fisheries Center, National Marine Fisheries
Service. Available from
\href{\%7Bhttp://spo.nmfs.noaa.gov/mfr423-4/mfr423-42.pdf\%7D}{\{http://spo.nmfs.noaa.gov/mfr423-4/mfr423-42.pdf\}}.

\leavevmode\hypertarget{ref-Hamel2015}{}%
Hamel, Owen S. 2015. A method for calculating a meta-analytical prior
for the natural mortality rate using multiple life history correlates.
ICES Journal of Marine Science: Journal du Conseil \textbf{72}(1):
62--69. doi:
\href{https://doi.org/\%7B10.1093/icesjms/fsu131\%7D}{\{10.1093/icesjms/fsu131\}}.

\leavevmode\hypertarget{ref-Keller2017}{}%
Keller, A.A. and Wallace, J.R. and Methot, R.D. 2017. The Northwest
Fisheries Science Center's West Coast Groundfish Bottom Trawl Survey:
History, Design, and Description. NOAA Technical Memorandum NMFS
NOAA-TM-NMFS-NWFSC-136: 38 pp.

\leavevmode\hypertarget{ref-maunder2018growth}{}%
Maunder, M.N., Deriso, R.B., Schaefer, K.M., Fuller, D.W.,
Aires-da-Silva, A.M., Minte-Vera, C.V., and Campana, S.E. 2018. The
growth cessation model: A growth model for species showing a near
cessation in growth with application to bigeye tuna (thunnus obesus).
Marine biology \textbf{165}(4): 76. Springer.

\leavevmode\hypertarget{ref-McFandKing2006}{}%
McFarlane GA and King JR. 2006. Age and growth of big skate (\emph{Raja
binoculata}) and longnose skate (\emph{Raja rhina}) in British Columbia
waters. Fisheries Research \textbf{May 1 (2-3)}: 169--78.

\leavevmode\hypertarget{ref-Methot2013}{}%
Methot, Richard D. and Wetzel, Chantell R. 2013. Stock synthesis: A
biological and statistical framework for fish stock assessment and
fishery management. Fisheries Research \textbf{142}: 86--99.

\leavevmode\hypertarget{ref-Taylor2019}{}%
Taylor, I.G., Stewart, I.J., Hicks, A.C., Garrison, T.M., Punt, A.E.,
Wallace, J.R., Wetzel, C.R., Thorson, J.T., Takeuchi, Y., Ono, K.,
Monnahan, C.C., Stawitz, C.C., A'mar, Z.T., Whitten, A.R., Johnson,
K.F., Emmet, R.L., Anderson, S.C., Lambert, G.I., Stachura, M.M.,
Cooper, A.B., Stephens, A., Klaer, N.L., McGilliard, C.R., Iwasaki,
W.M., Doering, K., and Havron, A.M. 2019. R4ss: R code for stock
synthesis. Available from \url{https://github.com/r4ss}.

\leavevmode\hypertarget{ref-Thorson2017a}{}%
Thorson, James T. and Barnett, Lewis A. K. 2017. Comparing estimates of
abundance trends and distribution shifts using single- and multispecies
models of fishes and biogenic habitat. ICES Journal of Marine Science:
Journal du Conseil: fsw193. doi:
\href{https://doi.org/\%7B10.1093/icesjms/fsw193\%7D}{\{10.1093/icesjms/fsw193\}}.

\leavevmode\hypertarget{ref-Thorson2015}{}%
Thorson, J. T. and Shelton, A. O. and Ward, E. J. and Skaug, H. J. 2015.
Geostatistical delta-generalized linear mixed models improve precision
for estimated abundance indices for West Coast groundfishes. ICES
Journal of Marine Science \textbf{72}(5): 1297--1310. doi:
\href{https://doi.org/\%7B10.1093/icesjms/fsu243\%7D}{\{10.1093/icesjms/fsu243\}}.

\end{document}
