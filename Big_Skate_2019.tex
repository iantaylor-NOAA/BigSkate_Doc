\documentclass[12pt,]{article}
%\usepackage{lmodern}  Melissa removed to deal with font rendering issue
\usepackage{amssymb,amsmath}
\usepackage{ifxetex,ifluatex}
\usepackage{fixltx2e} % provides \textsubscript

%Melissa removed the following section to deal with font rendering issue
%\ifnum 0\ifxetex 1\fi\ifluatex 1\fi=0 % if pdftex
%  \usepackage[T1]{fontenc}
%  \usepackage[utf8]{inputenc}
%%\else % if luatex or xelatex
%  \ifxetex
%    \usepackage{mathspec}
%  \else
%    \usepackage{fontspec}
%  \fi
%  \defaultfontfeatures{Ligatures=TeX,Scale=MatchLowercase}
%  \newcommand{\euro}{€}
%%%%%%\fi

% use upquote if available, for straight quotes in verbatim environments
\IfFileExists{upquote.sty}{\usepackage{upquote}}{}
% use microtype if available
\IfFileExists{microtype.sty}{%
\usepackage{microtype}
\UseMicrotypeSet[protrusion]{basicmath} % disable protrusion for tt fonts
}{}
\usepackage[margin=1in]{geometry}
\usepackage{hyperref}
\PassOptionsToPackage{usenames,dvipsnames}{color} % color is loaded by hyperref
\hypersetup{unicode=true,
            pdftitle={Status of Big Skate (Beringraja binoculata) Off the U.S. Pacific Coast in 2019},
            pdfborder={0 0 0},
            breaklinks=true}
\urlstyle{same}  % don't use monospace font for urls
\usepackage{graphicx,grffile}
\makeatletter
\def\maxwidth{\ifdim\Gin@nat@width>\linewidth\linewidth\else\Gin@nat@width\fi}
\def\maxheight{\ifdim\Gin@nat@height>\textheight\textheight\else\Gin@nat@height\fi}
\makeatother
% Scale images if necessary, so that they will not overflow the page
% margins by default, and it is still possible to overwrite the defaults
% using explicit options in \includegraphics[width, height, ...]{}
\setkeys{Gin}{width=\maxwidth,height=\maxheight,keepaspectratio}
\setlength{\parindent}{0pt}
\setlength{\parskip}{6pt plus 2pt minus 1pt}
\setlength{\emergencystretch}{3em}  % prevent overfull lines
\providecommand{\tightlist}{%
  \setlength{\itemsep}{0pt}\setlength{\parskip}{0pt}}
\setcounter{secnumdepth}{5}

%%% Use protect on footnotes to avoid problems with footnotes in titles
\let\rmarkdownfootnote\footnote%
\def\footnote{\protect\rmarkdownfootnote}

%%% Change title format to be more compact
\usepackage{titling}

% Create subtitle command for use in maketitle
\newcommand{\subtitle}[1]{
  \posttitle{
    \begin{center}\large#1\end{center}
    }
}

\setlength{\droptitle}{-2em}
  \title{Status of Big Skate (\emph{Beringraja binoculata}) Off the U.S. Pacific
Coast in 2019}
  \pretitle{\vspace{\droptitle}\centering\huge}
  \posttitle{\par}
  \author{}
  \preauthor{}\postauthor{}
  \date{}
  \predate{}\postdate{}


% This file contains all of the LaTeX packages you may need to compile the document
% Documentation for each package can be found onlines
\usepackage{tabularx}                                             % table environment providing flexibility
\usepackage{caption}                                              % for creating captions  
\usepackage{longtable}                                            % allows tables to span multiple pages
\usepackage{rotating}                                             % allows for sideways tables
\usepackage{float}                                                % floating environments; may not need in rmarkdown
\usepackage{placeins}                                             % keeps floats from moving
\usepackage{indentfirst}                                          % indents first paragraph of a section
\usepackage{mdwtab}                                               % continued float multi-page figure
\usepackage{enumerate}                                            % create lists
\usepackage{hyperref}                                             % highlight cross references
\hypersetup{colorlinks=true, urlcolor=blue, linktoc=page, linkcolor=blue, citecolor=blue} %define referencing colors
%\usepackage{makebox}                                             % make boxes around text
\usepackage[usenames,dvipsnames]{xcolor}                          % color name options
%\usepackage[space]{grffile}                                      % spaces in file name path
\usepackage{soul}                                                 % highlight text
\usepackage{enumitem}                                             % numbered lists
\usepackage{lineno}                                               % Line numbers; comment out for final
\usepackage{upquote}                                              % produce grave accent in latex
\usepackage{verbatim}                                             % produces verbatim results
\usepackage{fancyvrb}                                             % verbatim in a box
%\usepackage{draftwatermark}                                      % places Draft watermark in background; comment out for final
\usepackage{textcomp}                                             % fixes error with packages interfering
\usepackage{lscape}                                               % rotate pages - to allow for landscape longtables
%\pdfinterwordspaceon                                             % fix loss of inter word spacing
\usepackage{cmap}                                                 % fix mapping characters to unicode
\RequirePackage[linewidth = 1]{pdfcomment}                        % pdf comments
\RequirePackage[l2tabu, orthodox]{nag}                            % checks packages related to the accessibility?
\usepackage[inline]{showlabels}                                   % show table and figure labels; comment out for final
%\RequirePackage[tagged]{accessibilityMeta}


\linenumbers                                                      % specify use of line numbers


\definecolor{light-gray}{gray}{.85}                               % define light-gray as a color
%\usepackage[tagged]{accessibility-meta}

 
%\showlabels[\color{mred}]{label}

% Redefines (sub)paragraphs to behave more like sections
\ifx\paragraph\undefined\else
\let\oldparagraph\paragraph
\renewcommand{\paragraph}[1]{\oldparagraph{#1}\mbox{}}
\fi
\ifx\subparagraph\undefined\else
\let\oldsubparagraph\subparagraph
\renewcommand{\subparagraph}[1]{\oldsubparagraph{#1}\mbox{}}
\fi

\begin{document}
\maketitle


\begin{center}
\thispagestyle{empty}

\vspace{.7cm}

% \includegraphics{cover_photo}~\\[1cm]
\pdftooltip{\includegraphics{cover_photo}}{This is a fish.}

\vspace{.5cm}

Ian G. Taylor\textsuperscript{1}\\
Vladlena Gertseva\textsuperscript{1}\\
Joseph Bizzarro\textsuperscript{2}\\
Andi Stephens\textsuperscript{3}\\

\vspace{.7cm}

\small

\textsuperscript{1}Northwest Fisheries Science Center, U.S. Department of Commerce, National Oceanic and Atmospheric Administration, National Marine Fisheries Service, 2725 Montlake Boulevard East, Seattle, Washington 98112\\

\vspace{.3cm}

\textsuperscript{2}Southwest Fisheries Science Center, U.S. Department of Commerce, National Oceanic and Atmospheric Administration, National Marine Fisheries Service, 110 Shaffer Road, Santa Cruz, California 95060\\

\vspace{.3cm}

\textsuperscript{3}Northwest Fisheries Science Center, U.S. Department of Commerce, National Oceanic and Atmospheric Administration, National Marine Fisheries Service, 2032 S.E. OSU Drive Newport, Oregon 97365


\vspace{.5cm}

\vfill
DRAFT SAFE\\
Disclaimer: This information is distributed solely for the purpose of pre-dissemination
peer review under applicable information quality guidelines. It has not been formally
disseminated by NOAA Fisheries. It does not represent and should not be construed to
represent any agency determination or policy. 

\vspace{.3cm}
%Bottom of the page
%{\large \today}


\newpage{\thispagestyle{empty}}

\vspace*{\fill}
\begin{flushleft}
This report may be cited as:

Taylor, I.G., Gertseva, V., Bizzarro, J., and Stephens, A. Status of Big Skate (\emph{Beringraja binoculata}) Off the U.S. West Coast, 2019. Pacific Fishery Management Council, Portland, OR. Available from http://www.pcouncil.org/groundfish/stock-assessments/
\end{flushleft}

\maketitle

\pagenumbering{roman}
\setcounter{page}{1}
\end{center}

{
\setcounter{tocdepth}{4}
\tableofcontents
}
\setlength{\parskip}{5mm plus1mm minus1mm}
\pagebreak

\pagenumbering{arabic}

\renewcommand{\thefigure}{\alph{figure}}
\renewcommand{\thetable}{\alph{table}}

\hypertarget{executive-summary}{%
\section*{Executive Summary}\label{executive-summary}}
\addcontentsline{toc}{section}{Executive Summary}

\hypertarget{stock}{%
\subsection*{Stock}\label{stock}}
\addcontentsline{toc}{subsection}{Stock}

This assessment reports the status of the Big Skate
(\emph{Beringraja binoculata}) resource in U.S. waters off the coast of
\ldots{} using data through 2018.

\hypertarget{catches}{%
\subsection*{Catches}\label{catches}}
\addcontentsline{toc}{subsection}{Catches}

Information on historical landings of Big Skate are available back to
xxxx\ldots{} (Table \ref{tab:Exec_catch}). Commercial landings were
small during the years of World War II, ranging between 329 to 395
metric tons (mt) per year.

(Figures \ref{fig:Exec_catch1}-\ref{fig:Exec_catch2})\\
(Figure \ref{fig:r4ss_catches})

Since 2000, annual total landings of Big Skate have ranged between
135-412 mt, with landings in 2018 totaling 173 mt.

\FloatBarrier

\begin{figure}
\centering
\includegraphics{Big_Skate_2019_files/figure-latex/unnamed-chunk-15-1.pdf}
\caption{Big Skate catch history for the recreational fleets.
\label{fig:Exec_catch1}}
\end{figure}

\begin{figure}
\centering
\includegraphics{Big_Skate_2019_files/figure-latex/unnamed-chunk-16-1.pdf}
\caption{Stacked line plot of Big Skate catch history for the commercial
fleets. \label{fig:Exec_catch2}}
\end{figure}

\FloatBarrier

\begin{figure}
\centering
\includegraphics{r4ss/plots_mod1/catch2 landings stacked.png}
\caption{Catch history of Big Skate in the model.
\label{fig:r4ss_catches}}
\end{figure}

\begin{table}[ht]
\centering
\caption{Recent Big Skate landings (mt) by 
                                            fleet.} 
\label{tab:Exec_catch}
\begin{tabular}{l>{\centering}p{1in}>{\centering}p{1in}>{\centering}p{1in}>{\centering}p{.9in}>{\centering}p{.9in}>{\centering}p{.6in}}
  \hline
Year & Landings 1 & Landings 2 & Landings 3 & Landings 4 & Landings 5 & Total \\ 
  \hline
2005 & - & - & - & - & - & - \\ 
  2006 & - & - & - & - & - & - \\ 
  2007 & - & - & - & - & - & - \\ 
  2008 & - & - & - & - & - & - \\ 
  2009 & - & - & - & - & - & - \\ 
  2010 & - & - & - & - & - & - \\ 
  2011 & - & - & - & - & - & - \\ 
  2012 & - & - & - & - & - & - \\ 
  2013 & - & - & - & - & - & - \\ 
  2014 & - & - & - & - & - & - \\ 
   \hline
\end{tabular}
\end{table}

\FloatBarrier

\newpage

\hypertarget{data-and-assessment}{%
\subsection*{Data and Assessment}\label{data-and-assessment}}
\addcontentsline{toc}{subsection}{Data and Assessment}

This the first full assessment for Big Skate, which was last assessed as
part of the ``Other species'' Complex. This assessment uses the newest
version of Stock Synthesis (3.30.xx). The model begins in 1916, and
assumes the stock was at an unfished equilibrium that year.

(Figure \ref{fig:assess_region_map}).

\begin{figure}
\centering
\includegraphics{Figures/assess_region_map.png}
\caption{Map depicting the distribution of California scorpionfish out
to 600 ft. The stock assessment is bounded at Pt. Conception in the
north to the U.S./Mexico border in the south.
\label{fig:assess_region_map}}
\end{figure}

\FloatBarrier

\#\#Stock Biomass\{-\} (Figure \ref{fig:Spawnbio_all} and Table
\ref{tab:SpawningDeplete_mod1}).

The 2018 estimated spawning biomass relative to unfished equilibrium
spawning biomass is above the target of 40\% of unfished spawning
biomass at 99.8\% (95\% asymptotic interval: \(\pm\) 99.8\%-99.8\%)
(Figure \ref{fig:RelDeplete_all}). Approximate confidence intervals
based on the asymptotic variance estimates show that the uncertainty in
the estimated spawning biomass is high.

\FloatBarrier

\begin{table}[ht]
\centering
\caption{Recent trend in beginning of the 
                                      year spawning output and depletion for
                                      the model for Big Skate.} 
\label{tab:SpawningDeplete_mod1}
\begin{tabular}{l>{\centering}p{1.3in}>{\centering}p{1.2in}>{\centering}p{1in}>{\centering}p{1.2in}}
  \hline
Year & Spawning Output (million eggs) & \~{} 95\% confidence interval & Estimated depletion & \~{} 95\% confidence interval \\ 
  \hline
2010 & 70693.200 & (70693.2-70693.2) & 0.998 & (0.998-0.998) \\ 
  2011 & 70697.500 & (70697.5-70697.5) & 0.998 & (0.998-0.998) \\ 
  2012 & 70699.900 & (70699.9-70699.9) & 0.998 & (0.998-0.998) \\ 
  2013 & 70702.400 & (70702.4-70702.4) & 0.998 & (0.998-0.998) \\ 
  2014 & 70709.200 & (70709.2-70709.2) & 0.998 & (0.998-0.998) \\ 
  2015 & 70708.700 & (70708.7-70708.7) & 0.998 & (0.998-0.998) \\ 
  2016 & 70708.900 & (70708.9-70708.9) & 0.998 & (0.998-0.998) \\ 
  2017 & 70706.000 & (70706-70706) & 0.998 & (0.998-0.998) \\ 
  2018 & 70706.500 & (70706.5-70706.5) & 0.998 & (0.998-0.998) \\ 
  2019 & 70709.900 & (70709.9-70709.9) & 0.998 & (0.998-0.998) \\ 
   \hline
\end{tabular}
\end{table}

\FloatBarrier

\begin{figure}
\centering
\includegraphics{r4ss/plots_mod1/ts7_Spawning_output_with_95_asymptotic_intervals_intervals.png}
\caption{Time series of spawning biomass trajectory (circles and line:
median; light broken lines: 95\% credibility intervals) for the base
case assessment model. \label{fig:Spawnbio_all}}
\end{figure}

\begin{figure}
\centering
\includegraphics{r4ss/plots_mod1/ts9_Spawning_depletion_with_95_asymptotic_intervals_intervals.png}
\caption{Estimated relative depletion with approximate 95\% asymptotic
confidence intervals (dashed lines) for the base case assessment model.
\label{fig:RelDeplete_all}}
\end{figure}

\FloatBarrier

\hypertarget{recruitment}{%
\subsection*{Recruitment}\label{recruitment}}
\addcontentsline{toc}{subsection}{Recruitment}

Recruitment deviations were estimated from xxxx-xxxx (Figure
\ref{fig:Recruits_all} and Table \ref{tab:Recruit_mod1}).

\begin{table}[ht]
\centering
\caption{Recent recruitment for the model.} 
\label{tab:Recruit_mod1}
\begin{tabular}{>{\centering}p{.8in}>{\centering}p{1.6in}>{\centering}p{1.3in}}
  \hline
Year & Estimated Recruitment (millions) & \~{} 95\% confidence interval \\ 
  \hline
2010 & 749.57 & (749.57 - 749.57) \\ 
  2011 & 749.59 & (749.59 - 749.59) \\ 
  2012 & 749.60 & (749.6 - 749.6) \\ 
  2013 & 749.61 & (749.61 - 749.61) \\ 
  2014 & 749.64 & (749.64 - 749.64) \\ 
  2015 & 749.63 & (749.63 - 749.63) \\ 
  2016 & 749.63 & (749.63 - 749.63) \\ 
  2017 & 749.62 & (749.62 - 749.62) \\ 
  2018 & 749.62 & (749.63 - 749.63) \\ 
  2019 & 749.64 & (749.64 - 749.64) \\ 
   \hline
\end{tabular}
\end{table}

\FloatBarrier

\begin{figure}
\centering
\includegraphics{r4ss/plots_mod1/ts11_Age-0_recruits_(1000s)_with_95_asymptotic_intervals.png}
\caption{Time series of estimated Big Skate recruitments for the
base-case model with 95\% confidence or credibility intervals.
\label{fig:Recruits_all}}
\end{figure}

\FloatBarrier

\hypertarget{exploitation-status}{%
\subsection*{Exploitation status}\label{exploitation-status}}
\addcontentsline{toc}{subsection}{Exploitation status}

Harvest rates estimated by the base model \ldots{}.. management target
levels (Table \ref{tab:SPR_Exploit_mod1} and Figure \ref{fig:SPR_all}).

\FloatBarrier

\begin{table}[ht]
\centering
\caption{Recent trend in spawning potential 
                                        ratio and exploitation for Big Skate in the model.  Fishing intensity is (1-SPR) 
                                        divided by 50\% (the SPR target) and exploitation 
                                        is F divided by F\textsubscript{SPR}.} 
\label{tab:SPR_Exploit_mod1}
\begin{tabular}{l>{\centering}p{1in}>{\centering}p{1.2in}>{\centering}p{1in}>{\centering}p{1.2in}}
  \hline
Year & Fishing intensity & \~{} 95\% confidence interval & Exploitation rate & \~{} 95\% confidence interval \\ 
  \hline
2009 & 0.00 & (0-0) & 0.00 & (0-0) \\ 
  2010 & 0.00 & (0-0) & 0.00 & (0-0) \\ 
  2011 & 0.00 & (0-0) & 0.00 & (0-0) \\ 
  2012 & 0.00 & (0-0) & 0.00 & (0-0) \\ 
  2013 & 0.00 & (0-0) & 0.00 & (0-0) \\ 
  2014 & 0.00 & (0-0) & 0.00 & (0-0) \\ 
  2015 & 0.00 & (0-0) & 0.00 & (0-0) \\ 
  2016 & 0.00 & (0-0) & 0.00 & (0-0) \\ 
  2017 & 0.00 & (0-0) & 0.00 & (0-0) \\ 
  2018 & 0.00 & (0-0) & 0.00 & (0-0) \\ 
   \hline
\end{tabular}
\end{table}

\FloatBarrier

\begin{figure}
\centering
\includegraphics{r4ss/plots_mod1/SPR2_minusSPRseries.png}
\caption{Estimated spawning potential ratio (SPR) for the base-case
model. One minus SPR is plotted so that higher exploitation rates occur
on the upper portion of the y-axis. The management target is plotted as
a red horizontal line and values above this reflect harvests in excess
of the overfishing proxy based on the SPR\textsubscript{50\%} harvest
rate. The last year in the time series is 2018. \label{fig:SPR_all}}
\end{figure}

\FloatBarrier

\hypertarget{ecosystem-considerations}{%
\subsection*{Ecosystem Considerations}\label{ecosystem-considerations}}
\addcontentsline{toc}{subsection}{Ecosystem Considerations}

In this assessment, ecosystem considerations were not explicitly
included in the analysis.\\
This is primarily due to a lack of relevant data and results of analyses
(conducted elsewhere) that could contribute ecosystem-related
quantitative information for the assessment.

\hypertarget{reference-points}\)), and well above the minimum stock size
threshold (\(SB_{25\%}\)). The estimated relative depletion level for
the base model in 2019 is 99.8\% (95\% asymptotic interval: \(\pm\)
99.8\%-99.8\%, corresponding to an unfished spawning biomass of 70709.9
million eggs (95\% asymptotic interval: 70709.9-70709.9 million eggs) of
spawning biomass in the base model (Table \ref{tab:Ref_pts_mod1}).
Unfished age 1+ biomass was estimated to be 2,814 mt in the base case
model. The target spawning biomass (\(SB_{40\%}\)) is 2,834 million
eggs, which corresponds with an equilibrium yield of 5,906 mt.
Equilibrium yield at the proxy \(F_{MSY}\) harvest rate corresponding to
\(SPR_{50\%}\) is 5,070 mt (Figure \ref{fig:Yield_all}).

\FloatBarrier

\begin{table}[ht]
\centering
\caption{Summary of reference 
                                      points and management quantities for the 
                                      base case model.} 
\label{tab:Ref_pts_mod1}
\begin{tabular}{>{\raggedright}p{4.1in}>{\raggedleft}p{.62in}>{\raggedleft}p{.62in}>{\raggedleft}p{.62in}}
  \hline
\textbf{Quantity} & \textbf{Estimate} & \textbf{Low 2.5\%  limit} & \textbf{High 2.5\%  limit} \\ 
  \hline
Unfished spawning output (million eggs) & 7,086 & 7,086 & 7,086 \\ 
  Unfished age 1+ biomass (mt) & 2,814 & 2,814 & 2,814 \\ 
  Unfished recruitment ($R_{0}$) & 7,502 & 7,502 & 7,502 \\ 
  Spawning output(2018 million eggs) & 7,071 & 7,071 & 7,071 \\ 
  Depletion (2018) & 0.998 & 0.998 & 0.998 \\ 
  \textbf{$\text{Reference points based on } \mathbf{SB_{40\%}}$} &  &  &  \\ 
  Proxy spawning output ($B_{40\%}$) & 2,834 & 2,834 & 2,834 \\ 
  SPR resulting in $B_{40\%}$ ($SPR_{B40\%}$) & 0.625 & 0.625 & 0.625 \\ 
  Exploitation rate resulting in $B_{40\%}$ & 0.04 & 0.04 & 0.04 \\ 
  Yield with $SPR_{B40\%}$ at $B_{40\%}$ (mt) & 5,906 & 5,906 & 5,906 \\ 
  \textbf{\textit{Reference points based on SPR proxy for MSY}} &  &  &  \\ 
  Spawning output & 1,417 & 1,417 & 1,417 \\ 
  $SPR_{proxy}$ & 0.5 &  &  \\ 
  Exploitation rate corresponding to $SPR_{proxy}$ & 0.058 & 0.058 & 0.058 \\ 
  Yield with $SPR_{proxy}$ at $SB_{SPR}$ (mt) & 5,070 & 5,070 & 5,070 \\ 
  \textbf{\textit{Reference points based on estimated MSY values}} &  &  &  \\ 
  Spawning output at $MSY$ ($SB_{MSY}$) & 2,578 & 2,578 & 2,578 \\ 
  $SPR_{MSY}$ & 0.602 & 0.602 & 0.602 \\ 
  Exploitation rate at $MSY$ & 0.043 & 0.043 & 0.043 \\ 
  Dead Catch $MSY$ (mt) & 5,939 & 5,939 & 5,939 \\ 
  Retained Catch $MSY$ (mt) & 5,939 & 5,939 & 5,939 \\ 
   \hline
\end{tabular}
\end{table}

\FloatBarrier

\hypertarget{management-performance}{%
\subsection*{Management Performance}\label{management-performance}}
\addcontentsline{toc}{subsection}{Management Performance}

Table \ref{tab:mnmgt_perform}

\begin{table}[ht]
\centering
\caption{Recent trend in total catch and commercial 
                              landings (mt) relative to the management guidelines. 
                              Estimated total catch reflect the commercial landings 
                              plus the model estimated discarded biomass.} 
\label{tab:mnmgt_perform}
\scalebox{0.9}{
\begin{tabular}{>{\raggedleft}p{1in}>{\centering}p{1in}>{\centering}p{1in}>{\centering}p{1in}>{\centering}p{1in}}
  \hline
Year & OFL (mt; ABC prior to 2011) & ABC (mt) & ACL (mt; OY prior to 2011) & Estimated total catch (mt) \\ 
  \hline
\textbf{2007} & - & - & - & - \\ 
  \textbf{2008} & - & - & - & - \\ 
  \textbf{2009} & - & - & - & - \\ 
  \textbf{2010} & - & - & - & - \\ 
  \textbf{2011} & - & - & - & - \\ 
  \textbf{2012} & - & - & - & - \\ 
  \textbf{2013} & - & - & - & - \\ 
  \textbf{2014} & - & - & - & - \\ 
  \textbf{2015} & - & - & - & - \\ 
  \textbf{2016} & - & - & - & - \\ 
  \textbf{2017} & - & - & - & - \\ 
  \textbf{2018} & - & - & - & - \\ 
   \hline
\end{tabular}
}
\end{table}

\hypertarget{unresolved-problems-and-major-uncertainties}{%
\subsection*{Unresolved Problems and Major
Uncertainties}\label{unresolved-problems-and-major-uncertainties}}
\addcontentsline{toc}{subsection}{Unresolved Problems and Major
Uncertainties}

\FloatBarrier

\hypertarget{decision-table}{%
\subsection*{Decision Table}\label{decision-table}}
\addcontentsline{toc}{subsection}{Decision Table}

\begin{table}[ht]
\centering
\caption{Projections of potential OFL (mt) for 
                                        each model, using the base model forecast.} 
\label{tab:OFL_projection}
\begin{tabular}{lr}
  \hline
Year & OFL \\ 
  \hline
2019 & 158932.00 \\ 
  2020 & 149035.00 \\ 
  2021 & 141655.00 \\ 
  2022 & 136395.00 \\ 
  2023 & 132529.00 \\ 
  2024 & 129293.00 \\ 
  2025 & 126187.00 \\ 
  2026 & 122991.00 \\ 
  2027 & 119650.00 \\ 
  2028 & 116197.00 \\ 
  2029 & 112719.00 \\ 
  2030 & 109333.00 \\ 
   \hline
\end{tabular}
\end{table}
\begin{table}[ht]
\centering
\caption{Summary of 10-year 
                                             projections beginning in 2020 
                                             for alternate states of nature based on 
                                             an axis of uncertainty for the model.  Columns range over low, mid, and high
                                             states of nature, and rows range over different 
                                             assumptions of catch levels. An entry of "--" 
                                             indicates that the stock is driven to very low 
                                             abundance under the particular scenario.} 
\label{tab:Decision_table_mod1}
\scalebox{0.85}{
\begin{tabular}{l|cc|>{\centering}p{.7in}c|>{\centering}p{.7in}c|>{\centering}p{.7in}c}
   \multicolumn{3}{c}{}  &  \multicolumn{2}{c}{} 
                               & \multicolumn{2}{c}{\textbf{States of nature}} 
                               & \multicolumn{2}{c}{} \\
  \multicolumn{3}{c}{}  &  \multicolumn{2}{c}{Low M 0.05} 
                               & \multicolumn{2}{c}{Base M 0.07} 
                               &  \multicolumn{2}{c}{High M 0.09} \\
 \hline
 & Year & Catch & Spawning Output & Depletion & Spawning Output & Depletion & Spawning Output & Depletion \\ 
  \hline
 & 2019 & - & - & - & - & - & - & - \\ 
   & 2020 & - & - & - & - & - & - & - \\ 
   & 2021 & - & - & - & - & - & - & - \\ 
  40-10 Rule,  & 2022 & - & - & - & - & - & - & - \\ 
  Low M & 2023 & - & - & - & - & - & - & - \\ 
   & 2024 & - & - & - & - & - & - & - \\ 
   & 2025 & - & - & - & - & - & - & - \\ 
   & 2026 & - & - & - & - & - & - & - \\ 
   & 2027 & - & - & - & - & - & - & - \\ 
   & 2028 & - & - & - & - & - & - & - \\ 
   \hline
 & 2019 & - & - & - & - & - & - & - \\ 
   & 2020 & - & - & - & - & - & - & - \\ 
   & 2021 & - & - & - & - & - & - & - \\ 
  40-10 Rule & 2022 & - & - & - & - & - & - & - \\ 
   & 2023 & - & - & - & - & - & - & - \\ 
   & 2024 & - & - & - & - & - & - & - \\ 
   & 2025 & - & - & - & - & - & - & - \\ 
   & 2026 & - & - & - & - & - & - & - \\ 
   & 2027 & - & - & - & - & - & - & - \\ 
   & 2028 & - & - & - & - & - & - & - \\ 
   \hline
 & 2019 & - & - & - & - & - & - & - \\ 
   & 2020 & - & - & - & - & - & - & - \\ 
   & 2021 & - & - & - & - & - & - & - \\ 
  40-10 Rule, & 2022 & - & - & - & - & - & - & - \\ 
  High M & 2023 & - & - & - & - & - & - & - \\ 
   & 2024 & - & - & - & - & - & - & - \\ 
   & 2025 & - & - & - & - & - & - & - \\ 
   & 2026 & - & - & - & - & - & - & - \\ 
   & 2027 & - & - & - & - & - & - & - \\ 
   & 2028 & - & - & - & - & - & - & - \\ 
   \hline
 & 2019 & - & - & - & - & - & - & - \\ 
   & 2020 & - & - & - & - & - & - & - \\ 
   & 2021 & - & - & - & - & - & - & - \\ 
  Average & 2022 & - & - & - & - & - & - & - \\ 
  Catch & 2023 & - & - & - & - & - & - & - \\ 
   & 2024 & - & - & - & - & - & - & - \\ 
   & 2025 & - & - & - & - & - & - & - \\ 
   & 2026 & - & - & - & - & - & - & - \\ 
   & 2027 & - & - & - & - & - & - & - \\ 
   & 2028 & - & - & - & - & - & - & - \\ 
   \hline
\end{tabular}
}
\end{table}

\begin{sidewaystable}[ht]
\centering
\caption{Base case results summary.} 
\label{tab:base_summary}
\scalebox{0.6}{
\begin{tabular}{r>{\centering}p{1.1in}>{\centering}p{1.1in}>{\centering}p{1.1in}>{\centering}p{1.1in}>{\centering}p{1.1in}>{\centering}p{1.1in}>{\centering}p{1.1in}>{\centering}p{1.1in}>{\centering}p{1.1in}>{\centering}p{1.1in}}
  \hline
Quantity & 2010 & 2011 & 2012 & 2013 & 2014 & 2015 & 2016 & 2017 & 2018 & 2019 \\ 
  \hline
Landings (mt) &  &  &  &  &  &  &  &  &  &  \\ 
  Total Est. Catch (mt) &  &  &  &  &  &  &  &  &  &  \\ 
  OFL (mt) &  &  &  &  &  &  &  &  &  &  \\ 
  ACL (mt) &  &  &  &  &  &  &  &  &  &  \\ 
   \hline
(1-$SPR$)(1-$SPR_{50\%}$) &  0 &  0 &  0 &  0 &  0 &  0 &  0 &  0 &  0 &  \\ 
   \hline
Exploitation rate &  0 &  0 &  0 &  0 &  0 &  0 &  0 &  0 &  0 &  \\ 
  Age 1+ biomass (mt) & 2654110 & 2654240 & 2654360 & 2654400 & 2654430 & 2654570 & 2654490 & 2654470 & 2654390 & 2654450 \\ 
   \hline
Spawning Output & 70693.2 & 70697.5 & 70699.9 & 70702.4 & 70709.2 & 70708.7 & 70708.9 & 70706.0 & 70706.5 & 70709.9 \\ 
  ~95\% CI & (70693.2-70693.2) & (70697.5-70697.5) & (70699.9-70699.9) & (70702.4-70702.4) & (70709.2-70709.2) & (70708.7-70708.7) & (70708.9-70708.9) & (70706-70706) & (70706.5-70706.5) & (70709.9-70709.9) \\ 
   \hline
Depletion & 1 & 1 & 1 & 1 & 1 & 1 & 1 & 1 & 1 & 1 \\ 
  ~95\% CI & (0.998-0.998) & (0.998-0.998) & (0.998-0.998) & (0.998-0.998) & (0.998-0.998) & (0.998-0.998) & (0.998-0.998) & (0.998-0.998) & (0.998-0.998) & (0.998-0.998) \\ 
   \hline
Recruits & 749.57 & 749.59 & 749.60 & 749.61 & 749.64 & 749.63 & 749.63 & 749.62 & 749.62 & 749.64 \\ 
  ~95\% CI & (749.57 - 749.57) & (749.59 - 749.59) & (749.6 - 749.6) & (749.61 - 749.61) & (749.64 - 749.64) & (749.63 - 749.63) & (749.63 - 749.63) & (749.62 - 749.62) & (749.63 - 749.63) & (749.64 - 749.64) \\ 
   \hline
\end{tabular}
}
\end{sidewaystable}

\begin{figure}
\centering
\includegraphics{r4ss/plots_mod1/yield1_yield_curve.png}
\caption{Equilibrium yield curve for the base case model. Values are
based on the 2018 fishery selectivity and with steepness fixed at 0.718.
\label{fig:Yield_all}}
\end{figure}

\FloatBarrier

\newpage

\hypertarget{research-and-data-needs}{%
\subsection*{Research and Data Needs}\label{research-and-data-needs}}
\addcontentsline{toc}{subsection}{Research and Data Needs}

We recommend the following research be conducted before the next
assessment:

\begin{enumerate}

\item \textbf{xxxx}: 

\item \textbf{xxxx}:

\item \textbf{xxxx}:

\item \textbf{xxxx}:

\item \textbf{xxxx}:

\end{enumerate}

\FloatBarrier

\newpage
\renewcommand{\thefigure}{\arabic{figure}}
\renewcommand{\thetable}{\arabic{table}}
\setcounter{figure}{0}
\setcounter{table}{0}

\newpage
\renewcommand{\thefigure}{\arabic{figure}}
\renewcommand{\thetable}{\arabic{table}}
\setcounter{figure}{0}
\setcounter{table}{0}

\hypertarget{introduction}{%
\section{Introduction}\label{introduction}}

\begin{verbatim}
## \begin{table}[ht]
## \centering
## \begin{tabular}{rrrr}
##   \toprule
##  & id & var1 & var2 \\ 
##   \midrule
## 1 &   1 & -0.23 & 1.00 \\ 
##    \rowcolor[gray]{0.95}2 &   2 & -1.28 & 0.11 \\ 
##   3 &   3 & -0.71 & 0.77 \\ 
##    \rowcolor[gray]{0.95}4 &   4 & 0.74 & 0.88 \\ 
##   5 &   5 & -2.43 & 0.99 \\ 
##    \rowcolor[gray]{0.95}6 &   6 & 1.73 & 0.46 \\ 
##   7 &   7 & -0.33 & 0.29 \\ 
##    \rowcolor[gray]{0.95}8 &   8 & -0.53 & 0.11 \\ 
##   9 &   9 & -0.86 & 0.63 \\ 
##    \rowcolor[gray]{0.95}10 &  10 & -1.24 & 0.33 \\ 
##    \bottomrule
## \end{tabular}
## \end{table}
\end{verbatim}

\hypertarget{distribution-and-life-history}{%
\subsection{Distribution and Life
History}\label{distribution-and-life-history}}

Big Skate (\emph{Raja binoculata}) is the largest of the skate species
in North America with a documented maximum length of 244 cm total length
and a maximum weight of 91 kg (Eschmeyer and Herald
\protect\hyperlink{ref-Eschmeyer1983}{1983}). The species name
``binoculata'' (two-eyed) refers to the prominent ocellus at the base of
each pectoral fin. Big skate range from the Bering Sea to Cedros Island
in Baja California, but are uncommon south of Pt. Conception. Big skate
have a shallow depth distribution of 3-800 m, but are most common in the
3-110 m depth zone. Big Skate are observed in progressively shallower
water in the northern parts of its range. They occur in coastal bays,
estuaries, and over the continental shelf, usually on sandy or muddy
bottoms, but occasionally on low strands of kelp.

Skates are the largest and most widely distributed group of batoid fish
with approximately 245 species ascribed to two families (Ebert and
Compagno \protect\hyperlink{ref-Ebert2007biodiversity}{2007})(McEachran
and Miyake \protect\hyperlink{ref-McEachran1990}{1990}). Skates are
benthic fish that are found in all coastal waters but are most common in
cold temperatures and polar waters (Ebert and Compagno 2007).

There are eleven species of skates in three genera (Amblyraja,
Bathyraja, and Raja) present in the Northeast Pacific Ocean off
California, Oregon and Washington (Ebert 2003). Of that number, just
three species (Longnose Skate, \emph{Raja rhina}; Big Skate, \emph{Raja
binoculata}; and Sandpaper Skate, \emph{Bathyraja interrupta}) make up
over 95 percent of West Coast Groundfish Bottom Trawl Survey (WCGBTS)
catches in terms of biomass and numbers, with the Longnose Skate leading
in both categories (with 62 percent of biomass and 56 percent of
numbers).

Mating has been observed with distinct pairing with embrace. Big Skate
are oviparous and lay horned egg cases up to a foot in length with up to
seven embryos per egg case (Eschmeyer and Herald
\protect\hyperlink{ref-Eschmeyer1983}{1983}). The female deposits her
eggs in pairs on sandy or muddy flats; there is no discrete
breedingseason and egg-laying occurs year-round (Ebert 2003). Females
may use discrete spawning beds, as large numbers of egg cases have been
found in certain localized areas (IUCN/SSC Shark Specialist Group 2005).
The young emerge after 9 months and measure 18--23 cm (7--9 in).

Female Big Skates mature at 1.3--1.4 m (4 ft 3 in--4 ft 7 in) long and
12--13 years old, while males mature at 0.9--1.1 m (2 ft 11 in--3 ft 7
in) long and seven to eight years old (Bester, C.
\protect\hyperlink{ref-Bester2009}{2009}). The growth rate of Big Skates
in the Gulf of Alaska are comparable to those off California, but differ
from those off British Columbia. The lifespans of big skates off Alaska
are up to 15 years, while those off British Columbia are up to 26 years.

Big Skates are usually seen buried in sediment with only their eyes
showing. They feed on polychaete worms, mollusks, crustaceans, and small
benthic fishes. Polychaetes and mollusks comprise a slightly greater
percentage of the diet of younger individuals. The eyespots on the
skates' wings are believed to serve as decoys to confuse predators. A
known predator of big skates is the Broadnose Sevengill Shark
(\emph{Notorhynchus cepedianus}). Juvenile Northern Elephant Seals
(\emph{Mirounga angustirostris}) are known to consume the egg cases of
the Big Skate. Known parasites include the copepod \emph{Lepeophtheirus
cuneifer}.

\hypertarget{early-life-history}{%
\subsection{Early Life History}\label{early-life-history}}

Bizzarro.

\hypertarget{map}{%
\subsection{Map}\label{map}}

A map showing the scope of the assessment and depicting boundaries for
fisheries or data collection strata is provided in Figure
\ref{fig:boundary_map}.

\hypertarget{ecosystem-considerations-1}{%
\subsection{Ecosystem Considerations}\label{ecosystem-considerations-1}}

In this assessment, ecosystem considerations were not explicitly
included in the analysis. This is primarily due to a lack of relevant
data and results of analyses (conducted elsewhere) that could contribute
ecosystem-related quantitative information for the assessment.

\hypertarget{fishery-information}{%
\subsection{Fishery Information}\label{fishery-information}}

Big Skate are caught in commercial and recreational fisheries on the
West Coast using line and trawl gears. There is a limited market for
pectoral fins (skate wings).

The history of Big Skate (\emph{Raja binoculata}) is not well
documented. They were used as a food source by the native Coastal and
Salish Tribes (Batdorf, C \protect\hyperlink{ref-Batdorf1990}{1990})
long before Europeans settled in the Pacific Northwest and then as
fertilizer by the settlers (Bowers, G. M.
\protect\hyperlink{ref-Bowers1909}{1909}). No directed fishery for Big
Skate has been documented; rather, they were taken along with other
skates and rays as ``scrap fish'' and used for fertilizer, fish meal and
oil.

Skates have been regarded as a predator on desirable market species such
as Dungeness crab, and were thought of as nuisance fish with no appeal
as a food item save for small local markets. They had been discarded or
harvested at a minimal level until their livers became valued along with
those of other cartilaginous fishes for the extraction of vitamin A in
the 1940s. Chapman (Chapman, W.M.
\protect\hyperlink{ref-Chapman1944}{1944}) recorded that ``At present
they are being fished heavily, in common with the other elasmobranchs of
the coast, forthe vitamins in their livers. The carcasses are either
thrown away at sea or made into fish meal. Little use is made of the
excellent meat of the wings''.

Little information is available about the historic fishery for Big
Skate. In records before 2000, they are lumped together with other
skates or in market categories; this necessitates considerable attention
to reconstructing the fishery by observing the composition of catches in
the modern fishery and applying those to historical records.

\hypertarget{stock-status-and-management-history}{%
\subsection{Stock Status and Management
History}\label{stock-status-and-management-history}}

Big Skate were managed in the Other Fish complex until 2015 when they
were designated an Ecosystem Component (EC) species. Catches of Big
Skate are estimated to have averaged 95 mt from 2007--2011, along with
large landings of ``Unspecified Skate''. Analysis of Oregon
port-sampling data indicates that about 98 percent of the recent
Unspecified Skate landings in Oregon were comprised of Big Skate. Such
large landings indicates targeting of Big Skate has occurred and an EC
designation was not warranted. Based on this evidence, Big Skate was
redesignated as an actively-managed species in the fishery. Big skate
have been managed with stock-specific harvest specifications since 2017.

The recent OFL of 541 mt was calculated by applying approximate MSY
harvest rates toestimates of stock biomass from the Northwest Fisheries
Science Center (NWFSC) West Coast Groundfish Bottom Trawl Survey. This
survey-based biomass estimate is likely underestimated since Big Skate
are distributed all the way to the shoreline and no West Coast trawl
surveys have been conducted in water shallower than 55 meters. This
introduces an extra source of uncertainty to management and suggests
that increased precaution is needed to reduce the risk of overfishing
the stock.

There has been consideration for managing Big Skate in a complex with
Longnose Skate, the other actively-managed West Coast skate species, but
the two species have disparate distributions and fishery interactions
(Longnose Skate is much more deeply distributed than Big Skate) and that
option was not endorsed. The Pacific Fishery Management Council has
chosen to set the Annual Catch Limit (ACL) equal to the Allowable
Biological Catch (ABC) with a buffer for management uncertainty (P*) of
0.45.

\hypertarget{management-performance-1}{%
\subsection{Management Performance}\label{management-performance-1}}

Table \ref{tab:mnmgt_perform}

\hypertarget{fisheries-off-mexico-or-canada}{%
\subsection{Fisheries Off Mexico or
Canada}\label{fisheries-off-mexico-or-canada}}

Big Skate and Longnose Skate are landed in the commercial trawl and
hook-and-line fisheries in the waters off British Columbia. Assessments
of Longnose Skate and Big Skate were conducted by Canada's Division of
Fisheries and Oceans in 2015(King, J.R., Surry, A.M., Garcia, S., and
P.J. Starr \protect\hyperlink{ref-King2015}{2015}).

For Big Skate, a Bayesian surplus production model failed to provide
plausible results, and two data-limited approaches were investigated:
Depletion-Corrected Average Catch Analysis (DCAC), and a Catch-MSY
(maximum sustainable yield) Approach.

DCAC produced a range of potential yield estimates that were above the
long-term average catch, with an upper bound that was three orders of
magnitude larger than the long-term average catch. The Catch-MSY
approach was found to be quite sensitive to assumptions and was not
recommended as the sole basis of advice to managers.

The recommendation for managment for the two skate species was that they
should be managed with harvest yeilds based on mean historic catch, with
consideration given to survey trends and to the ranges of maximum
sustainable yield estimates identified by the Catch-MSY Approach.
However, the analysis found no significant trends in abundance indices
for Big Skate, and mean historical catches were below the maximum MSY
estimate from the catch-MSY results.

\newpage

\hypertarget{fishery-data}{%
\section{Fishery Data}\label{fishery-data}}

\hypertarget{data}{%
\subsection{Data}\label{data}}

Data used in the Big Skate assessment are summarized in Figure
\ref{fig:data_plot}. Descriptions of the data sources are in the
following sections.

\hypertarget{commercial-fishery-landings}{%
\subsection{Commercial Fishery
Landings}\label{commercial-fishery-landings}}

\hypertarget{catch-reconstructions-for-wa-or-and-ca}{%
\subsubsection{Catch reconstructions for WA, OR, and
CA}\label{catch-reconstructions-for-wa-or-and-ca}}

\textbf{\(\color{red}{\text{Washington Commercial Skate Landings Reconstruction}}\)}

Information for Big Skate is very limited, in part because the
requirement to sort landings of Big Skate in the shore-based Individual
Fishing Quota fishery from landings in the ``Unidentified Skate''
category was not implemented until June 2015. The historical catch of
Big Skate therefore relies on the historical reconstruction of Longnose
Skate.

For the 2019 assessment, a new approach has been developed for
estimating the catch history for Longnose Skate based on a linear
regression model that predicts the catch of Longnose Skate from the
catch of Dover sole, for which historical catch estimates are available
(Gertseva, V. \protect\hyperlink{ref-Gertseva2019}{2019}). The dependent
variable for the linear regression model was the West Coast Groundfish
Observer Program (WCGOP) annual estimates of the coastwide total catch
(landings plus discards) of Longnose Skate for the period 2009 to 2017
and the independent variable was the corresponding WCGOP annual
estimates of coastwide total catch (landings plus discards) of Dover
sole. The regression model has good predictive power
(R\textsuperscript{2} = 95.7\%) over the range of the Dover sole catches
(6,500 to 12,500 mt).

The discard component of the catch reconstruction for Big Skate may be
based either on the catch reconstruction for Longnose Skate and the
assumption that the two species experience similar discard rates
(discard / total catch) or on a similar analysis with links to species
that co-occur with big skate. Data from the Pikitch discard study
(1985-1987) and from WCGOP (2015-2017) support the idea that discard
rates for the two species are very similar. Also, market demand for
skates does not seem to distinguish between the two species. There are
insufficient years of data from the WCGOP to develop a regression model
for Big Skate as was done for Longnose Skate.

\textbf{Oregon Commercial Skate Landings Reconstruction}

Oregon Department of Fish and Wildlife (ODFW) provided newly
reconstructed commercial landings for all observed skate species for the
2019 assessment cycle (1978 -- 2018). In addition, the methods were
reviewed at a pre-assessment workshop. Historically, skates were landed
as a single skate complex in Oregon. In 2009, longnose skates were
separated into their own single-species landing category, and in 2014,
big skates were also separated. The reconstruction methodology differed
by these three time blocks in which species composition collections
diverged (1978 -- 2008; 2009 -- 2014; 2015 -- 2018).

Species compositions of skate complexes from commercial port sampling
are available throughout this time period but are generally limited,
which precluded the use of all strata for reconstructing landings.
Quarter and port were excluded, retaining gear type, PMFC area, and
market category for stratifying reconstructed landings within the three
time blocks. Bottom trawl gear types include multiple bottom trawl
gears, and account for greater than 98\% of skate landings . Minor gear
types include primarily bottom longline gear, but also include mid-water
trawl, hook and line, shrimp trawl, pot gear and scallop dredge.

For bottom trawl gears, trawl logbook areas and adjusted skate catches
were matched with strata-specific species compositions. In Time Block 1
(1978 -- 2008), all bottom trawl gear types were aggregated due to a
lack of specificity in the gear recorded on the fish tickets. However,
in Time Blocks 2 and 3, individual bottom trawl gear types were
retained. Some borrowing of species compositions was required (31\% of
strata) and when necessary, borrowed from the closest area or from the
most similar gear type . Longline gear landings were reconstructed in a
similar fashion as to bottom trawl and required some borrowing among
strata as well (25\%).

Due to insufficient species compositions, mid-water trawl landings were
reconstructed using a novel depth-based approach. Available compositions
indicate that the proportion by weight of big skates within a
composition drops to zero at approximately 100 fathoms, and an inverse
relationship is observed for longnose skate, where the proportion by
weight is consistently one beyond 100 -- 150 fathoms . Complex-level
landings were assigned a depth from logbook entries and these species
specific depth associations were used to parse out landings by species.
The approach differed somewhat by time block . Landings from shrimp
trawls were handled using a similar methodology. Finally, very minor
landings from hook and line, pot gear and scallop dredges were assigned
a single aggregated species composition, as they lack any gear-specific
composition samples. Landings from within a time block were apportioned
by year using the proportion of the annual ticket landings.

Results indicate that the species-specific landings from this
reconstruction are very similar to those from Oregon's commercial catch
reconstruction (Karnowski et al.~2014) during the overlapping years but
cover a greater time period with methodology more applicable to skates
in particular. ODFW intends to incorporate reconstructed skate landings
into PacFIN in the future (A. Whitman, ODFW; pers. comm.).

\textbf{California Catch Reconstruction}

A reconstruction of historical skate landings from California waters was
developed for the 1916--2017 time period using a combination of
commercial catch data (spatially explicit block summary catches and port
sample data from 2009-2017) and fishery-independent survey data
(Bizzarro, J. \protect\hyperlink{ref-Bizzarro2019}{2019}). Virtually all
landings in California were of ``unspecified skate'' until
species-composition sampling of skate market categories began in 2009.

From 2009 through 2017, catch estimates were based on these market
category species-composition samples, and the average of those
species-compositions was hindcast to 2002, based on the assumption that
those data were representative of the era of large area closures in the
post-2000 period.

For the period from 1936-1980, spatially explicit landings data (the
California Department of Fisheries and Wildlife (CDFW) block summary
data) were merged with survey data to provide species-specific
estimates.

For years 1981-2001, a ``blended'' product of these two approaches was
taken, in which a linear weighting scheme blended the two sets of catch
estimates through that period. Landings estimates were also scaled
upwards by an expansion factor for skates landed as ``dressed'' based on
fish ticket data. Prior to 1981 these data had not been reported and
skate landings were scaled by the ``average'' percentage landed as
dressed in the 1981-1985 time period, but by the late 1980s nearly all
skates were landed round.

As no spatial information on catch is available from 1916-1930, and the
block summary data were very sparse in the first few years of the CDFW
fish ticket program (1931--1934), spatial information from the late
1930's was used to hindcast to the 1916--1935 time period.

\hypertarget{tribal-catch-in-washington}{%
\subsubsection{Tribal Catch in
Washington}\label{tribal-catch-in-washington}}

\hypertarget{commercial-discards}{%
\subsubsection{Commercial Discards}\label{commercial-discards}}

Commercial discards of Big Skate are highly uncertain. The method used
to estimate discards for Longnose Skate was based on a strong
correlation between total mortality of that species, and total mortality
of Dover Sole for the years 2009--2017 during which Longnose were landed
separately from other skates. In contrast, the sorting requirement for
Big Skate occurred too recently to provide an adequate range of years
for this type of correlation. Furthermore, there is greater uncertainty
in the total mortality for the shallow-water species with which Big
Skate most often co-occurs, such as Sand Sole and Starry Flounder, than
there is for Dover Sole, which has been the subject of recurring stock
assessments.

However, those involved in the fishery for both skate species report
that discarding for Big Skate and Longnose Skate in the years prior to
1995 were driven by the same market forced and the discard rates were
similar. primarily lack of margets or fish processors accepting only
skate wings that had been separated at-sea, as well as the quantitative
have more uncertainty in their own catch estimates have no stock
assessment and more uncertain mortality estimated total mortality and
Dover Sole for which a correlation between relationship (Gertseva, V.
\protect\hyperlink{ref-Gertseva2019}{2019}),

\hypertarget{commercial-fishery-length-and-age-data}{%
\subsubsection{Commercial Fishery Length and Age
Data}\label{commercial-fishery-length-and-age-data}}

The input sample sizes were calculated via the Stewart Method (Ian
Stewart, personal communication, IPHC):

\begin{centering}

Input effN = $N_{\text{trips}} + 0.138 * N_{\text{fish}}$ if $N_{\text{fish}}/N_{\text{trips}}$ is $<$ 44

Input effN = $7.06 * N_{\text{trips}}$ if $N_{\text{fish}}/N_{\text{trips}}$ is $\geq$ 44

\end{centering}

\#\#\#Sport Fishery Removals and Discards

Biological samples from the recreational fleets are described in the
sections below.

\#\#\#Fishery-Dependent Indices of Abundance

\textbf{Data Source 1}

\emph{Data Source 1 Index Standardization}

\emph{Data Source 1 Length Composition}

\textbf{Data Source 2}

\textbf{Data Source 3}

\#\#\#Fishery-Independent Data Sources

\textbf{Alaska Fisheries Science Center (AFSC) Triennial Shelf Survey}\\
Research surveys have been used since the 1970s to provide
fishery-independent information about the abundance, distribution, and
biological characteristics of Big Skate. A coast-wide survey was
conducted in 1977 (Gunderson, Donald Raymond and Sample, Terrance M.
\protect\hyperlink{ref-Gunderson1980}{1980}) by the Alaska Fisheries
Science Center, and repeated every three years through 2001. The final
year of this survey, 2004, was conducted by the NWFSC according to the
AFSC protocol. We refer to this as the \textbf{Triennial Survey}.

The survey design used equally-spaced transects from which searches for
tows in a specific depth range were initiated. The depth range and
latitudinal range was not consistent across years, but all years in the
period 1980-2004 included the area from 40\(^\circ\) 10'N north to the
Canadian border and a depth range that included 55-366 meters, which
spans the range where the vast majority of Big Skate encountered in all
trawl surveys. Therefore the index was based on this depth range. The
survey as conducted in 1977 had incomplete coverage and is not believe
to be comparable to the later years, and is not used in the index.

An index of abundance was estimated based on the VAST delta-GLMM model
as described for the NWFSC Combo Index above. In this case as well, Q-Q
plots indicated slightly better performance of the gamma over lognormal
models for positive tows (Figure \ref{fig:VAST_QQ}).

\textbf{Northwest Fisheries Science Center West Coast Groundfish Bottom
Trawl Survey}

In 2003, the NWFSC took over an ongoing slope survey the AFSC had been
conducting, and expanded it spatially to include the continental shelf.
This survey, referred to in this document as the \textbf{NWFSC Combo
Survey}, has been conducted annually since. It uses a random-grid design
covering the coastal waters from a depth of 55 m to 1,280 m from
late-May to early-October (Bradburn, M.J. and Keller, A.A and Horness,
B.H. \protect\hyperlink{ref-Bradburn2011}{2011} , Keller, A.A. and
Wallace, J.R. and Methot, R.D.
\protect\hyperlink{ref-Keller2017}{2017}). Four chartered industry
vessels are used each year (with the exception of 2013 when the U.S.
federal-government shutdown curtailed the survey). Yellowtail catches in
the NWFSC Combo Survey are shown in \ref{fig:assess_region_map2}.

The data from the NWFSC Combo survey was analyzed using a
spatio-temporal delta-model (Thorson, J. T. and Shelton, A. O. and Ward,
E. J. and Skaug, H. J. \protect\hyperlink{ref-Thorson2015}{2015}),
implemented as an R package VAST (Thorson, James T. and Barnett, Lewis
A. K. \protect\hyperlink{ref-Thorson2017a}{2017}) and publicly available
online (\url{https://github.com/James-Thorson/VAST}). Spatial and
spatio-temporal variation is specifically included in both encounter
probability and positive catch rates, a logit-link for encounter
probability, and a log-link for positive catch rates. Vessel-year
effects were included for each unique combination of vessel and year in
the database.

\emph{Data Source 1 Index Standardization} VAST

\emph{Data Source 1 Length Composition}

\textbf{Triennial Survey} \emph{Data Source 2 Index Standardization}
VAST

\newpage

\#\#\#Biological Parameters and Data

\textbf{Measurement Details and Conversion Factors}

Disc width to total length (estimated by Ian on Apr 15, similar to Ebert
2008 estimates for Alaska) L = 1.3399 * W estimated from 95 samples from
WCGBTS where both measurements collected (R-squared = 0.9983). Little
sex difference observed, so using single relationship for both sexes.
Inter-spiracle width to total length from Downs \& Cheng (2013): L =
12.111 + 9.761\emph{ISW (females) L = 3.824 + 10.927}ISW (males)

Love et al.~(\protect\hyperlink{ref-Love1987}{1987})

\textbf{Length and Age Compositions}

Length comps (some based on widths)

WCGBTS Lengths from all years except 2006 and 2007 Widths in 2006 and
2007

Triennial Survey Sample sizes: 3 in 1998 (all widths), 84 in 2001 (3
widths, 81 lengths), 100 in 2004 (all lengths) Triennial survey About
90+ samples in each of 2001 and 2004 Only 3 unsexed fish from 1998

Commercial fisheries In process Discard comps from 2010-2015

Length compositions were provided from the following sources:

\begin{itemize}[noitemsep,nolistsep,topsep=0pt]
  \item Source 1 (\emph{type, e.g., commercial dead fish, research, recreational}, yyyy-yyyy)    
  \item Source 2 (\emph{type}, yyyy-yyyy)    
  \item Source 3 (\emph{research}, yyyy, yyyy, yyyy, yyyy) 
\end{itemize}

The length composition of all fisheries aggregated across time by fleet
is in Figure \ref{fig:comp_lendat_aggregated_across_time}. Descriptions
and details of the length composition data are in the above section for
each fleet or survey.

\vspace{.5cm}

\textbf{Age Structures}

von Bertalanffy growth curve (von Bertalanffy, L
\protect\hyperlink{ref-vonBertalanffy1938}{1938}),
\(L_i = L_{\infty}e^{(-k[t-t_0])}\), where \(L_i\) is the length (cm) at
age \(i\), \(t\) is age in years, \(k\) is rate of increase in growth,
\(t_0\) is the intercept, and \(L_{\infty}\) is the asymptotic length.

Ages WCGBTS Currently only 333 ages from 2010 present in data warehouse
as of Apr 15 Patrick submitting an 300 additional ages from 2016 and
2017 to Beth on Apr 2 and promised further additions during the week of
Apr 15.

Triennial Survey No ages

Commercial fisheries 2009 samples from WA were stratified by length, so
should be treated as conditionals

\vspace{.5cm}

\textbf{Aging Precision and Bias}

\vspace{.5cm}

\textbf{Weight-Length}

Estimated by Ian based on WCGBT samples (n = 1159)
\(Weight = 0.0000074924 * Length ^ 2.9925\) (Figure
\ref{fig:weight-length}).

\vspace{.5cm}

\textbf{Sex Ratio, Maturity, and Fecundity}

The female maturity relationship was based on visual maturity estimates
from port samplers (n = 278, of which 241 were from Oregon and 37 from
Washington, with 24 mature specimens) as well as 55 samples from the
WCGBTS (of which 4 were mature). The resulting relationship was
\(L_{50\%} = 148.2453\) with a slope parameter of \(Beta = -0.13155\) in
the relationship \(M = (1 + Beta(L - L_{50\%}))^{-1}\) (Figure
\ref{fig:maturity}).

\vspace{.5cm}

\textbf{Natural Mortality}

The Hamel prior for M is lognormal(ln(5.4/max age),.438), which based on
1 age-15 fish out of 1034 observed in the WCGBTS results in lognormal(
-1.021651, 0.438)

If it needs to be fixed, it should be set to M = 5.4/max age = 5.4/15 =
0.36

\vspace{.5cm}

\#\#\#Environmental or Ecosystem Data Included in the Assessment In this
assessment, neither environmental nor ecosystem considerations were
explicitly included in the analysis. This is primarily due to a lack of
relevant data and results of analyses (conducted elsewhere) that could
contribute ecosystem-related quantitative information for the
assessment.

\newpage

\#\#Previous Assessments

\#\#\#History of Modeling Approaches Used for this Stock

Deriving estimates of OFL for species in the ``Other Fish'' complex or
potential alternative complexes

The current ``Other Fish'' complex and proposed alternatives include a
number of species for which estimates of OFL contributions are not
available from stock assessments or data-poor methods. Four of the
species had OFL contributions for the 2013--2014 management cycle
calculated by applying approximate MSY harvest rates to estimates of
stock biomass from the NWFSC West Coast Bottom Trawl Survey (Bradburn et
al., 2012). This approach is described in detail in Cope et al.~(2012).

\#\#\#yyyy Assessment Recommendations

\begin{description}[style=unboxed]

  \item[Recommendation 1: ] \hfill \\

   STAT response: xxxxx

\item[Recommendation 2: ] \hfill \\

  STAT response: xxxxx

\item[Recommendation 3: ] \hfill \\

  STAT response: xxxx

  
\end{description}

\#\#Model Description

\#\#\#Transition to the Current Stock Assessment

\#\#\#Summary of Data for Fleets and Areas There are xxx fleets in the
base model. They include:

\emph{Commercial}: The commercial fleets include \ldots{}

\emph{Recreational}: The recreational fleets include \ldots{}

\emph{Research}: There are xx sources of fishery-independent data
available \ldots{}

\#\#\#Other Specifications

\#\#\#Modeling Software The STAT team used Stock Synthesis 3 version
3.30.05.03 by Dr.~Richard Methot at the NWFSC. This most recent version
was used, since it included improvements and corrections to older
versions. The r4SS package (GitHub release number v1.27.0) was used to
post-processing output data from Stock Synthesis.

\#\#\#Data Weighting

\#\#\#Priors The log-normal prior for female natural mortality were
based on a meta-analysis completed by Hamel
(\protect\hyperlink{ref-Hamel2015}{2015}), as described under ``Natural
Mortality.'' Female natural mortality was fixed at the median of the
prior, 0.xxx for an assumed maximum age of xx. An uninformative prior
was used for the male offset natural mortality, which was estimated.

The prior for steepness (\emph{h}) assumes a beta distribution with
parameters based on an update for the Thorson-Dorn rockfish prior (Dorn,
M. and Thorson, J., pers. comm.), which was endorsed by the Science and
Statistical Committee in 2018. The prior is a beta distribution with
\(mu\)=0.xxx and \(sigma\)=0.xxx. Steepness is fixed in the base model
at the mean of the prior. The priors were applied in sensitivity
analyses where these parameters were estimated.

\#\#\#Estimated and Fixed Parameters A full list of all estimated and
fixed parameters is provided in Tables \ref{tab:model_params}.

The base model has a total of xxx estimated parameters in the following
categories:

\begin{itemize}
  \item xxx,
  \item xxx
  \item xxx, and
  \item xxx selectivity parameters
\end{itemize}

The estimated parameters are described in greater detail below and a
full list of all estimated and parameters is provided in Table
\ref{tab:model_params}.

\emph{Growth.}

\emph{Natural Mortality.}

\emph{Selectivity.}

\emph{Other Estimated Parameters.}

\emph{Other Fixed Parameters.}

\#\#Model Selection and Evaluation \#\#\#Key Assumptions and Structural
Choices

\#\#\#Alternate Models Considered

\#\#\#Convergence

\#\#Response to the Current STAR Panel Requests

\begin{description}[style=sameline]

\item[Request No. 1: ] \hfill \\
  
\textbf{Rationale:} xxx   
    
\textbf{STAT Response:} xxx


\item[Request No. 2: ] \hfill \\


\textbf{Rationale:} xxx 


\textbf{STAT Response:} xxx
    

\item[Request No. 3: ] \hfill \\

\textbf{Rationale:} x.  
    
  
\textbf{STAT Response:} xxx

\item[Request No. 4: ] \hfill \\

\textbf{Rationale:} xxx 
    
    
\textbf{STAT Response:} xxx


\item[Request No. 5: ] \hfill \\

\textbf{Rationale:} xxx
  
\textbf{STAT Response:} xxx  
    


\end{description}

\#\#Base Case Model Results The following description of the model
results reflects a base model that incorporates all of the changes made
during the STAR panel (see previous section). The base model parameter
estimates and their approximate asymptotic standard errors are shown in
Table \ref{tab:model_params} and the likelihood components are in Table
\ref{tab:like_components}. Estimates of derived reference points and
approximate 95\% asymptotic confidence intervals are shown in Table
\ref{tab:Ref_pts_mod1}. Time-series of estimated stock size over time
are shown in Table \ref{tab:Timeseries_mod1}.

\#\#\#Parameter Estimates

The additional survey variability (process error added directly to each
year's input variability) for all surveys was estimated within the
model.

(Figure
\ref{fig:ts11_Age-0_recruits_(1000s)_with_95_asymptotic_intervals} ).

The stock-recruit curve \ldots{} Figure \ref{fig:SR_curve2} with
estimated recruitments also shown.

\#\#\#Fits to the Data Model fits to the indices of abundance, fishery
length composition, survey length composition, and conditional
age-at-length observations are all discussed below.

\#\#\#Uncertainty and Sensitivity Analyses A number of sensitivity
analyses were conducted, including:

\begin{enumerate}

  \item Sensitivity 1
  
  \item Sensitivity 2
  
  \item Sensitivity 3
  
  \item Sensitivity 4
  
  \item Sensitivity 5, etc/
  
  
\end{enumerate}

\#\#\#Retrospective Analysis

\#\#\#Likelihood Profiles

\#\#\#Reference Points Reference points were calculated using the
estimated selectivities and catch distribution among fleets in the most
recent year of the model, (2017). Sustainable total yield (landings plus
discards) were 5,070 mt when using an \(SPR_{50\%}\) reference harvest
rate and with a 95\% confidence interval of 5,070 mt based on estimates
of uncertainty. The spawning biomass equivalent to 40\% of the unfished
level (\(SB_{40\%}\)) was 2,834 mt.

(Figure
\ref{fig:ts7_Spawning_biomass_(mt)_with_95_asymptotic_intervals_intervals}

The 2018 spawning biomass relative to unfished equilibrium spawning
biomass is above/below the target of 40\% of unfished levels (Figure
\ref{fig:ts9_Spawning_depletion_with_95_asymptotic_intervals_intervals}).
The relative fishing intensity, \((1-SPR)/(1-SPR_{50\%})\), has been xxx
the management target for the entire time series of the model.

Table \ref{tab:Ref_pts_mod1} shows the full suite of estimated reference
points for the base model and Figure \ref{fig:yield1_yield_curve} shows
the equilibrium curve based on a steepness value xxx.

\newpage

\#Harvest Projections and Decision Tables The forecasts of stock
abundance and yield were developed using the final base model, with the
forecasted projections of the OFL presented in Table
\ref{tab:OFL_projection}.

The forecasted projections of the OFL for each model are presented in
Table \ref{tab:Decision_table_mod1}.

\newpage

\#Regional Management Considerations \newpage

\#Research Needs There are a number of areas of research that could
improve the stock assessment for Big Skate. Below are issues identified
by the STAT team and the STAR panel:

\begin{enumerate}

\item \textbf{xxxx}: 

\item \textbf{xxxx}:

\item \textbf{xxxx}:

\item \textbf{xxxx}:

\item \textbf{xxxx}:

\end{enumerate}

\#Acknowledgments

\newpage
\FloatBarrier
\newpage

\FloatBarrier

\FloatBarrier

\FloatBarrier
\newpage

\newpage
\FloatBarrier

\FloatBarrier

\FloatBarrier

\FloatBarrier

\newpage

\FloatBarrier

\newpage

\#Figures

\begin{figure}
\centering
\includegraphics{Figures/boundary_map.png}
\caption{Map showing the state boundary lines for management of the
recreational fishing fleets \label{fig:boundary_map}}
\end{figure}

\begin{figure}
\centering
\includegraphics{r4ss/plots_mod1/data_plot.png}
\caption{Summary of data sources used in the model.
\label{fig:data_plot}}
\end{figure}

\FloatBarrier

\FloatBarrier

\FloatBarrier

\FloatBarrier

\FloatBarrier

\FloatBarrier

\FloatBarrier

\newpage

\begin{figure}
\centering
\includegraphics{r4ss/plots_mod1/comp_lendat__aggregated_across_time.png}
\caption{Length comp data, aggregated across time by fleet. Labels
`retained' and `discard' indicate discarded or retained sampled for each
fleet. Panels without this designation represent the whole catch.
\label{fig:comp_lendat_aggregated_across_time}}
\end{figure}

\begin{figure}
\centering
\includegraphics{Figures/Big Skate bio relationships.png}
\caption{Estimated relationship between length and weight (left) and
disc-width and length (right) for Big Skate. Colored points show
observed values and the black line indicates the estimated relationship
\(W = 0.0000074924L^{2.9925}\).\label{fig:weight-length}}
\end{figure}

\begin{figure}
\centering
\includegraphics{Figures/BigSkate_maturity.png}
\caption{Estimated maturity relationship for female Big Skate. Gray
points indicate average observed functional maturity within each length
bin with point size proportional to the number of samples (indicated by
text within each point).\label{fig:maturity}}
\end{figure}

\newpage

\FloatBarrier

\FloatBarrier

\FloatBarrier

\FloatBarrier

\begin{figure}
\centering
\includegraphics{r4ss/plots_mod1/sel01_multiple_fleets_length1.png}
\caption{Selectivity at length for all of the fleets in the base model.
\label{fig:sel01_multiple_fleets_length1}}
\end{figure}

\FloatBarrier

\begin{figure}
\centering
\includegraphics{r4ss/plots_mod1/ts11_Age-0_recruits_(1000s)_with_95_asymptotic_intervals.png}
\caption{Estimated time-series of recruitment for Big Skate.
\label{fig:ts11_Age-0_recruits_(1000s)_with_95_asymptotic_intervals}}
\end{figure}

\begin{figure}
\centering
\includegraphics{r4ss/plots_mod1/SR_curve2.png}
\caption{Estimated recruitment (red circles) and the assumed
stock-recruit relationship (black line) for Big Skate. The green line
shows the effect of the bias correction for the lognormal distribution.
\label{fig:SR_curve2}}
\end{figure}

\FloatBarrier

\begin{figure}
\centering
\includegraphics{./r4ss/plots_mod1/comp_condAALfit_Andre_plotsflt1mkt2_page1.png}
\caption{Conditional AAL plot, retained, Fishery\_current (plot 1 of 2)
These plots show mean age and std. dev. in conditional AAL. Left plots
are mean AAL by size\_class (obs. and pred.) with 90\% CIs based on
adding 1.64 SE of mean to the data. Right plots in each pair are SE of
mean AAL (obs. and pred.) with 90\% CIs based on the chi\_square
distribution.
\label{fig:mod1_4_comp_condAALfit_Andre_plotsflt1mkt2_page1}}
\end{figure}

\includegraphics{./r4ss/plots_mod1/comp_condAALfit_Andre_plotsflt1mkt2_page2.png}

\begin{center} 

              Figure continued from previous page 

             \end{center}

\begin{figure}
\centering
\includegraphics{./r4ss/plots_mod1/comp_condAALfit_data_weighting_TA1.8_condAgeWCGBTS.png}
\caption{Mean age from conditional data (aggregated across length bins)
for WCGBTS with 95\% confidence intervals based on current samples
sizes. Francis data weighting method TA1.8: thinner intervals (with
capped ends) show result of further adjusting sample sizes based on
suggested multiplier (with 95\% interval) for conditional
age\_at\_length data from WCGBTS: 1.3806 (0.8289\_39.92) For more info,
see Francis, R.I.C.C. (2011). Data weighting in statistical fisheries
stock assessment models. Can. J. Fish. Aquat. Sci. 68: 1124\_1138.
\label{fig:mod1_6_comp_condAALfit_data_weighting_TA1.8_condAgeWCGBTS}}
\end{figure}

\begin{figure}
\centering
\includegraphics{./r4ss/plots_mod1/comp_condAALfit_Andre_plotsflt5mkt0_page1.png}
\caption{Conditional AAL plot, whole catch, WCGBTS (plot 1 of 2) These
plots show mean age and std. dev. in conditional AAL. Left plots are
mean AAL by size\_class (obs. and pred.) with 90\% CIs based on adding
1.64 SE of mean to the data. Right plots in each pair are SE of mean AAL
(obs. and pred.) with 90\% CIs based on the chi\_square distribution.
\label{fig:mod1_7_comp_condAALfit_Andre_plotsflt5mkt0_page1}}
\end{figure}

\includegraphics{./r4ss/plots_mod1/comp_condAALfit_Andre_plotsflt5mkt0_page2.png}

\begin{center} 

              Figure continued from previous page 

             \end{center}

\FloatBarrier

\FloatBarrier

\FloatBarrier

\FloatBarrier

\begin{figure}
\centering
\includegraphics{r4ss/plots_mod1/ts7_Spawning_output_with_95_asymptotic_intervals_intervals.png}
\caption{Estimated spawning biomass (mt) with approximate 95\%
asymptotic intervals.
\label{fig:ts7_Spawning_biomass_(mt)_with_95_asymptotic_intervals_intervals}}
\end{figure}

\begin{figure}
\centering
\includegraphics{r4ss/plots_mod1/ts9_Spawning_depletion_with_95_asymptotic_intervals_intervals.png}
\caption{Estimated spawning depletion with approximate 95\% asymptotic
intervals.
\label{fig:ts9_Spawning_depletion_with_95_asymptotic_intervals_intervals}}
\end{figure}

\begin{figure}
\centering
\includegraphics{r4ss/plots_mod1/yield1_yield_curve.png}
\caption{Equilibrium yield curve for the base case model. Values are
based on the 2018 fishery selectivity and with steepness fixed at 0.718.
\label{fig:yield1_yield_curve}}
\end{figure}

\FloatBarrier

\newpage

\FloatBarrier
\newpage

\#Appendix A. Detailed fits to length composition data \{-\}
\renewcommand{\thepage}{A-\arabic{page}}

\renewcommand{\thefigure}{A\arabic{figure}}
\setcounter{page}{1}

\begin{figure}
\centering
\includegraphics{./r4ss/plots_mod1/comp_lenfit_flt1mkt2.png}
\caption{Length comps, retained, Fishery\_current. `N adj.' is the input
sample size after data\_weighting adjustment. N eff. is the calculated
effective sample size used in the McAllister\_Iannelli tuning method.
\label{fig:mod1_1_comp_lenfit_flt1mkt2}}
\end{figure}

\begin{figure}
\centering
\includegraphics{./r4ss/plots_mod1/comp_lenfit_flt1mkt1.png}
\caption{Length comps, discard, Fishery\_current. `N adj.' is the input
sample size after data\_weighting adjustment. N eff. is the calculated
effective sample size used in the McAllister\_Iannelli tuning method.
\label{fig:mod1_2_comp_lenfit_flt1mkt1}}
\end{figure}

\begin{figure}
\centering
\includegraphics{./r4ss/plots_mod1/comp_lenfit_flt5mkt0.png}
\caption{Length comps, whole catch, WCGBTS. `N adj.' is the input sample
size after data\_weighting adjustment. N eff. is the calculated
effective sample size used in the McAllister\_Iannelli tuning method.
\label{fig:mod1_3_comp_lenfit_flt5mkt0}}
\end{figure}

\begin{figure}
\centering
\includegraphics{./r4ss/plots_mod1/comp_lenfit_flt6mkt0.png}
\caption{Length comps, whole catch, Triennial. `N adj.' is the input
sample size after data\_weighting adjustment. N eff. is the calculated
effective sample size used in the McAllister\_Iannelli tuning method.
\label{fig:mod1_4_comp_lenfit_flt6mkt0}}
\end{figure}

\begin{figure}
\centering
\includegraphics{./r4ss/plots_mod1/comp_lenfit_flt7mkt2.png}
\caption{Length comps, retained, IPHC. `N adj.' is the input sample size
after data\_weighting adjustment. N eff. is the calculated effective
sample size used in the McAllister\_Iannelli tuning method.
\label{fig:mod1_5_comp_lenfit_flt7mkt2}}
\end{figure}

\color{black}

\#References\{-\}

\renewcommand{\thepage}{}

\hypertarget{refs}{}
\leavevmode\hypertarget{ref-Batdorf1990}{}%
Batdorf, C. 1990. Northwest Native Harvest. Hancock House Publishers
Ltd.; Surrey, B.C., Canada.

\leavevmode\hypertarget{ref-Bester2009}{}%
Bester, C. 2009. Biological Profiles: Big Skate. Florida Museum of
Natural History Ichthyology Department.

\leavevmode\hypertarget{ref-Bizzarro2019}{}%
Bizzarro, J. 2019. Manuscript in preparation.

\leavevmode\hypertarget{ref-Bowers1909}{}%
Bowers, G. M. 1909. Report of The Commissioner For the Year Ending June
30, 1909. Part XXVIII. Washington Printing Office.

\leavevmode\hypertarget{ref-Bradburn2011}{}%
Bradburn, M.J. and Keller, A.A and Horness, B.H. 2011. The 2003 to 2008
US West Coast bottom trawl surveys of groundfish resources off
Washington, Oregon, and California: estimates of distribution,
abundance, length, and age composition. NOAA Technical Memorandum NMFS
NOAA-TM-NMFS-NWFSC-114: 323 pp.

\leavevmode\hypertarget{ref-Chapman1944}{}%
Chapman, W.M. 1944. The Latent Fisheries of Washington and Alaska.
Washington State Department of Fisheries.

\leavevmode\hypertarget{ref-Ebert2007biodiversity}{}%
Ebert, D.A., and Compagno, L.J. 2007. Biodiversity and systematics of
skates (chondrichthyes: Rajiformes: Rajoidei). \emph{In} Biology of
skates. Springer. pp. 5--18.

\leavevmode\hypertarget{ref-Eschmeyer1983}{}%
Eschmeyer, W.N., and Herald, E.S. 1983. A field guide to pacific coast
fishes: North america. Houghton Mifflin Harcourt.

\leavevmode\hypertarget{ref-Gertseva2019}{}%
Gertseva, V. 2019. Manuscript in preparation.

\leavevmode\hypertarget{ref-Gunderson1980}{}%
Gunderson, Donald Raymond and Sample, Terrance M. 1980. Distribution and
abundance of rockfish off Washington, Oregon and California during 1977.
Northwest and Alaska Fisheries Center, National Marine Fisheries
Service. Available from
\href{\%7Bhttp://spo.nmfs.noaa.gov/mfr423-4/mfr423-42.pdf\%7D}{\{http://spo.nmfs.noaa.gov/mfr423-4/mfr423-42.pdf\}}.

\leavevmode\hypertarget{ref-Hamel2015}{}%
Hamel, Owen S. 2015. A method for calculating a meta-analytical prior
for the natural mortality rate using multiple life history correlates.
ICES Journal of Marine Science: Journal du Conseil \textbf{72}(1):
62--69. doi:
\href{https://doi.org/\%7B10.1093/icesjms/fsu131\%7D}{\{10.1093/icesjms/fsu131\}}.

\leavevmode\hypertarget{ref-Keller2017}{}%
Keller, A.A. and Wallace, J.R. and Methot, R.D. 2017. The Northwest
Fisheries Science Center's West Coast Groundfish Bottom Trawl Survey:
History, Design, and Description. NOAA Technical Memorandum NMFS
NOAA-TM-NMFS-NWFSC-136: 38 pp.

\leavevmode\hypertarget{ref-King2015}{}%
King, J.R., Surry, A.M., Garcia, S., and P.J. Starr. 2015. Big skate
(Raja binoculata) and longnose skate (R. rhina) stock assessments for
British Columbia. Ottawa : Canadian Science Advisory Secretariat.

\leavevmode\hypertarget{ref-Love1987}{}%
Love, Milton S and Axell, Brita and Morris, Pamela and Collins, Robson
and Brooks, Andrew. 1987. Life history and fishery of the California
scorpionfish,\\
emphScorpaena guttata, within the Southern California Bight. Fishery
Bulletin \textbf{85}: 99--116.

\leavevmode\hypertarget{ref-McEachran1990}{}%
McEachran, J., and Miyake, T. 1990. 1990. Zoogeography and bathymetry of
skates (chondrichthyes, rajidae). Elasmobranchs as living resources.
Advances in biology, Ecology, Systematics and the status of the
fisheries: 305--326.

\leavevmode\hypertarget{ref-Thorson2017a}{}%
Thorson, James T. and Barnett, Lewis A. K. 2017. Comparing estimates of
abundance trends and distribution shifts using single- and multispecies
models of fishes and biogenic habitat. ICES Journal of Marine Science:
Journal du Conseil: fsw193. doi:
\href{https://doi.org/\%7B10.1093/icesjms/fsw193\%7D}{\{10.1093/icesjms/fsw193\}}.

\leavevmode\hypertarget{ref-Thorson2015}{}%
Thorson, J. T. and Shelton, A. O. and Ward, E. J. and Skaug, H. J. 2015.
Geostatistical delta-generalized linear mixed models improve precision
for estimated abundance indices for West Coast groundfishes. ICES
Journal of Marine Science \textbf{72}(5): 1297--1310. doi:
\href{https://doi.org/\%7B10.1093/icesjms/fsu243\%7D}{\{10.1093/icesjms/fsu243\}}.

\leavevmode\hypertarget{ref-vonBertalanffy1938}{}%
von Bertalanffy, L. 1938. A quantitative theory of organic growth. Human
Biology \textbf{10}: 181--213.

\end{document}
